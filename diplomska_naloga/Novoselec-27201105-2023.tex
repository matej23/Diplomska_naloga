\documentclass[mat1, tisk]{fmfdelo}
% \documentclass[fin1, tisk]{fmfdelo}
% Če pobrišete možnost tisk, bodo povezave obarvane,
% na začetku pa ne bo praznih strani po naslovu, …

%%%%%%%%%%%%%%%%%%%%%%%%%%%%%%%%%%%%%%%%%%%%%%%%%%%%%%%%%%%%%%%%%%%%%%%%%%%%%%%
% METAPODATKI
%%%%%%%%%%%%%%%%%%%%%%%%%%%%%%%%%%%%%%%%%%%%%%%%%%%%%%%%%%%%%%%%%%%%%%%%%%%%%%%

% - vaše ime
\avtor{Matej Novoselec}

% - naslov dela v slovenščini
\naslov{Schwarzov princip zrcaljenja za harmonične funkcije}

% - naslov dela v angleščini
\title{Schwarz Reflection Principle for Harmonic Functions}

% - ime mentorja/mentorice s polnim nazivom:
%   - doc.~dr.~Ime Priimek
%   - izr.~prof.~dr.~Ime Priimek
%   - prof.~dr.~Ime Priimek
%   za druge variante uporabite ustrezne ukaze
%\mentor{prof. dr. Barbara Drinovec Drnovšek}
% \somentor{...}
\mentorica{prof. dr. Barbara Drinovec Drnovšek}
% \somentorica{...}
% \mentorja{...}{...}
% \somentorja{...}{...}
% \mentorici{...}{...}
% \somentorici{...}{...}

% - leto diplome
\letnica{2023} 

% - povzetek v slovenščini
%   V povzetku na kratko opišite vsebinske rezultate dela. Sem ne sodi razlaga
%   organizacije dela, torej v katerem razdelku je kaj, pač pa le opis vsebine.
\povzetek{TU BO POVZETEK}

% - povzetek v angleščini
\abstract{TU BO POVZETEK V ANG}

% - klasifikacijske oznake, ločene z vejicami
%   Oznake, ki opisujejo področje dela, so dostopne na strani https://www.ams.org/msc/
\klasifikacija{..., ...}

% - ključne besede, ki nastopajo v delu, ločene s \sep
\kljucnebesede{...\sep ...}

% - angleški prevod ključnih besed
\keywords{...\sep ...} % angleški prevod ključnih besed

% - angleško-slovenski slovar strokovnih izrazov
\slovar{
% \geslo{angleški izraz}{slovenski izraz}
% ...
}

% - ime datoteke z viri (vključno s končnico .bib), če uporabljate BibTeX
% \literatura{....bib}

%%%%%%%%%%%%%%%%%%%%%%%%%%%%%%%%%%%%%%%%%%%%%%%%%%%%%%%%%%%%%%%%%%%%%%%%%%%%%%%
% DODATNE DEFINICIJE
%%%%%%%%%%%%%%%%%%%%%%%%%%%%%%%%%%%%%%%%%%%%%%%%%%%%%%%%%%%%%%%%%%%%%%%%%%%%%%%

% naložite dodatne pakete, ki jih potrebujete
% \usepackage{...}
\usepackage{graphicx}
\usepackage{amsmath}
\usepackage{yhmath}
\usepackage{float}
\usepackage[shortlabels]{enumitem}
\usepackage[justification=centering]{caption}

% deklarirajte vse matematične operatorje, da jih bo LaTeX pravilno stavil
% \DeclareMathOperator{\...}{...}

% vstavite svoje definicije ...
% \newcommand{\...}{...}
\newcommand{\R}{\mathbb R}
\newcommand{\N}{\mathbb N}
\newcommand{\Z}{\mathbb Z}
\newcommand{\C}{\mathbb C}
\newcommand{\Q}{\mathbb Q}

%%%%%%%%%%%%%%%%%%%%%%%%%%%%%%%%%%%%%%%%%%%%%%%%%%%%%%%%%%%%%%%%%%%%%%%%%%%%%%%
% ZAČETEK VSEBINE
%%%%%%%%%%%%%%%%%%%%%%%%%%%%%%%%%%%%%%%%%%%%%%%%%%%%%%%%%%%%%%%%%%%%%%%%%%%%%%%

\begin{document}

\section{Uvod}
V diplomski nalogi bomo spoznali osnovne lastnosti harmoničnih fukcij, ki jih bomo proti koncu s pridom uporabili za dokaz glavnega izreka, katerega ime nosi naslov naloge.
Vlekli bomo številne vzporednice s kompleksno analizo, gre namreč za področje, močno povezano s študijo holomorfnih funkcij.

V prvem poglavju bomo spoznali, kaj so harmonične funkcije in poudarili, katere njihove lastnosti bodo za nadaljevanje pomembne. Ogledali si bomo tudi njihov odnos s holomorfnimi funkcijami. 
V drugem poglavju bomo spoznali Dirichletov problem za enotski disk, ki nam bo dal osnovo za definicijo Poissonovega jedra in Poissonovega integrala. Ogledali si bomo nekaj lastnosti obeh definiranih pojmov in z njuno pomočjo rešili Dirichletov problem za enotski disk.
Tretje poglavje je namenjeno karakterizaciji harmoničnih funkcij s pomočjo lastnosti povprečne vrednosti in analizi pomembnosti te karakterizacije. 
V zadnjem poglavju bomo s pomočjo orodij, spoznanih v prejšnih poglavjih, navedli in dokazali glavni izrek diplomskega dela - Schwarzov princip zrcaljenja za harmonične funkcije.
%------------------------------------------------------------
\section{Harmonične funkcije}
\subsection{Osnovni pojmi}
    \begin{definicija}
        \label{harm}
        Naj bo $U$ odprta podmnožica v $\mathbb{R}^n$. Naj bo $u$ funkcija, definirana na $U$ in naj bo na definicijskem območju dvakrat zvezno odvedljiva.  
        Pravimo, da je funkcija $u(x_1, x_2, \dots, x_n)$ \emph{harmonična}, če velja
        $$
        \frac{\partial^2 u}{\partial x_1 ^ 2} +  \frac{\partial^2 u}{\partial x_2 ^ 2} + \dots + \frac{\partial^2 u}{\partial x_n ^ 2} = 0.
        $$
        Operatorju $\Delta  = \frac{\partial^2}{\partial x_1 ^ 2} +  \frac{\partial^2}{\partial x_2 ^ 2} + \dots + \frac{\partial^2}{\partial x_n ^ 2}$ pravimo \emph{Laplaceov operator} in pišemo
        $$
        \Delta u = 0.
        $$
    \end{definicija}

    Pogoj za harmoničnost podaja Laplaceovo parcialno diferencialno enačbo, zapisano bodisi s Laplaceovim operatorjem ali razpisano s parcialnimi odvodi drugega reda. 
    Funkcija je torej harmonična, če zadošča zgoraj zapisani parcialni diferencialni enačbi. 
    %Po tihem tu seveda privzemamo obstoj (vsaj) drugih parcialnih odvodov, saj drugače o harmoničnosti funkcije ne moremo govoriti.

    \begin{opomba}
        Vredno je omeniti, da v zgornji definiciji nismo specificirali, ali gre pri funkciji $u$ za realno ali kompleksno funkcijo. 
        Pojem harmoničnosti smo definirali v splošnem, torej tako za kompleksne kot tudi realne funkcije.
        Znotraj diplomske naloge se bomo omejili na funkcije dveh realnih spremenljivk ali funkcijo ene kompleksne spremenljivke. To bomo nato velikokrat delili na realni in imaginarni del ($z = x + iy $) in na ta način prešli nazaj na funkcije dveh realnih spremenljivk.

        Pogoj za harmoničnost takrat zapišemo kot: 
            $$
                \Delta u = \frac{\partial^2 u}{\partial x ^ 2} +  \frac{\partial^2 u}{\partial y ^ 2}= 0.
            $$
    \end{opomba}

    \begin{definicija}
        \emph{Območje} $D$ je povezana odprta podmnožica v $\mathbb{R}^n$.
        Če obstaja $a \in D$, da za vse $b \in D$ in vse $t \in [0,~1]$ tudi $t a + (1-t)b \in D$ pravimo, da je $D$ \emph{zvezdasto območje}.
    \end{definicija}

\subsection{Lastnosti harmoničnih funkcij}
    \begin{trditev}
        \label{hh}
        Naj bo $U \subseteq \mathbb{C}$ odprta množica. Naj bo $f = u + iv$ holomorfna funkcija, $u$ in $v$ pa realni funkciji, definirani na $U$. Potem sta funkciji $u$ in $v$ na $U$ harmonični.
    \end{trditev}

    \begin{dokaz}
        Ker je $f$ holomorfna, zadošča Cauchy-Riemannovemu sistemu enačb. Sledi: 
        \begin{align*}
            \frac{\partial^2 u}{\partial x^2} &= \frac{\partial^2 v}{\partial x \partial y}~~~~\text{in}~~~~\frac{\partial^2 u}{\partial x \partial y} = \frac{\partial^2 v}{\partial y^2}~,~~\text{ter}\\ 
            \frac{\partial^2 u}{\partial y^2} &=  - \frac{\partial^2 v}{\partial x \partial y}~~~~\text{in}~~~~\frac{\partial^2 u}{\partial x \partial y} =  - \frac{\partial^2 v}{\partial x^2}.
        \end{align*}
        Zato velja $\frac{\partial^2 u}{\partial x^2} + \frac{\partial^2 u}{\partial y^2} = \frac{\partial^2 v}{\partial x \partial y} - \frac{\partial^2 v}{\partial x \partial y} =0$ in $\frac{\partial^2 v}{\partial x^2} + \frac{\partial^2 v}{\partial y^2} = \frac{\partial^2 u}{\partial x \partial y} - \frac{\partial^2 u}{\partial x \partial y}=0$.
    \end{dokaz}

    \begin{opomba}
        Če označimo $u = \text{Re}{f}$ in $v = \text{Im}{f}$, nam zgornja trditev v resnici pove, da sta realni in imaginarni del holomorfne funkcije harmonični funkciji. 
    \end{opomba}

    \begin{definicija}
        Naj bo $u$ realna harmonična funkcija, definirana na območju $D$. Če obstaja realna harmonična funkcija $v$, definirana na $D$, da je funkcija $f = u + iv$ na $D$ holomorfna, potem funkciji $v$ pravimo \emph{harmonična konjugiranka funkcije $u$ na $D$}.    
    \end{definicija}

    \begin{trditev}
        \label{konj}
        Naj bo $u$ realna harmonična funkcija, definirana na zvezdastem območju $D$. Potem za $u$ na $D$ obstaja harmonična konjugiranka $v$ in je do konstante natančno enolično določena. 
    \end{trditev}
    \begin{dokaz}
        Konstrukcijo harmonične konjugiranke si bralec, za primer ko je zvezdasto območje kar odprt disk, lahko ogleda v \cite{osnova} na strani $56$ in $57$. 
        Ideja dokaza za splošno zvezdasto območje je podobna. 
    \end{dokaz}

    \begin{opomba}
        V duhu trditev \ref{hh} in \ref{konj}, je velikokrat smiselno na realno harmonično funkcijo $u$ gledati kot na realni del holomorfne funkcije $f = u + iv$, kjer je $v$ njena harmonična konjugiranka. To že nakazuje, da se bodo nekatere lepe lastnosti holomorfnih funkcij prenesle tudi na harmonične funkcije.
    \end{opomba}
    %%V duhu zgornjih dveh opomb, opazimo, da lahko kompleksno funkcijo $f$, zapišemo v obliki realnega in imaginarnega dela oziroma kot $f = u + i v$, s pomočjo nekih realnih funkcij (dveh spremenljivk) $u$ in $v$, ter lahko zaradi linearnosti parcialnih odvodov sklepamo, da je za harmoničnost $f$ kot kompleksne funkcije, dovolj zahtevati harmoničnost $u$ in $v$ kot realnih funkcij.
    %%Podoben argument nam da vedeti, da nam harmoničnost $f = u + iv$, implicira tudi harmoničnost $u$ in $v$. 
    \begin{opomba}
        \label{lin}
        Zaradi linearnosti parcialnih odvodov, je tudi linearna kombinacija harmoničnih funkcij harmonična. To bomo pri reševanju enega izmed glavnih problemov diplomskega dela s pridom uporabili.
    \end{opomba}
    \begin{opomba}
        V literaturi se v definiciji harmoničnosti, za funkcijo $u$, pojavlja tudi zahteva, da je funkcija gladka oziroma $u \in C^{\infty}$. V zgornji definiciji zahtevamo le obstoj drugih parcialnih odvodov, oziroma $u \in C^2$. 
        Da sta definiciji za realne funkcije dveh spremenljivk med seboj ekvivalentni, potrdi spodnja trditev. 
    \end{opomba}
    \begin{trditev}
        \label{gladkosth}
        Naj bo $u$ realna harmonična funkcija, definirana na območju $D$. Potem je $u$ na $D$ gladka, oziroma $u \in C^{\infty}(D)$. 
    \end{trditev}
    \begin{dokaz}
        Ker je  $D$ odprta množica, lahko za vsako točko $z \in D$ najdemo dovolj majhno zvezdasto okolico $U_z$, da bo okolica v celoti vsebovana v $D$. Dovolj je vzeti kar dovolj majhen disk. 
        Po trditvi \ref{konj} na $U_z$ obstaja harmonična konjugiranka $v$, da je $f = u+ iv$ na $U_z$ holomorfna, ter zato na $U_z$ tudi gladka. Sledi, da je na $U_z$ gladka tudi funkcija $u$.
        Ker smo za vsako točko iz $D$ našli odprto okolico, na kateri je funkcija $u$ gladka, je $u$ gladka na celotnem definicijskem območju.
    \end{dokaz}

    \begin{trditev}
        \label{komp_s_hol}
        Naj bo $u$ harmonična funkcija, definirana na območju $D$. Naj bo $\Omega$ območje in $f : \Omega \to D$ holomorfna funkcija. Potem je $u \circ f$ na $\Omega$ harmonična funkcija.
    \end{trditev}
    \begin{dokaz}
        Naj bo $z_0$ poljubna točka iz $\Omega$. Dovolj je pokazati, da ima $z_0$ odprto okolico, na kateri je $u \circ f$ harmonična. Ker je $D$ območje, obstaja $r > 0$, da je \mbox{$\mathbb{D}(f(z_0), r) \subseteq D$}. 
        Po trditvi \ref{konj} zato za $u$ na $\mathbb{D}(f(z_0),r)$ obstaja $v$, da je $g = u + iv$ na $\mathbb{D}(f(z_0), r)$ holomorfna. Po trditvi \ref{hh} je $g \circ f$ holomorfna na $\left\{z~|~ f(z) \in \mathbb{D}(f(z_0), r)\right\}$.
        Ker je $f$ holomorfna, je $A_{z_0} = \left\{z~|~ f(z) \in \mathbb{D}(f(z_0), r)\right\}$ odprta okolica za $z_0$. Sledi, da je na odprti okolici $A_{z_0} \subseteq \Omega$, točke $z_0$, $\text{Re}[g \circ f] = \text{Re}[(u + iv)\circ f] = \text{Re}[(u \circ f) + i(v \circ f)] = u \circ f$ harmonična funkcija. 
        Zato je $u \circ f$ harmonična na $\Omega$.
    \end{dokaz}

\section{Lastnost povprečne vrednosti}
\subsection{Definicija in osnovne lastnosti}
    \begin{definicija}  
        Naj bo $h$ zvezna funkcija, definirana na območju $D$. Denimo, da za $z_0 \in D$ in $r > 0$ velja $\overline{\mathbb{D}}(z_0, r) \subseteq D$. \emph{Povprečje funkcije} $h$ na $\overline{\mathbb{D}}(z_0, r)$ definiramo kot:
        $$
            A(r) = \int_{0}^{2 \pi}{h \big(z_0 + r e^{i\theta}\big)\frac{d\theta}{2 \pi}}.
        $$
    \end{definicija}
    \begin{trditev}
        \label{zvpov}
        Naj bo $h$ zvezna funkcija, definirana na območju $D$, ter denimo, da za $z_0 \in D$ in $r > 0$ velja $\overline{\mathbb{D}}(z_0, r) \subseteq D$. 
        Potem je $A(r)$ zvezna na $(0,~r]$ in velja: $\lim_{r \downarrow 0}{A(r)} = h(z_0)$.
    \end{trditev}
    \begin{dokaz}
        Ker je povprečje zvezne funkcije definirano kot integral zvezne funkcije, je funkcija $A(r)$ zvezna na $(0,~r]$. Drugi del trditve dokažimo po definiciji.
        Naj bo $\epsilon > 0$ poljubno majhen. Velja:
        $$
            |A(r) - h(z_0)| = \bigg|\int_{0}^{2\pi} \big[h(z_0 + r e^{i\theta})  - h(z_0)\big] \frac{d\theta}{2\pi} \bigg| \leq \int_{0}^{2 \pi} \big| h(z_0 + r e^{i\theta}) - h(z_0) \big| \frac{d\theta}{2 \pi}.
        $$
        Ker je $h$ zvezna, obstaja $\delta > 0$, da za vsak $r < \delta$ in vsak $\theta \in [0,~2\pi]$ velja \mbox{$|h(z_0 + r e^{i\theta}) - h(z_0)| < \epsilon$}.
        Za vsak $r$, ki je manjši od $\delta$, torej $|A(r) - h(z_0)| < \epsilon$. Ker je bil $\epsilon$ poljubno majhen, res velja $\lim_{r \downarrow 0}{A(r)} = h(z_0)$.
    \end{dokaz}

    \begin{definicija}
        Naj bo $h$ zvezna funkcija, definirana na območju $D \subseteq \mathbb{C}$. Pravimo, da ima funkcija $h$ na $D$ \emph{lastnost povprečne vrednosti}, če za vsak $z_0 \in D$ obstaja $\epsilon_0 > 0$, da je $\overline{\mathbb{D}}(z_0, \epsilon_0) \subseteq D$ in za vsak $0 < \epsilon \leq \epsilon_0 $ velja:
        $$
            h(z_0) = \frac{1}{2 \pi} \int_{0}^{2 \pi}{h(z_0 + \epsilon e^{i \theta}) d\theta}.
        $$
    \end{definicija}
    \begin{opomba}
        Definicija nam pove, da ima zvezna funkcija $h$ na območju $D \subseteq \mathbb{C}$ lastnost povprečne vrednosti, če za vsak $z_0$ iz $D$ velja, 
        da je $h(z_0)$ povprečje vrednosti $h(z)$, kjer $z$ teče po majhni krožnici s središčem v $z_0$.
    \end{opomba}

    Smiselno se je vprašati, zakaj je pogoj, da ima funkcija lastnost povprečne vrednosti, definiran prek krivuljnega integrala po robu diska in ne ploskovnega integrala po območju, ki ga omejuje rob diska.
    Naslednja trditev nam pove, da funkcija, ki ima lastnost povprečne vrednosti, zadošča tudi pogoju, da je vrednost v središču enaka povprečju vrednosti na celotnem disku.

    \begin{trditev}
        Naj bo $u$ zvezna funkcija, definirana na območju $D$. Naj ima $u$ na $D$ lastnost povprečne vrednosti, ter denimo da za $z_0 \in D$ in $r>0$ velja $\overline{\mathbb{D}}(z_0,r) \subseteq D$. Potem velja:
        $$
            u(z_0) = \frac{1}{\pi r^2} \iint_{\mathbb{D}(z_0,r)}{u(x + iy)~dxdy}.
        $$
    \end{trditev}
    \begin{dokaz}
        Dvojni integral po disku, lahko z uvedbo polarnih koordinat $x = z_0 + \rho \cos(\theta)$, $y = z_0 + \rho \sin(\theta)$ prevedemo na dvakratnega. Velja:
        $$
        \frac{1}{\pi r^2} \iint_{\mathbb{D}(z_0,r)}{u(x + iy)~dxdy} = \frac{1}{\pi r^2} \int_{0}^{r}{\rho \left[\int_{0}^{2 \pi} u(z_0 + \rho e^{i\theta})~d\theta \right]d\rho}. 
        $$
        Ker ima $u$ na $D$ lastnost povprečne vrednosti, velja:
        $$
            2 \pi~u(z_0) = \int_{0}^{2 \pi}{u(z_0 + \rho e^{i \theta}) d\theta},~\text{za vsak}~\rho \in [0, r].
        $$
        Sledi:
        $$
        \frac{1}{\pi r^2} \iint_{\mathbb{D}(z_0,r)}{u(x + iy)~dxdy} = \frac{2}{r^2} \int_{0}^{r}{\rho \left[u(z_0)\right]d\rho} = u(z_0) \left[\frac{2}{r^2} \int_{0}^{r}{\rho~d\rho}\right] = u(z_0).
        $$
    \end{dokaz}

    \begin{opomba}
        \label{linlpv}
        Iz linearnosti integrala sledi, da je tudi linearna kombinacija funkcij z lastnostjo povprečne vrednosti funkcija z lastnostjo povprečne vrednosti. 
    \end{opomba}

    %\begin{lema}
    %    \label{lema_princip_max}
    %    Naj bo $u$ zvezna realna funkcija, definirana na območju $D$. Naj ima $u$ na $D$ lastnost povprečne vrednosti in naj obstaja $M \in \mathbb{R}$, da za vsak $z \in D$ velja: $u(z) \leq M$. 
    %    Če obstaja $z_0 \in D$, da velja: $u(z_0) = M$, potem je $u(z) = M$ za vsak $z \in D$.
    %\end{lema}
    %\begin{dokaz}
    %    Ker je $u$ na $D$ zvezna, je množica $A = \{z \in D~|~u(z) < M\}$ odprta. Po definiciji dokažimo, da je odprta tudi množica $B = \{z \in D~|~u(z) = M\}$. Naj bo $z_1 \in B$. Ker je $D$ odprta, obstaja $\rho_{z_1}$, da je za vsak $0 < r < \rho_{z_1}$ tudi $\overline{\mathbb{D}}(z_1, r) \subseteq D$. 
    %    Funkcija $u$ ima na $D$ lastnost povprečne vrednosti, zato velja:
    %    $$
    %        u(z_1) = \int_{0}^{2\pi}{u(z_1 + re^{i\theta}) \frac{d\theta}{2\pi}},~\text{za vsak}~0<r<\rho_{z_1}, 
    %    $$
    %    oziroma: 
    %    $$
    %    0 = \int_{0}^{2\pi}{\left[u(z_1) - u(z_1 + re^{i\theta}) \right]\frac{d\theta}{2\pi}},~\text{za vsak}~0<r<\rho_{z_1}.
    %    $$
    %    Ker je integrand nenegativen in zvezen, integral pa enak nič, je integrand enak nič. 
    %    Sledi, da je $M = u(z_1) = u(z_1 + r e^{i \theta})$, za vsak $\theta \in [0,~2\pi]$ in vsak $0 < r < \rho_{z_1}$, oziroma ima poljuben $z_1 \in B$ okolico, ki je vsebovana v $B$. Torej je $B$ odprta.
    %    Predpostavka nam pove, da za vsak $z \in D$ velja $u(z) \leq M$, zato je $D$ disjunktna unija množic $A$ in $B$. Ker je $D$ povezana, $A$ in $B$ pa odprti množici, je ena izmed niju prazna, druga pa posledično enaka $D$. Ker po predpostavki $B$ vsebuje $z_0$, je $A$ prazna, $B$ pa je enaka $D$. Sledi, da je $u(z) = M$ za vsak $z \in D$. 
    %\end{dokaz}
\subsection{Princip maksima}
    \begin{trditev}[Princip maksima za funkcije z lastnostjo povprečne vrednosti]
        \label{pm_lpv}
        Naj bo $h$ zvezna kompleksna funkcija, definirana na območju $D$. Naj ima $h$ na $D$ lastnost povprečne vrednosti in naj obstaja $M \in \mathbb{R}$, da velja $|h(z)| \leq M$ za vsak $z \in D$. 
        Če obstaja $z_0 \in D$, da je $|h(z_0)| = M$, potem je funkcija $h$ na $D$ konstantna. 
    \end{trditev}
    \begin{dokaz}
        Ker je $|h(z_0)| = M$, obstaja $\varphi \in [0,~2\pi]$, da velja: \mbox{$h(z_0) = M e^{i \varphi}$}.
        Definirajmo: $H(z) = h(z) e^{-i\varphi},~z \in D$. Po opombi \ref{linlpv} ima $H$ na $D$ lastnost povprečne vrednosti. Opazimo, da za vsak $z \in D$ velja $|H(z)|  \leq M$, obenem pa \mbox{$H(z_0) = M$}.
        Ker je funkcija $H$ na $D$ zvezna, $\{H(z_0)\}$ pa zaprta množica, je njena praslika \mbox{$A = \{z \in D~|~H(z) = H(z_0) = M\}$} prav tako zaprta množica. Po definiciji dokažimo, da je $A$ tudi odprta množica. 
        Naj bo $z_1 \in A$. Ker je $D$ odprta, obstaja $\rho_{z_1}$, da je za vsak $0 < r < \rho_{z_1}$ tudi $\overline{\mathbb{D}}(z_1, r) \subseteq D$. 
        Funkcija $H$ ima na $D$ lastnost povprečne vrednosti, zato velja:
        $$
            H(z_1) = \int_{0}^{2\pi}{H(z_1 + re^{i\theta}) \frac{d\theta}{2\pi}},~\text{za vsak}~0<r<\rho_{z_1}.
        $$
        Ker je $z_1 \in A$, velja $H(z_1) = M$, ter zato $\text{Re}[H(z_1)] = H(z_1)$. 
        Sledi:
        $$
        H(z_1) = \text{Re}[H(z_1)] = \int_{0}^{2\pi}{\text{Re}[H(z_1 + re^{i\theta})]~\frac{d\theta}{2\pi}},~\text{za vsak}~0<r<\rho_{z_1},
        $$
        oziroma:
        $$
        0 = \int_{0}^{2\pi}{\left(H(z_1) - \text{Re}[H(z_1 + re^{i\theta})] \right)\frac{d\theta}{2\pi}},~\text{za vsak}~0<r<\rho_{z_1}.
        $$
        Vemo, da za vsak $z \in D$ velja $|H(z)| \leq M = H(z_1)$, zato za vsak $z \in D$ velja $\text{Re}[H(z)] \leq |H(z)| \leq H(z_1)$.
        Sedaj vemo, da je integrand integrala nenegativen in zvezen, integral pa enak nič, zato je integrand enak nič. 
        Sledi, da je \mbox{$H(z_1) = \text{Re}[H(z_1 + r e^{i \theta})]$}, za vsak $\theta \in [0,~2\pi]$ in vsak $0 < r < \rho_{z_1}$. Definicija absolutne vrednosti nam da $|H(z)| = \sqrt{\text{Re}[H(z)]^2 + \text{Im}[H(z)]^2}$. 
        Ker vemo, da za vsak $z \in D$ velja $|H(z)| \leq H(z_1) = M$, za vsak $w \in \mathbb{D}(z_1,\rho_{z_1})$ pa $H(z_1) = \text{Re}[H(w)]$, sledi, da je za vsak $w \in \mathbb{D}(z_1,\rho_{z_1})$ $\text{Im}[H(w)] = 0$, oziroma $H(w) = H(z_1)$. 
        Poljuben $z_1 \in A$ ima torej odprto okolico, $ \mathbb{D}(z_1, \rho_{z_1}) \subseteq A$. Torej je $A$ odprta.
        Predpostavka trditve nam pove, da $A$ ni prazna, saj $z_0 \in A$. Ker je $D$ povezana, $A$ pa odprta in hkrati zaprta neprazna množica, velja $A = D$. 
        Po definiciji množice $A$ je zato funkcija $H$ na $D$ konstantna. \mbox{Posledično je na $D$ konstantna tudi funkcija $h$.} 
    \end{dokaz}

    \begin{posledica}
        \label{posledica_pm_lpv}
        Naj bo $h$  kompleksna funkcija, definirana na omejenem območju $D$. Naj bo $h$ zvezna na $\overline{D}$ in naj ima na $D$ lastnost povprečne vrednosti. 
        Če obstaja $M \in \mathbb{R}$, da velja $|h(z)| \leq M$ za vsak $z \in \partial D$, potem velja $|h(z)| \leq M$ za vsak $z \in \overline{D}$. 
    \end{posledica}
    \begin{dokaz}
        Ker je funkcija $h$ zvezna na kompaktnem območju $\overline{D}$, na $\overline{D}$ zavzame minimum in maksimum. Posledično funkcija $|h|$ na $\overline{D}$ zavzame maksimum. 
        Torej $|h|$ zavzame maksimum na $D$ ali na $\partial D$. Če funkcija $|h|$ maksimum zavzame na $D$, potem je po trditvi \ref{pm_lpv} funkcija $h$ na $D$ konstantna, oziroma zaradi zveznosti konstantna celo na $\overline{D}$. 
        Sledi, da $h$ maksimum gotove zavzame na $\partial D$ iz česar lahko sklepamo želeno.
    \end{dokaz}

    \begin{trditev}
        Naj bo $f$ holomorfna funkcija, definirana na območju $D$. Potem ima funkcija $f$ na $D$ lastnost povprečne vrednosti.
    \end{trditev}
    \begin{dokaz}
        Ker je funkcija $f$ na območju $D$ holomorfna, lahko uporabimo Cauchyjevo integralsko formulo za vsak zaprt disk, katerega zaprtje je vsebovano v $D$. Za vsak $z \in D$ in vsak $r > 0$, kjer je $\overline{\mathbb{D}}(z,r) \subseteq \mathbb{D}$ torej velja:
        $$
        f(z) = \frac{1}{2 \pi i} \int_{\partial \mathbb{D}(z, r)}{\frac{f(\xi)}{\xi  - z}}d\xi.
        $$
        Ko rob diska parametriziramo z $\xi = z + r e^{i \varphi}$, dobimo:
        $$
        f(z) = \frac{1}{2 \pi} \int_{0}^{2\pi}{f(z + re^{i\varphi})}d\varphi.
        $$
    \end{dokaz}

    \begin{trditev}
        \label{harmonicnapovp}
        Naj bo $u$ harmonična funkcija, definirana na območju $D \subseteq \mathbb{C}$. Naj bo $z_0 \in D$ in $\rho > 0$, da velja $\overline{\mathbb{D}}(z_0, \rho) \subseteq D$. Za vsak $0 < r < \rho$ potem velja:
            $$
                u(z_0) = \frac{1}{2 \pi} \int_{0}^{2 \pi}{u(z_0 + r e^{i \theta}) d\theta}.
            $$
    \end{trditev}
    \begin{dokaz}
        Na $\overline{\mathbb{D}}(z_0, \rho) \subseteq D$ označimo $P = -\frac{\partial u}{\partial y}$ in $Q = \frac{\partial u}{\partial x}$, ter uporabimo Greenovo integralsko formulo:
        $$
            \int_{\partial \mathbb{D}(z_0, \rho)}{P dx + Q dy} = \iint_{\overline{\mathbb{D}}(z_0, \rho)}{\bigg(\frac{\partial Q}{\partial x} - \frac{\partial P}{\partial y}\bigg)dx dy}.
        $$ 
        Sedaj upoštevajmo harmoničnost funkcije $u$ in vstavimo v Greenovo formulo:
        $$
        \int_{\partial \mathbb{D}(z_0, \rho)}{-\frac{\partial u}{\partial y} dx + \frac{\partial u}{\partial x} dy} = \iint_{\overline{\mathbb{D}}(z_0, \rho)}{\bigg(\frac{\partial^2 u}{\partial x^2} + \frac{\partial^2 u}{\partial y^2}\bigg)dx dy} = 0. 
        $$
        Rob diska lahko pri $z_0 = x_0 + iy_0$ parametriziramo z $x(\theta) = x_0 + \rho \cos(\theta),~y(\theta) = y_0 + \rho \sin(\theta)$. Ko to vstavimo v zgornjo enakost, dobimo:
        $$
        0 = \rho \int_{0}^{2 \pi}{\bigg[\frac{\partial u}{\partial x} \cos(\theta) + \frac{\partial u}{\partial y} \sin(\theta)\bigg] d\theta} = \rho \int_{0}^{2\pi}{\frac{\partial u}{\partial \rho}\big({z_0 + \rho e^{i\theta}\big)d\theta}}.
        $$
        Ker je $u$ harmonična, je po trditvi \ref{gladkosth} gladka, zato lahko zamenjamo limitna procesa integriranja in odvajanja. Delimo z $2\pi \rho$ in dobimo:
        $$
        0 = \frac{\partial}{\partial \rho} \int_{0}^{2\pi}{u\big({z_0 + \rho e^{i\theta}\big)\frac{d\theta}{2 \pi}}}.
        $$
        Sledi, da je vrednost zgornjega integrala konstantna, za vsak $0 <r < \rho$. Obstaja torej $c \in \mathbb{C}$, da je: 
        $$
        \int_{0}^{2\pi}{u\big({z_0 + r e^{i\theta}\big)\frac{d\theta}{2 \pi}}} = c,~\text{za vsak}~ 0 < r < \rho.
        $$
        Potrebno je le še pokazati, da $c = u(z_0)$.
        Ker je $u$ zvezna, po trditvi \ref{zvpov} velja, da v limiti $r \downarrow 0$ dobimo:
        $$
        c = \lim_{r \downarrow 0}{\bigg[\int_{0}^{2\pi}{u\big({z_0 + r e^{i\theta}\big)\frac{d\theta}{2 \pi}}}\bigg]} = \int_{0}^{2\pi}{{u(z_0)~\frac{d\theta}{2 \pi}}} = u(z_0).
        $$
    \end{dokaz}

    \begin{opomba}
        Dokazali smo že, da imajo holomorfne funkcije lastnost povprečne vrednosti, zgornja trditev pa nam pove, da imajo lastnost povprečne vrednosti tudi harmonične funkcije. 
    \end{opomba}

    \begin{posledica}[Princip maksima za harmonične funkcije]
        \label{pm_harm}
        Naj bo $h$ kompleksna funkcija, definirana na območju $D$. Naj bo funkcija $h$ na $D$ harmonična in naj obstaja $M \in \mathbb{R}$, da velja $|h(z)| \leq M$ za vsak $z \in D$. 
        Če obstaja $z_0 \in D$, da je $|h(z_0)| = M$, potem je funkcija $h$ na $D$ konstantna.  
    \end{posledica}
    \begin{dokaz}
        Ker je funkcija $h$ na $D$ harmonična, ima po trditvi \ref{harmonicnapovp} na $D$ lastnost povprečne vrednosti. Sedaj funkcija $h$ zadošča pogojem za trditev \ref{pm_lpv}, zato je na $D$ konstantna.
    \end{dokaz}

    \begin{posledica}
        Naj bo $h$ kompleksna funkcija, definirana na omejenem območju $D$. Naj bo $h$ zvezna na $\overline{D}$ in naj bo na $D$ harmonična. 
        Če obstaja $M \in \mathbb{R}$, da velja $|h(z)| \leq M$ za vsak $z \in \partial D$, potem velja $|h(z)| \leq M$ za vsak $z \in \overline{D}$. 
    \end{posledica}
    \begin{dokaz}
        Ker je funkcija $h$ na $D$ harmonična, ima po trditvi \ref{harmonicnapovp} na $D$ lastnost povprečne vrednosti. Sedaj funkcija $h$ zadošča pogojem za trditev \ref{posledica_pm_lpv}, ki dokazuje želeno.
    \end{dokaz}

    \begin{opomba}
        \label{motivacija}
        Vredno je že na tej točki omeniti, da za harmonične funkcije velja celo več kot nam to narekuje trditev \ref{harmonicnapovp}. Harmonične funkcije se da namreč karakterizirati s pomočjo lastnosti povprečne vrednosti. Motivacija za ogled naslednjega razdelka temelji ravno na dokazu omenjenega rezultata.  
    \end{opomba}

\section{Dirichletov problem za enotski disk}
\subsection{Formulacija problema}
    Za začetek navedimo problem, ki ga bomo znotraj poglavja reševali.
    Naj bo $\mathbb{D}$ enotski disk. Zvezno kompleksno funkcijo $h$, definirano na $\partial \mathbb{D}$, bi želeli razširiti do funkcije $H$, da bo $H$ harmonična na $\mathbb{D}$ in zvezna na $\overline{\mathbb{D}}$.
    Opisan problem je grafično prikazan na spodnji sliki. 
    \begin{figure}[H]
        \begin{center}
            \includegraphics[width = 0.7 \textwidth]{dirichlet_form.png}
            \caption{Dirichletov problem za enotski disk, z začetnim pogojem $h$.}
        \end{center}    
    \end{figure}

    Kot smo že komentirali po definiciji \ref{harm}, so funkcije harmonične natanko tedaj ko zadoščajo Laplaceovi parcialni diferencialni enačbi.   
    Vredno je omeniti, da lahko na Direchletov problem za enotski disk gledamo tudi iz stališča teorije diferencialnih enačb. Gre za problem iskanja funkcije, ki na notranjosti območja reši Laplaceovo diferencialno enačbo ob robnem pogoju, ki ga določa v naprej podana funkcija na robu območja.     
    V našem primeru, je notranjost območja kar notranjost enotskega diska, roben pogoj zvezne funkcije pa podaja začetna zvezna funkcija, podana na enotski krožnici.
    Navadno se v teoriji parcialnih diferencialnih enačb srečamo s tako imenovanimi dobro postavljenimi matematičnimi problemi, o katerih si bralec več lahko prebere na REFERENCA.     
    V nadaljevanju se bomo le seznanili z njihovo definicijo in pokazali, da zgoraj zastavljen Dirichletov problem na enotskem disku ustreza definiciji. 

    \begin{definicija}[J. Hadamard, 1902]
        \label{def_dp}
        Pravimo, da je matematičen problem parcialnih diferencialnih enačb z robnimi in začetnimi pogoji \emph{dobro postavljen}, če zanj velja:
        \begin{enumerate}[label={\Alph*)}]
            \item rešitev problema obstaja,
            \item rešitev problema je ena sama, oziroma rešitev je enolično določena,
            \item rešitev je zvezno odvisna od začetnih podatkov problema.
        \end{enumerate}
    \end{definicija}

    \begin{lema}
        \label{enolicno}
        Če rešitev za Dirichletov problem na enotskem disku obstaja, je enolično določena.
    \end{lema}
    \begin{dokaz}
        Denimo, da obstajata dve različni rešitvi, $H_1$ in $H_2$, Dirichletovega problema za enotski disk z začetnim pogojem $h$.
        Oglejmo si razliko $H_1 - H_2$. Vemo, da je njuna razlika na $\partial \mathbb{D}$ enaka $0$, saj sta tam enaki $h$. Ker sta funkciji $H_1$ in $H_2$ na $\mathbb{D}$ harmonični, po opombi \ref{lin} vemo, da je tudi njuna razlika harmonična funkcija. 
        Po posledici \ref{pm_harm}, oziroma principu maksima za harmonične funkcije sledi, da je $H_1 - H_2 \equiv 0$ tudi na $\mathbb{D}$. Sledi enakost $h_1$ in $h_2$ tudi na $\mathbb{D}$ in protislovje. 
    \end{dokaz}
    
    Lema \ref{enolicno} nam pove, da je rešitev problema največ ena, zato se je dovolj posvetiti konstrukciji potencialne rešitve. 
    Opazimo, da lahko točke $z \in \partial \mathbb{D}$ zapišemo z $e^{i \theta} \in \partial \mathbb{D}$, kjer $\theta \in [0,2\pi]$. Funkcijo $h(z),~z \in \partial \mathbb{D}$ lahko tako pišemo kot kompozitum $h(e^{i \theta}),~\theta \in [0,2\pi]$.
    
    Sedaj poskusimo skonstruirati harmonično razširitev, ki bo zadoščala Dirichletovemu problemu na enotskem disku za zvezno funkcijo $h(e^{i \theta})$.
    Za začetek se posvetimo enostavnim primerom in si šele za tem teorijo oziroma konstrukcijo oglejmo v splošnem. 
    Naravno je za enostavne zvezne funkcije, definirane na $\partial \mathbb{D}$, vzeti kar polinome. V luči zapisa $h(e^{i\theta})$ je polinomska spremenljivka lahko kar $e^{i\theta}$, kjer bo $\theta$ iz intervala $[0,2\pi]$. 
    Zaradi linearnosti Laplaceovega opeartorja, se problem preveda na iskanje rešitve za monome. 
    Zato si oglejmo funkcije oblike $h(e^{i \theta}) = e^{i k \theta}, k \in \mathbb{Z}$, ter za njih poskusimo skonstruirati želeno razširitev. 
    Hitro opazimo, da se nam v tem primeru pojavlja preprosta eksplicitna razširitev s predpisom $H(r e^{i \theta}) = r^{|k|}e^{i k \theta},~\text{za}~r\in [0, 1]~\text{in}~\theta \in [0, 2\pi]$. 
    Tako predpisana razširitev je za $k \geq 0$ na $\mathbb{D}$ harmonična, saj je $r^k e^{ik\theta} = z^k$ celo holomorfna funkcija in zato harmonična. Pri $k < 0$ dobimo razširitev $r^{-k} e^{ik\theta} = \overline{z}^{-k}$, ki v splošnem ni holomorfna, a je \mbox{kljub} temu harmonična. 
    Gre namreč za monom v konjugirani spremenljivki, harmoničnost katerega lahko preverimo po definiciji prek parcialnih odvodov.
    Obenem je za vsak $k \in \mathbb{Z}$ razširitev zvezna na $\overline{\mathbb{D}}$ in se na $\partial \mathbb{D}$ ujema z začetnimi pogoji. Razširitev torej zadošča pogojem za rešitev Dirichletovega problema na enotskem disku, ki je po lemi \ref{enolicno} enolično določena. 
    Z upoštevanjem linearnosti Laplaceovega operatorja lahko za začetne funkcije oblike $h(e^{i\theta}) = \sum_{k = -N}^{N}{a_k e^{ik\theta}};~\theta \in [0,2\pi]$, konstruiramo harmonično razširitev s predpisom
    $H(r e^{i \theta}) = \sum_{k = -N}^{N}{a_k r^{|k|}e^{ik\theta}};~r \in [0,1],~\theta \in [0,2\pi]$. Smiselno bi bilo koeficiente $a_k$ izraziti direktno prek funkcije $h$. 
    V ta namen si oglejmo $\int_{-\pi}^{\pi}{e^{ij\theta} e^{-ik\theta}d\theta}$, kjer $k,l \in \mathbb{Z}$. Prek integracije zapisane kompleksne funkcije preverimo tako imenovano ortogonalno relacijo med kompleksnimi eksponenti:
        $$
        \int_{-\pi}^{\pi}{e^{ij\theta} e^{-ik\theta}~\frac{d\theta}{2\pi}} = 
        \begin{cases}
            1~;~&j=k\\
            0~;~&j \neq k\\
        \end{cases}
        .$$

        Zgornja relacija nam pri $h(e^{i\theta}) = \sum_{k = -N}^{N}{a_k r^{|k|} e^{ik\theta}}$ omogoča izražavo koeficientov $a_k$ kot:
        $$
            a_ k = \int_{-\pi}^{\pi}{h \left(e^{i\theta}\right)e^{-ik\theta}~\frac{d\theta}{2\pi}}.
        $$
    Z uporabo zgornje zveze izrazimo:
    $$
        \sum_{k = - N}^{N}{ a_k r^{|k|}e^{ik\theta}} = \sum_{k = - N}^{N} \left(\int_{-\pi}^{\pi}{h(e^{i \varphi}) e^{- i k \varphi}~\frac{d \varphi}{2 \pi}}\right) r^{|k|} e^{i k \theta}.
    $$
    Zamenjamo vsoto in integral, ter množico, po kateri teče indeks vsote razširimo na vsa cela števila. Pri tem se rezultat ne spremeni, saj je sumand pri dodanih indeksih enak nič. Dobimo ekspliciten zapis:
    \begin{equation}
        \label{int1}
        \widetilde{h}(r e^{i \theta}) = \int_{-\pi}^{\pi}{h(e^{i \varphi}) \left[\sum_{k = - \infty}^{\infty} r^{|k|} e^{- i k \varphi} e^{i k \theta} \right] \frac{d \varphi}{2 \pi}}, ~~~ r e^{i\theta} \in \overline{\mathbb{D}}.
    \end{equation}

    Zgornji ekspliciten zapis razširitve je le drugače zapis že komentirane rešitve Dirichletovega problema na enotskem disku, za polinom v spremenljivki $e^{i \theta}$, $\theta \in [0,2\pi]$.
    Sedaj bomo zgornjo funkcijo, ki smo jo na intuitiven način konstruirali s pomočjo predpostavljene polinomske oblike funkcije $h$ vzeli za definicijo novega pojma in z njegovo pomočjo prišli do razširitve za splošne zvezne funkcije $h$.
    
\subsection{Poissonovo jedro}
    \begin{definicija}
        \emph{Poissonovo jedro} je funkcija definirana s predpisom
        $$
           P_r(\theta) = \sum_{k = -\infty}^{\infty}{r^{|k|} e^{i k \theta}}\text{, kjer je}~\theta \in [-\pi, \pi]~\text{in}~ r < 1.
        $$
    \end{definicija}
    \begin{opomba}
        Na Poissonovo jedro lahko gledamo kot funkcijo dveh spremenljivk, $\theta$ in $r$, ali pa kot na družino funkcij, indeksiranih s parametrom $r$.
    \end{opomba}

    Smiselno se je vprašati, ali za vsako vrednost iz definicijskega območje Poissonovega jedra vrsta na desni strani definicijske enakosti sploh konvergira. Potencialne strahove pomiri naslednja trditev.
    
    \begin{trditev}
        Naj bo $\rho < 1$. Vrsta, definirana s Poissonovim jedrom konvergira enakomerno na množici $\{(r,\theta)~|~r \in [0,\rho],~ \theta \in [0,2\pi]\}$.
    \end{trditev}
    \begin{dokaz}
        Na množici $\{(r,\theta)~|~r \in [0,\rho],~ \theta \in [0,2\pi]\}$ velja $|r^{|k|} e^{i k \theta}| \leq \rho^{|k|}$. Ker je $\rho < 1$, številska vrsta $\sum_{k = -\infty}^{\infty}{\rho ^{|k|}}$ konvergira. 
        Po Weierstrassovem M-testu zato Poissonovo jedro konvergira enakomerno na množici $\{(r,\theta)~|~r \in [0,\rho],~ \theta \in [0,2\pi]\}$.
    \end{dokaz}

    Poissonovo jedro lahko zapišemo na več načinov. Če vrsto v definiciji Poissonovega jedra razbijemo na tri dele, glede na predznačenost indeksa vrste, dobimo:
    \begin{align}
        P_r(\theta) &=  1 + \sum_{k = 1}^{\infty}{r^{k} e^{i k \theta}} +  \sum_{j = 1}^{\infty}{r^{j} e^{-i j \theta}}~;  &  &r \in [0,1),~\theta \in [0,2\pi] \label{eq1}\\
        &= 1 + \sum_{k=1}^{\infty}{z^k} + \sum_{j=1}^{\infty}{\overline{z}^{j}}~;   & &z = r e^{i\theta} \in \mathbb{D}. \notag
    \end{align}
    Ker je Poissonovo jedro definirano na notranjosti enotskega diska, kjer je absolutna vrednost kompleksne spremenljivke manjša od ena, obe vrsti v zapisu \eqref{eq1} konvergirata.
    Ko ju seštejemo s formulo za geometrijsko vrsto dobimo:

    \begin{align}
        \label{eq2}
        P_r(\theta) &= 1 + \frac{z}{1 - z}+ \frac{\overline{z}}{1 - \overline{z}} = \frac{1 - |z|^2}{|1-z|^2}~;& &z = r e^{i\theta} \in \mathbb{D}.
    \end{align}
    Opazimo, da bi \eqref{eq2} lahko zapisali tudi nekoliko drugače:
    \begin{align}
        \label{eq3}
        P_r(\theta) &= 1 + \frac{z}{1 - z}+ \frac{\overline{z}}{1 - \overline{z}} = 1 + \left(\frac{z}{1 - z} \right)+ \overline{\left(\frac{z}{1 - z}\right)} \notag \\
        &= 1 + 2~\text{Re}\left[\frac{z}{1-z}\right] =\text{Re}\left[\frac{1 + z}{1-z}\right]; & &z = r e^{i\theta} \in \mathbb{D}.
    \end{align}
    Enakost $|1 - z|^2 = (\overline{1 - z})(1 - z) = (1 - \overline{z})(1 - z) = 1 + r^2 - 2r \cos(\theta)$ pa nam omogoči zapis:
    \begin{align}
        \label{eq4}
        P_r(\theta) & = \frac{1-r^2}{1+ r^2 - 2r \cos(\theta)}~; & &r \in [0,1),~\theta \in [0,2\pi].
    \end{align}
    % 
    %Enostavno je pokazati ekvivalenco z zapisom \ref{eq1}. Vrsto v definiciji le razbijemo na tri dele (glede na predznačenost iterativnega indeksa) in člen vsake izmed vsot zapišemo s kompleksno spremenljivko. 
    %Za dokaz ekvivalence obliko \ref{eq2} se moramo nekoliko bolj potruditi in si pomagati z \ref{eq1}. Vrsti v \ref{eq1} sta definirani na notranjosti enotskega diska, kjer je absolutna vrednost manjša od 1, zato obe vrsti konvergirata in ju lahko seštejemo s pomočjo formule za geometrijsko vrsto. 
    %Prek enakosti $|1 - z|^2 = (\overline{1 - z})(1 - z) = (1 - \overline{z})(1 - z) = 1 + r^2 - 2r \cos(\theta)$ dobimo ekvivalenco z zapisom \ref{eq2}:
    %$$
    %    P_r(\theta) = 1 + \frac{z}{1 - z} + \frac{\overline{z}}{1 - \overline{z}} = \frac{1 - |z|^2}{|1 - z|^2} = \frac{1 - r^2}{1 + r^2 - 2r\cos(\theta)}.
    %$$
    %%%\begin{opomba}
    %%%    \label{par_s_t_Pr}
    %%%    V literaturi se pogosto predpis za Poissonovo jedro za vsak $r \in [0,1)$, namesto z $\theta \in [0, 2\pi]$, zapiše v odvisosti od spremenljivke $t \in [0,1]$.
    %%%    Predpis za Poissonovo jedro tako postane:
    %%%    \begin{equation}
    %%%        \label{poisson_t}
    %%%       P_r(t) = \sum_{k = -\infty}^{\infty}{r^{|k|} e^{i (2 \pi t)k }} = \frac{1-r^2}{1+ r^2 - 2r \cos(2 \pi t)}\text{~,~~kjer je}~ t\in [0, 1]~\text{in}~ r < 1.
    %%%    \end{equation}
    %%%    Prednost takšnega zapisa, bo jasna v nadaljevanju.
    %%%\end{opomba}

    Preden se lotimo uporabe Poissonovega jedra, si s pomočjo zgornjih zapisov oglejmo še nekaj njegovih lastnosti, ki so zbrane v spodnjih trditvah. 
    
    \begin{trditev}
        \label{lastpk}
        Poissonovo jedro ima naslednje lastnosti:
        \begin{enumerate}[label={\alph*)}]
            \item kot funkcija spremenljivke $\theta$ je periodično s periodo $2\pi$, 
            \item za vsak $r \in [0,1)~\text{in vsak}~\theta \in [0,2\pi]~\text{je}~P_r(-\theta) = P_r(\theta)$,
            \item za vsak $r \in [0,1) P_r(\theta)~\text{za}~\theta \in [-\pi, 0 ]~\text{narašča in za}~\theta \in [0, \pi ]~\text{pada}$,
            \item za vsak $r \in [0,1)~\text{in vsak}~\theta\in [0,2\pi]~\text{je}~P_r(\theta) > 0$,
            \item za vsak $r \in [0,1)~\text{velja}~\int_{-\pi}^{\pi}{P_r(\theta) \frac{d\theta}{2\pi}} = 1$.
        \end{enumerate}
    \end{trditev}
    \begin{dokaz}
        Lastnosti a), b) in c) sledijo iz zapisa \eqref{eq4}. Opazimo, da je v tem zapisu od $\theta$ odvisen le člen s funkcijo kosinus. 
        Ta zagotavlja $2\pi$-periodičnost in sodost, ter podaja želen interval naraščanja in padanja Poissonovega jedra, v odvisnosti od spremenljivke $\theta$. 
        
        Za dokaz točke d) si oglejmo zapis \eqref{eq2}. Ker je Poissonovo jedro definirano na notranjosti enotskega diska, je absolutna vrednost v zapisu uporabljene kompleksne spremenljive manjša od ena. 
        Sledi, da sta števec in imenovalec v zapisu strogo pozitivna, kar dokazuje trditev. 

        Pri dokazu točke e), si pomagajmo z izpeljavo rešitve za Dirichletov problem na enotskem disku. 
        Za robni pogoj, bomo na robu enotskega diska vzeli funkcijo, ki je identično enaka $1$. Označimo torej $h(z) \equiv 1,~ z \in \partial \mathbb{D}$. Opazimo, da je \mbox{$H(z) \equiv 1,~z \in \overline{\mathbb{D}}$} trivialna rešitev tako postavljenega Dirichletovega problema za enotski disk.
        Lema \ref{enolicno} nam pove, da je rešitev za tako podan Dirichletov problem enolično določena. Vemo tudi, da je harmonična razširitev eksplicitno podana prek formule \eqref{int1}.
        Velja torej:
        $$
        1 = \int_{-\pi}^{\pi}{\left[\sum_{k=-\infty}^{\infty}{r^{|k|} e^{ik(\theta - \varphi)}}\right] \frac{d \varphi}{2 \pi}} = \int_{-\pi}^{\pi}{P_r(\theta - \varphi)\frac{d \varphi}{2 \pi}}. 
        $$
        Sedaj v integral uvedemo novo spremenljivko $\tau = \theta - \varphi$, ter upoštevamo točko a). Dobimo: 
        $$
        1 = \int_{\theta - \pi}^{\theta + \pi}{P_r(\tau)\frac{d \tau}{2 \pi}} = \left(\int_{-\pi}^{\pi} - \int_{-\pi}^{\theta -\pi} + \int_{\pi}^{\theta + \pi}\right)\left[ P_r(\tau)~\frac{d\tau}{2 \pi}\right] = \int_{-\pi}^{\pi}{P_r(\tau)\frac{d \tau}{2 \pi}}.
        $$
    \end{dokaz}

    \begin{posledica}
        Za vsak $r \in [0, 1)$ funkcija $\frac{1}{2 \pi} P_r(\theta),~ \theta \in [-\pi, \pi]$ določa gostoto zvezno porazdeljene slučajne spremenljivke.
    \end{posledica}
    \begin{dokaz}
        Iz zapisa \eqref{eq4} je jasno, da je pri poljubnem $r \in [0,1)$ zapisana funkcija v odvisnosti od $\theta$ zvezna na intervalu $[-\pi, \pi]$, točki d) in e) trditve \ref{lastpk} pa nam dokazujeta, da je na definicijskem območju funkcija strogo pozitivna in se pointegrira v $1$.
    \end{dokaz}

    \begin{definicija}
        Naj bo $\{F_{\lambda}~|~\lambda \in \Lambda,~\text{kjer je $\Lambda$ povezana podmnožica v $[0, \infty)$}\}$ družina omejenih integrabilnih funkcij, definiranih na $\partial \mathbb{D}$. Pravimo, da je družina funkcij \emph{približna enota}, če velja:
        \begin{enumerate}[label={\Alph*)}]
            \item $\int_{\partial \mathbb{D}}{F_\lambda} = 1,~\text{za vsak $\lambda \in \Lambda$}$,
            \item sup\{$\int_{\partial \mathbb{D}}{\left| F_{\lambda}\right|}~|~\lambda \in \Lambda$\} $< \infty$,
            \item za vsak $\delta > 0$ velja $\lim_{\lambda \to \mu}{\int_{\{|x| > \delta\}}{|F_{\lambda}(x)|~dx}} = 0$, kjer $\mu$ označuje supremum množice $\Lambda$, oziroma $\infty$, če supremum množice $\Lambda$ ne obstaja.
        \end{enumerate}
    \end{definicija}

    \begin{trditev}
        \label{proti0}
        Naj bo $\delta > 0$. Potem gre $\max\{P_r(\theta)~|~ \delta \leq |\theta| \leq \pi\} \to 0$ ko gre $r \to 1$.
    \end{trditev}
    \begin{dokaz}
        Ker je za vsak $r$ Poissonovo jedro v odvisnosti od spremenljivke $\theta$ na intervalu $[-\pi, \pi]$ zvezna funkcija, maksimum na kompaktu $[-\pi, \pi]$ gotovo zavzamemo. Točki a) in b) trditve \ref{lastpk} nam povesta, da bomo za vsak $r$ maksimum zavzeli kar v $\delta$. 
        Oglejmo si torej $P_r(\delta) = \frac{1 -r^2}{1 + r^2 - 2r \cos(\delta)}$. Ker je $\delta >0 $, bo imenovalec ulomka strogo pozitiven, števec ulomka pa gre proti $0$, ko gre $r \to 1$. 
        Sledi, da gre \mbox{$\max\{P_r(\theta)~|~ \delta \leq |\theta| \leq \pi\} = P_r(\delta) \to 0$}, ko gre $r \to 1$.
    \end{dokaz}
    \begin{posledica}
        \label{int_proti0}
        Naj bo $\delta > 0$. Potem gre ${\int_{\{\pi \geq |\theta| \geq \delta\}}{P_{r}(\theta)~\frac{d\theta}{2 \pi}}} \to 0$, ko gre $r \to 1$.
    \end{posledica}
    \begin{dokaz}
        Ker je interval po katerem integriramo zaprt, Poissonovo jedro pa v odvisnosti od spremenljivke $\theta$ zvezna funkcija, za vsak $r \in [0,1)$ obstaja $c_r \in \mathbb{R}$, da velja $c_r = \max\{P_r(\theta)~|~ \delta \leq |\theta| \leq \pi\}$. Zato velja:
        $$
        {\int_{\{\pi \geq |\theta| \geq \delta\}}{P_{r}(\theta)~\frac{d\theta}{2 \pi}}} \leq {\int_{\{\pi \geq |\theta| \geq \delta\}}{c_r~\frac{d\theta}{2 \pi}}} \leq c_r.
        $$
        Sedaj uporabimo trditev \ref{proti0}, ki nam pove, da gre $c_r \to 0$, ko gre $r \to 1$. Sledi, da gre tudi zapisan integral proti $0$, ko gre $r \to 1$.
    \end{dokaz}

    \begin{trditev}
        Družina funkcij $\{ \frac{1}{2 \pi} P_r~|~r \in [0,1)\}$ je približna enota.
    \end{trditev}
    \begin{dokaz}
        Zahteve glede definicijskega območja, omejenosti in integrabilnosti so izpoljene, zato je dovolj komentirati, da družina zadošča pogojem A), B) in C). 
        Da družina zadošča pogoju točke A) nam pove točka e) trditve \ref{lastpk}, točka d) trditve \ref{lastpk} pa nam pove, da je s tem zadoščeno tudi pogoju točke B). 
        Posledica \ref{int_proti0} nam pove, da družina zadošča tudi pogoju iz točke C).
    \end{dokaz}
    \begin{opomba}
        Vredno je, s pomočjo zgoraj zapisanih trditev, komentirati obnašanje funkcije Poissonovega jedra, ko pošljemo $r \to 1$. Trditev \ref{proti0} nam pove, da se vrednosti funkcije, v odvisnosti od spremenljivke $\theta$, ki so poljubno malo oddaljene od izhodišča, približujejo $0$. 
        Zapisano dodatno potrdi tudi posledica \ref{int_proti0}.
        Oglejmo si sedaj, kaj se zgodi z vrednostjo funkcije pri $\theta = 0$, ko pošljemo $r \to 1$. Iz zapisa \eqref{eq4} vemo, da velja:
        $$
        P_r(0) = \frac{1-r^2}{1+ r^2 - 2r} = \frac{1- r^2}{(1 - r)^2} = \frac{1 + r}{1 -r},~~r \in [0,1).
        $$
        Ko gre $r \to 1$, gre torej $P_r(0)$ proti $\infty$. Zapisani opažanji nam skupaj z ugotovitvijo iz točke e) trditve \ref{lastpk} nakazujeta, da se funkcija Poissonovega jedra, ko pošljemo $r \to 1$, približuje tako imenovani Diracovi delti. 
        O tem si bralec več lahko prebere na REFERENCA.
    \end{opomba}

    \begin{opomba}
        Lastnosti Poissonovega jedra, navedene v trditvi \ref{lastpk} in \ref{proti0}, so tudi dobro razvidne na spodnji sliki. 
    \end{opomba}

    \begin{figure}[H]
        \begin{center}
        \includegraphics[width=\linewidth]{grafi.png}
        \caption{Graf Poissonovega jedra glede na vrednost spremenljivke $r$.}
        \end{center}    
    \end{figure}

    \begin{trditev}
        \label{poiss_harm}
        Poissonovo jedro je na enotskem disku harmonična funkcija. 
    \end{trditev}
    \begin{dokaz}
        Trditev \ref{hh} nam pove, da je dovolj pokazati, da lahko Poissonovo jedro zapišemo kot realni del holomorfne funkcije, definirane na enotskem disku.
        Zapis \eqref{eq3} nam pove, da je Poissonovo jedro realni del funkcije $f(z) = \frac{1 + z}{1 - z},~z \in \mathbb{D}$. Ker je $f$ na enotskem disku tudi holomorfna, je trditev s tem dokazana. 
    \end{dokaz}

    Ker je enotski disk zvezdasto območje, Poissonovo jedro pa po trditvi \ref{poiss_harm} harmonična funkcija, po trditvi \ref{konj} obstaja harmonična konjugiranka Poissonovega jedra. 
    Opazimo, da smo v zapisu \eqref{eq3} Poissonovo jedro prikladno zapisali kot realni del holomorfne funkcije, zato se nam ponuja harmonična konjugiranka Poissonovega jedra kot imaginarni del zapisane holomorfne funkcije.
    Velja:
    \begin{align*}
        \text{Im}\left[\frac{1 + z}{1-z}\right] &= \frac{\frac{1 + z}{1-z} - \left(\overline{\frac{1 + z}{1-z}}\right)}{2i} = \frac{(1 + z)(1 - \overline{z}) - (1 + \overline{z})(1 - z)}{|1 - z|^2~2i} & & \\ 
        & = \frac{2 (z - \overline{z})}{|1 - z|^2~2i} = \frac{2~\text{Im}z}{|1 - z|^2} = \frac{2 r \sin(\theta)}{1+ r^2 - 2r \cos(\theta)}~; & & z = r e^{i\theta} \in \mathbb{D}.
    \end{align*}
    Prav to nas sedaj pripelje do naslednje definicije. 
    \begin{definicija}
        \emph{Konjugirano Poissonovo jedro}, je funkcija definirana s predpisom:
        \begin{align}
            Q_r(\theta) & = \frac{2 r \sin(\theta)}{1+ r^2 - 2r \cos(\theta)}~;~~r \in [0,1),~\theta \in [0,2\pi].
        \end{align}
    \end{definicija}

    %%%\begin{opomba}
    %%%    \label{par_s_t_Qr}
    %%%    Kot smo to storili v opombi \ref{par_s_t_Pr}, lahko tudi v predpisu za konjugirano Poissonovo jedro spremenljivko $\theta \in [0,2\pi]$ zamenjamo s spremenljivko $t \in [0,1]$. Dobimo:
    %%%    $$
    %%%    Q_r(\theta) = \frac{2 r \sin(2 \pi t)}{1+ r^2 - 2r \cos(2 \pi t)}\text{~,~~kjer je}~ t\in [0, 1]~\text{in}~ r < 1.
    %%%    $$
    %%%\end{opomba}

    \begin{opomba}
        Zapišimo zvezi, ki jih narekujeta zgornji komentar in definicija. Velja:
        $$
        P_r(\theta) + i~Q_r(\theta) = \left[ \frac{1-r^2}{1+ r^2 - 2r \cos(\theta)}\right] + i \left[\frac{2 r \sin(\theta)}{1+ r^2 - 2r \cos(\theta)}\right] = \frac{1 + re^{i\theta}}{1 - re^{i\theta}}~;~~r e^{i\theta} \in \mathbb{D}.
        $$
        %%%oziroma v duhu zapisa iz opomb \ref{par_s_t_Pr} in \ref{par_s_t_Qr}:
        %%%\begin{align*}
        %%%    P_r(t) + i~Q_r(t) &= \left[ \frac{1-r^2}{1+ r^2 - 2r \cos(2 \pi t)}\right] + i \left[\frac{2 r \sin(2 \pi t)}{1+ r^2 - 2r \cos(2 \pi t)}\right] \\ 
        %%%    &=  \text{Re}\left[\frac{1 + re^{2 \pi i t}}{1 - re^{2 \pi i t}}\right] + i~\text{Im}\left[\frac{1 + re^{2 \pi i t}}{1 - re^{2 \pi i t}}\right]=\frac{1 + re^{2 \pi i t}}{1 - re^{2 \pi i t}};~t\in [0, 1], ~ r < 1.
        %%%\end{align*}        
    \end{opomba}

    Sedaj se vrnimo k reševanju Dirichletovega problema za enotski disk. 
    Spomnimo se, da smo za polinomske funkcije $h$ skonstruirali predpis razširitve, ki je rešila zastavljen problem. 
    Opazimo, da bi si pri zapisu enačbe \eqref{int1} lahko pomagali z definiranim pojmom Poissonovega jedra. Lahko zapišemo:
    
    $$
    H(r e^{i \theta}) = \int_{-\pi}^{\pi}{h(e^{i \varphi}) \bigg[\sum_{k = - \infty}^{\infty} r^{|k|} e^{- i k \varphi} e^{i k \theta}} \bigg]\frac{d \varphi}{2 \pi} = 
    \int_{-\pi}^{\pi}{h(e^{i \varphi}) P_r(\theta - \varphi)\frac{d \varphi}{2 \pi}},~~r e^{i \theta} \in \overline{\mathbb{D}}.
    $$
    Zgoraj smo z intuitivno izpeljavo nevede že zapisali funkcijo, ki jo bomo sedaj vzeli za definicijo novega pojma. 
    Pokazali bomo, da to tudi pri splošnih zveznih funkcijah omogoča zapis rešitve Dirichletovega problema na enotskem disku.

\subsection{Poissonov integral}
    \begin{definicija}
        Naj bo $h(e^{i \theta})$ zvezna funkcija, definirana na robu enotskega diska.
        \emph{Poissonov integral} funkcije $h(e^{i\theta})$, ki ga označimo s $\widetilde{h}(z)$, je funkcija, definirana na notranjosti enotskega diska s predpisom:
        $$
        \widetilde{h}(z) = \int_{0}^{2\pi}{h(e^{i\varphi}) P_r(\theta - \varphi)~\frac{d\varphi}{2 \pi}}~\text{, kjer je}~~z = r e^{i\theta} \in \mathbb{D}.
        $$
     \end{definicija}
     \begin{opomba}
        \label{kom_poiss}
        Če v zgornji integral uvedemo novo spremenljivko $\tau = \theta - \varphi$, ter upoštevamo točko a) trditve \ref{lastpk}, lahko Poissonov integral funkcije $h$, zapišemo tudi kot:
        $$
        \widetilde{h}(z) = \int_{0}^{2\pi}{h\big(e^{i(\theta-\varphi)}\big) P_r(\varphi)~\frac{d\varphi}{2 \pi}}~\text{, kjer}~~z = r e^{i\theta} \in \mathbb{D}.
        $$
        Natančneje si opisan postopek, ki pripelje do zgornjega zapisa Poissonovega integrala bralec lahko ogleda v dokazu točke e) trditve \ref{lastpk}.
     \end{opomba}

     \begin{opomba}
        \label{opomba_konv1}
        %%%Spomnimo se alternativnega zapisa Poissonovega jedra, ki smo ga navedli v \eqref{poisson_t}. 
        %%%Opazimo, da bi se do takšnega zapisa Poissonovega jedra znotraj Poissonovega integrala lahko dokopali, z vpeljavo spremenljivke $s = \frac{\varphi}{ 2\pi}$, ter oznako $t = \frac{\theta}{2 \pi}$. 
        %%%Takrat namreč predpis za Poissonov integral postane:
        Če definiramo funkcijo $f(w) = h(e^{iw})$, lahko Poissonov integral zapišemo še nekoliko drugače. Velja:
        \begin{align}
            \label{konv_Pr}
            \widetilde{h}(z) &= \int_{0}^{2\pi}{h(e^{i\varphi}) P_r(\theta - \varphi)~\frac{d\varphi}{2 \pi}} = \int_{0}^{2\pi}{f(\varphi) P_r(\theta - \varphi)~\frac{d\varphi}{2 \pi}} \notag \\
            & = \frac{1}{2\pi} \left[(f * P_r)(\theta)\right] = \frac{1}{2\pi} \left[(P_r * f)(\theta)\right],~~z = r e^{i\theta},~\theta \in [0,2\pi],~r\in[0,1).
        \end{align}
        %$$Sedaj definirajmo funkcijo $f(w) = h(e^{(2 \pi i) w})$, ki nam omogoča zapis:
        %$$$$
        %$$\widetilde{h}\left(r e^{2 \pi i t}\right) = \int_{0}^{1}{f(s) P_r(t - s)~ds},~\text{za}~~t \in [0,1]~\text{in}~r < 1.
        %$$$$
        %$$Na tej točki se nam vse redukcije zapisa obrestujejo. V zgornjem zapisu Poissonovega integrala lahko namreč prepoznamo konvolucijo funkcije $f$ in Poissonovega jedra $P_r(t)$.
        %%%Velja torej:
        %%%\begin{equation}
        %%%\label{konv_Pr}
        %%%    \widetilde{h}\left(r e^{2 \pi i t}\right) = (f * P_r)(t) = (P_r * f)(t),~\text{za}~~t \in [0,1]~\text{in}~r < 1.
        %%%\end{equation}
        Zadnja enakost v zgornjem zapisu je posledica komutativnosti konvolucije, ter še na nekoliko drugačen način dokazuje zapis iz opombe \ref{kom_poiss}. Velja namreč:
        \begin{align*}
            \widetilde{h}\left(r e^{i \theta}\right) &= \frac{1}{2\pi} \left[(P_r * f)(\theta)\right] =  \frac{1}{2 \pi}\int_{0}^{2\pi}{P_r(\varphi) f(\theta - \varphi) ~d\varphi} \\
            & = \int_{0}^{2\pi}{h\big(e^{i(\theta-\varphi)}\big) P_r(\varphi)~\frac{d\varphi}{2 \pi}},~~z = r e^{i\theta},~\theta \in [0,2\pi],~r\in[0,1). \notag
        \end{align*}
     \end{opomba}

     \begin{opomba}
        \label{opomba_konv2}
        Ogledali smo si alternativni zapis Poissonovega integrala s konvolucijo Poissonovega jedra. Smiselno se je vprašati, kako na zgornji postopek zapisa s konvolucijo vpliva zamenjava Poissonovega jedra s konjugiranim Poissonovim jedrom. Opazimo, da lahko po enakem postopku tudi integral s konjugiranim Poissonovim jedrom izrazimo kot konvolucijo. 
        Ponovno definiramo $f(w) = h(e^{i w})$, ki nam omogoča zapis:
        \begin{equation*}
            \int_{0}^{2\pi}{h(e^{i\varphi}) Q_r(\theta - \varphi)~\frac{d\varphi}{2 \pi}}  = \frac{1}{2\pi}\int_{0}^{2 \pi}{f(\varphi) Q_r(\theta - \varphi)~d\varphi} = \frac{1}{2 \pi}[(f * Q_r)(\theta)],~re^{i\theta} \in \mathbb{D}.
        \end{equation*}
     \end{opomba}
     Konvolucijska zapisa integrala iz opombe \ref{opomba_konv1} in \ref{opomba_konv2} bomo uporabili nekoliko kasneje, sedaj pa se še nekoliko posvetimo lastnostim Poissonovega integrala.

     \begin{trditev}
        \label{lastpi}
        Naj bo $\Phi$ preslikava, ki zvezni funkciji $h$, definirani na robu enotskega diska, prireredi njen Poissonov integral $\widetilde{h}$, t.j. $\Phi : h \mapsto \widetilde{h}$.
        Potem velja:
        \begin{enumerate}[label={\alph*)}]
            \item $\Phi$ je $\mathbb{C}$-linearna preslikava, t.j. \mbox{$\text{za vsak}~h_1,h_2 \in C^0(\partial \mathbb{D})~\text{in vsak}~c_1,c_2 \in \mathbb{C}$} velja: $\Phi(c_1 h_1 + c_2 h_2) = c_1 \Phi(h_1) + c_2 \Phi(h_2)$,
            \item $\Phi$ ohranja omejenost, t.j. če za $h \in C^0(\partial \mathbb{D})$ in $M \in \mathbb{R}$ velja $|h(z)| \leq M$, za vsak $z \in \partial \mathbb{D}$, potem je $|[\Phi(h)](z)| \leq M$ za vsak $z \in \mathbb{D}$.
        \end{enumerate}
     \end{trditev}
     \begin{dokaz}
            Dokažimo najprej točko a). Naj bosta $h_1$ in $h_2$ poljubni zvezni funkciji, definirani na robu enotskega diska, ter $c_1, c_2$ poljubni kompleksni števili. 
            Potem je na robu enotskega diska zvezna tudi funkcija $c_1 h_1 + c_2 h_2$. Oglejmo si sedaj najprej Poissonov integral za zapisano linearno kombinacijo. Velja:
            \begin{align*}
                [\widetilde{c_1 h_1 + c_2 h_2}](z) &= \int_{-\pi}^{\pi}{\left([c_1 h_1 + c_2 h_2](e^{i\varphi}) \right)P_r(\theta - \varphi)~\frac{d\varphi}{2 \pi}}\\ 
                & = \int_{-\pi}^{\pi}{\left([c_1 h_1](e^{i\varphi}) + [c_2 h_2](e^{i\varphi})\right)P_r(\theta - \varphi)~\frac{d\varphi}{2 \pi}}\\
                & = c_1\int_{-\pi}^{\pi}{h_1(e^{i\varphi})P_r(\theta - \varphi)~\frac{d\varphi}{2 \pi}} + c_2\int_{-\pi}^{\pi}{h_2(e^{i\varphi})P_r(\theta - \varphi)~\frac{d\varphi}{2 \pi}}\\
                & = c_1 \widetilde{h_1}(z) + c_2 \widetilde{h_2}(z) = [c_1 \widetilde{h_1} + c_2 \widetilde{h_2}](z)~,~~~~z \in \mathbb{D}.
            \end{align*}
            Po definiciji preslikave $\Phi$ sedaj sledi $\Phi(c_1 h_1 + c_2 h_2) = c_1 \Phi(h_1) + c_2 \Phi(h_2)$.

            Dokažimo še točko b). Naj bo $h$ poljubna omejena zvezna funkcija, definirana na robu enotskega diska. Naj za $M \in \mathbb{R}$ velja $|h(z)| \leq M$, za vsak $z \in \partial \mathbb{D}$.
            Ponovno si najprej oglejmo Poissonov integral funkcije $h$ in uporabimo zapisano neenakost, ter točki d) in e) trditve \ref{lastpk}. Dobimo:
            \begin{align*}
                \left|\widetilde{h}(z)\right| &= \left| \int_{-\pi}^{\pi}{h(e^{i\varphi}) P_r(\theta - \varphi)~\frac{d\varphi}{2 \pi}} \right| \leq \int_{-\pi}^{\pi}{\left|h(e^{i\varphi}) \right|P_r(\theta - \varphi)~\frac{d\varphi}{2 \pi}} \\ 
                &\leq M \int_{-\pi}^{\pi}{P_r(\theta - \varphi)~\frac{d\varphi}{2 \pi}} = M,~~~~z \in \mathbb{D}.& & \\
            \end{align*}
            Po definiciji preslikave $\Phi$ zato $|[\Phi(h)](z)| \leq M$, za vsak $z \in \mathbb{D}$.
     \end{dokaz}

\subsection{Schwarzova integralska formula}
     \begin{lema}
        \label{realnidel}
        Naj bo $u$ realna zvezna funkcija, definirana na $\partial \mathbb{D}$. Potem je Poissonov integral funkcije $u$ na $\mathbb{D}$ harmonična funkcija.
     \end{lema}
     \begin{dokaz}
        Pomagajmo si z zapisom \eqref{eq3}. Za $z = re^{i\theta} \in \mathbb{D}$ lahko zapišemo:
            \begin{align}
                \label{lema_schwarz}
                \widetilde{u}(z) &= \widetilde{u}(r e^{i \theta}) = \int_{-\pi}^{\pi}{u(e^{i \varphi}) P_r(\theta - \varphi)\frac{d \varphi}{2 \pi}} = \int_{-\pi}^{\pi}{u(e^{i \varphi})~\text{Re}\bigg(\frac{1+re^{i(\theta - \varphi)}}{1-re^{i(\theta - \varphi)}}\bigg)\frac{d \varphi}{2 \pi}} \notag \\
                &= \int_{-\pi}^{\pi}{u(e^{i \varphi})~\text{Re}\bigg(\frac{e^{i\varphi}+re^{i\theta}}{e^{i\varphi}-re^{i\theta}}\bigg)\frac{d \varphi}{2 \pi}}=~\text{Re}~\bigg[\int_{-\pi}^{\pi}{u(e^{i \varphi})\bigg(\frac{e^{i\varphi}+z}{e^{i\varphi}-z}\bigg)\frac{d \varphi}{2 \pi}}\bigg].
            \end{align}
        Tako smo $\widetilde{u}(z)$ za $z \in \mathbb{D}$ izrazili kot realni del holomorfne funkcije, kar po trditvi \ref{hh} dokazuje harmoničnost Poissonovega integrala funkcije $u$, na notranjosti enotskega diska.
     \end{dokaz}

     Osredotočimo se na predpis holomorfne funkcije iz zapisa \eqref{lema_schwarz}. Opazimo, da smo jo konstruirali s pomočjo podane realne funkcije $u$. 
     Sedaj začnimo nekoliko drugače. Denimo da imamo podano funkcijo $F = U + iV$, ki je na $\overline{\mathbb{D}}$ zvezna in na $\mathbb{D}$ holomorfna. Vredno je omeniti, da sta tu v zapisu $U$ in $V$ realni funkciji, ki predstavljata realni oziroma imaginarni del funkcije $F$, ter sta zato na $\overline{\mathbb{D}}$ zvezni in na $\mathbb{D}$ harmonični.
     Iz realnega dela holomorfne funkcije $F$, torej funkcije $U$, lahko na enak način kot smo to storili v \eqref{lema_schwarz}, konstruiramo holomorfno funkcijo $\widetilde{F}$ s predpisom:
     \begin{equation}
        \label{priprav_schwarz}
        \widetilde{F}(z) = \int_{0}^{2 \pi}{U(e^{i \varphi})\bigg(\frac{e^{i\varphi}+z}{e^{i\varphi}-z}\bigg)\frac{d \varphi}{2 \pi}},~z \in \mathbb{D}.
    \end{equation}
    Enakost \eqref{lema_schwarz} nam pove, da za $z = r e^{i\theta}\in \mathbb{D}$ velja $\text{Re}(\widetilde{F})(z) = U(z)$. Smiselno se je vprašati ali lahko kaj podobnega na $\mathbb{D}$ trdimo tudi za $\widetilde{F}$ in $F$. 
    O tem nam več pove naslednja trditev. 

    \begin{trditev}[Schwarzova integralska formula]
        \label{schwarz_int_f}
        Naj bo $F = U + iV$ funkcija, holomorfna na $\mathbb{D}$ in zvezna na $\overline{\mathbb{D}}$. Potem velja: 
        $$ 
            F(z) = \int_{0}^{2 \pi}{U(e^{i \varphi})\bigg(\frac{e^{i\varphi}+z}{e^{i\varphi}-z}\bigg)\frac{d \varphi}{2 \pi}} + i V(0),~z \in \mathbb{D}.
        $$
    \end{trditev}
    \begin{dokaz}
        Enako kot smo to storili v \eqref{lema_schwarz} in \eqref{priprav_schwarz}, lahko definiramo holomorfno funkcijo:
        $$
        \widetilde{F}(z) = \int_{0}^{2 \pi}{U(e^{i \varphi})\bigg(\frac{e^{i\varphi}+z}{e^{i\varphi}-z}\bigg)\frac{d \varphi}{2 \pi}},~z \in \mathbb{D}.
        $$
        Komentirali smo že, da po \eqref{lema_schwarz} velja $\text{Re}(\widetilde{F}) \equiv \text{Re}(F)$, oziroma $\text{Re}[\widetilde{F} - F](z) = 0$, za vsak $z \in \mathbb{D}$. Razlika holomorfnih funkcij zadošča Cauchy-Riemannovemu sistem enačb, zato obstaja $C \in \mathbb{C}$, da velja $\text{Im}[\widetilde{F} - F](z) = C$ za vsak $z \in \mathbb{D}$.
        Ker je $U$ na $\mathbb{D}$ harmonična, ima na $\mathbb{D}$ lastnost povprečne vrednosti. Zato velja:
        $$
        \widetilde{F}(0) = \int_{0}^{2 \pi}{U(e^{i \varphi})\bigg(\frac{e^{i\varphi}+0}{e^{i\varphi}-0}\bigg)\frac{d \varphi}{2 \pi}} = \int_{0}^{2 \pi}{U(e^{i \varphi})\frac{d \varphi}{2 \pi}} = U(0).
        $$
        Ker je $F(0) = U(0) + iV(0)$, sledi $\text{Im}[F - \widetilde{F}] \equiv V(0)$.
        Torej velja:
        $$ 
        F(z) = \widetilde{F}(z) + iV(0) = \int_{0}^{2 \pi}{U(e^{i \varphi})\bigg(\frac{e^{i\varphi}+z}{e^{i\varphi}-z}\bigg)\frac{d \varphi}{2 \pi}} + i V(0),~z \in \mathbb{D}.
        $$
    \end{dokaz}

    \begin{opomba}
        Spomnimo se zapisa iz opomb \ref{opomba_konv1} in \ref{opomba_konv2}. Schwarzovo integralsko formulo, iz trditve \ref{schwarz_int_f}, lahko za holomorfno funkcijo $F = U + iV$ lahko alternativno zapišemo kot konvolucijo. Velja:
        $$
            F(r e^{i \theta}) = \int_{0}^{2 \pi}{U(e^{i \varphi})\left[\frac{e^{i \varphi}+r e^{i \theta}}{e^{i \varphi}-r e^{i \theta}}\right]~\frac{d \varphi}{2 \pi}} + i V(0)~;~~~~\theta \in [0,2 \pi],~r \in [0,1).
        $$
        Sedaj označimo $G(w) = U(e^{iw})$, ter poenostavimo. Dobimo:
        \begin{align*}
            F(r e^{i \theta}) & = \int_{0}^{2 \pi}{G(\varphi)\left[\frac{1 +r e^{i (\theta-\varphi)}}{1-r e^{i(\theta-\varphi)}}\right]}~\frac{d\varphi}{2 \pi} + iV(0)\\
            %%& =\int_{0}^{1}{G(s)\left[\text{Re}\left[\frac{1 +r e^{2 \pi i (t-s)}}{1-r e^{2 \pi i(t-s)}}\right] + i~\text{Im}\left[\frac{1 +r e^{2 \pi i (t-s)}}{1-r e^{2 \pi i(t-s)}}\right] \right]}ds + i V(0) \\
            & = \frac{1}{2 \pi} \left[\int_{0}^{2 \pi}{G(\varphi)[(P_r + i Q_r)(\theta - \varphi)]}~d\varphi\right]+  iV(0) \\
            & = \frac{1}{2 \pi} \big[G * (P_r + iQ_r)\big](\theta) + iV(0)\\
            %& = \left[(G * P_r) + i(G*Q_r)\right](t) + iV(0)\\
            & = \frac{1}{2 \pi}\bigg[(G * P_r)(t) + i\big[(G * Q_r)(t) + 2 \pi V(0)\big]\bigg]~,~~~~\theta\in [0,2 \pi],~r \in [0,1).
        \end{align*}
    \end{opomba}

\subsection{Rešitev Dirichletovega problema na enotskem disku}
    Vrnimo se k bistvu razdelka. Za preproste zvezne funkcije smo že dokazali, da rešitev Dirichletovega problema na enotskem disku obstaja. To nas je pripeljalo do definicije Poissonovega jedra in Poissonovega integrala, s pomočjo katerih bomo sedaj pokazali, da za poljubno zvezno funkcijo obstaja rešitev Dirichletovega problema na enotskem disku.
    \begin{trditev}
        \label{obstoj}
        Naj bo $h$ zvezna kompleksna funkcija, definirana na $\partial \mathbb{D}$. Rešitev Dirichletovega problema, z robnim pogojem $h$, za enotski disk obstaja in je na $\mathbb{D}$ definirana kot Poissonov integral funkcije $h$.
    \end{trditev}
    \begin{opomba}
        \label{opomba_obstoj}
        Pred dokazom trditve še na nekoliko drugačen način zapišimo, kaj nam trditev v resnici pove. 
        Definirajmo funkcijo $H$, na $\overline{\mathbb{D}}$, kot:
        $$
            H(z) = \begin{cases}
                    h(z);~~z \in \partial \mathbb{D}\\
                    \widetilde{h}(z);~~z \in \mathbb{D}
            \end{cases}.
        $$
        Trdimo, da je funkcija $H$ na $\overline{\mathbb{D}}$ zvezna, ter na $\mathbb{D}$ harmonična. Po definiciji funkcije $H$ opazimo, da v resnici trdimo, da je na $\mathbb{D}$ harmoničen Poissonov integral funkcije $h$. 
        Ker se zožitev tako definirane funkcije $H$ na $\partial \mathbb{D}$ ujema s $h$, želimo dokazati, da je ravno funkcija $H$ rešitev za Dirichletov problem na enotskem disku, z robnim pogojem $h$. 
        Formulacijo problema smo predstavili tudi grafično, pa naredimo to še za rešitev problema. 
        \begin{figure}[H]
            \begin{center}
                \includegraphics[width = 0.7 \textwidth]{dirichlet_resitev.png}
                \caption{Rešitev Dirichletovega problema na enotskem disku, z robnim pogojem $h$.}
            \end{center}    
        \end{figure}
     \end{opomba}

     \begin{dokaz}
        Trditev bomo dokazali v dveh korakih. Najprej dokažimo, da je Poissonov integral na $\mathbb{D}$ harmonična funkcija.
        Vemo, da lahko kompleksno funkcijo $h$ razcepimo na njen realni in imaginarni del kot $h = u + iv$, kjer sta $u$ in $v$ realni funkciji. 
        Točka a) trditve \ref{lastpi} nam pove, da velja $\widetilde{h} = \widetilde{u} + i \widetilde{v}$.
        Ker sta $u$ in $v$ realni zvezni funkciji, definirani na robu enotskega diska, lahko uporabimo lemo \ref{realnidel}. Ta nam pove, da sta $\widetilde{u}$ in $\widetilde{v}$ na $\mathbb{D}$ harmonični funkciji. 
        Dokaz prvega dela zaključi opomba \ref{lin}, ki pove, da je potem tudi funkcija $\widetilde{h} = \widetilde{u} + i \widetilde{v}$ na $\mathbb{D}$ harmonična funkcija.
        
        Za dokaz zveznosti na $\overline{\mathbb{D}}$ se moramo nekoliko bolj potruditi. Tako kot smo to storili v opombi \ref{opomba_obstoj}, s $H$ označimo funkcijo, ki se na robu enotskega diska ujema s $h$, na notranjosti enotskega diska pa je definirana kot Poissonov integral funkcije $h$. 
        Na  $\mathbb{D}$ je $\widetilde{h}$ oziroma $H$ harmonična, zato je tam tudi zvezna. Predpostavka trditve nam pove, da je $h$ zvezna na $\partial \mathbb{D}$, zato po definiciji tudi $H$. 
        Dovolj je torej dokazati, da je funkcija $H$ zvezna do roba enotskega diska.
        
        Naj bo $\epsilon >0$. Dokazati je potrebno, da obstaja $\delta >0$, da za vsak $z \in \mathbb{D}$ in $w \in \partial \mathbb{D}$ iz $|z - w| < \delta$ sledi $|H(z) - H(w)| < \epsilon$. 
        Označimo $z = r e^{i \theta},~\theta \in [0,2\pi]~\text{in}~r \in [0,1)$, ter $w = e^{i \tau},~\tau \in [0,2\pi]$.
        Po definiciji funkcije $H$ torej iščemo $\delta >0$, da bo za $\theta, \tau \in [0,2\pi]~\text{in}~r \in [0,1)$ iz $|r e^{i \theta} - e^{i\tau}| < \delta$ sledilo $|\widetilde{h}(r e^{i \theta}) - h(e^{i\tau})| < \epsilon$.
        Pomagajmo si s trikotniško neenakostjo. Velja:
        \begin{align}
            \label{trik_ocena}
            |\widetilde{h}(r e^{i \theta}) - h(e^{i\tau})| &= |\widetilde{h}(r e^{i \theta}) - h(e^{i \theta}) + h(e^{i \theta}) - h(e^{i\tau})| \notag \\
            &\leq |\widetilde{h}(r e^{i \theta}) - h(e^{i\theta})| + |h(e^{i\theta}) - h(e^{i\tau})|.
        \end{align}
        Ocenimo najprej prvi sumand iz zapisa \eqref{trik_ocena}. Uporabimo točko e) trditve \ref{lastpk}, ter po definiciji Poissonovega integrala zapišemo: 
        $$
            \left|\widetilde{h}(r e^{i \theta}) - h(e^{i\theta})\right| = \left|\int_{-\pi}^{\pi}{\left[h\left(e^{i(\theta - \varphi)}\right) - h\left(e^{i\theta}\right)\right]P_r(\varphi)~\frac{d\varphi}{2 \pi}}\right|.
        $$
        Sedaj uporabimo točko d) trditve \ref{lastpk}, ter ocenimo:
        $$
        \left|\widetilde{h}(r e^{i \theta}) - h(e^{i\theta})\right| \leq \int_{-\pi}^{\pi}{\left| h\left(e^{i\theta - \varphi)}\right) - h \left(e^{i\theta}\right) \right|P_r(\varphi)~\frac{d\varphi}{2 \pi}}.
        $$ 

        Ker je $h$ zvezna funkcija na kompaktu $\partial \mathbb{D}$, je omejena in enakomerno zvezna.  
        Obstaja torej $M \in \mathbb{R}$, da velja $|h(z)| \leq M$, za vsak $z \in \partial \mathbb{D}$, ter $\delta_0 >0$, da za vsak $\mu, \lambda \in [0,2\pi]$ iz $|\mu - \lambda| < \delta_0$ sledi $|h(e^{i \mu}) - h(e^{i \lambda})| < \frac{\epsilon}{3}$.
        Zgornji integral, lahko sedaj razdelimo na dva dela, da bomo lahko uporabili enakomerno zveznost funkcije $h$. Dobimo:
        $$
        \left|\widetilde{h}(re^{i\theta}) - h(e^{i\theta})\right| \leq \left(\int_{-\delta_0}^{\delta_0} + \int_{\delta_0 \leq |\varphi| \le \pi}\right){\left[\left| h\big(e^{i(\theta - \varphi)}\big) - h(e^{i\theta})\right|P_r(\varphi)\frac{d\varphi}{2\pi}\right]}.
        $$
        Absolutno vrednost znotraj prvega integral lahko zaradi enakomerne zveznosti funkcije $h$ navzgor ocenimo z $\frac{\epsilon}{3}$, vrednosti znotraj drugega integrala pa lahko zaradi omejenosti funkcije $h$ navzgor ocenimo kar z $2M$. Velja torej:
        $$
        \left|\widetilde{h}(re^{i\theta}) - h(e^{i\theta})\right| < \frac{\epsilon}{3} \int_{-\delta_0}^{\delta_0}{P_r(\varphi) \frac{d\varphi}{2\pi}} + 2M\int_{\delta_0 \leq |\varphi| \leq \pi}{P_r(\varphi)\frac{d\varphi}{2\pi}}.
        $$
        Vsakega od integralov lahko sedaj navzgor ocenimo s pomočjo točke e) trditve \ref{lastpk}. Velja:
        $$
        \left|\widetilde{h}(re^{i\theta}) - h(e^{i\theta})\right| < \frac{\epsilon}{3}  + 2M~\text{max}\{P_r(\varphi)~| ~\delta_0 \leq |\varphi| \leq \pi \}.
        $$
        Trditev \ref{proti0} nam pove, da obstaja $\delta_1 >0$, da iz $1 - r < \delta_1$ sledi $\text{max}\{P_r(\varphi)~| ~\delta_0 \leq |\varphi| \leq \pi \} < \frac{\epsilon}{6M}$.
        Zato lahko pri zapisanih pogojih ocenimo:
        $$
        \left|\widetilde{h}(re^{i\theta}) - h(e^{i\theta})\right| < \frac{2 \epsilon}{3}.
        $$
        Ocenimo še drugi sumand, ki smo ga dobili pri \eqref{trik_ocena}. Enakomerna zveznost nam pove, da iz $|\theta - \tau| < \delta_0$ sledi $|h\left(e^{i\theta}\right) - h\left(e^{i\tau}\right)| < \frac{\epsilon}{3}$.
        
        Združimo oceni in konstruirajmo $\delta>0$, ki bo zadostil pogoju za zveznost. Dokazali smo, da za vsak $\theta, \tau \in [0,2\pi]$ in vsak $r \in [0,1)$ iz $|\theta - \tau| < \delta_0$, ter $1- r < \delta_1$ sledi:
        $$
        \left|\widetilde{h}(r e^{i \theta}) - h(e^{i\tau})\right| \leq \left|\widetilde{h}(re^{i\theta}) - h(e^{i\theta})\right| + \left|h\left(e^{i\theta}\right) - h\left(e^{i\tau}\right)\right| < \frac{2 \epsilon}{3} + \frac{\epsilon}{3} = \epsilon.
        $$
        Zadošča $\delta = \text{min}\left\{\delta_1, \frac{\delta_0}{4}\right\}$, saj iz $|r e^{i \theta} - e^{i\tau}| < \delta$ sledi: 
        $$ 
            \delta_1 \geq \delta > |r e^{i \theta} - e^{i\tau}| \geq \left||r e^{i \theta}| - |e^{i\tau}|\right| \geq |r - 1| \geq 1 -r, 
        $$
        ter:
        $$ 
            \delta_0 \geq 4 \delta > 2 \left[\left| e^{i \theta} - r e^{i\theta} \right| +  \left|r e^{i\theta} - e^{i\tau} \right| \right] \geq 2 \left|e^{i\theta} - e^{i\tau} \right| \geq |\theta - \tau|.
        $$
    \end{dokaz}

    \begin{posledica}
        Dirichletov problem za enotski disk je dobro postavljen matematičen problem. 
    \end{posledica}
    \begin{dokaz}
        Dokažimo, da Dirichletov problem za enotski disk ustreza zahtevam A), B) in C) iz definicije \ref{def_dp}.
        Zahtevo A), oziroma obstoj rešitve za vsako zvezno funkcijo definirani na robu enotskega diska, nam preko eksplicitne konstrukcije zagotavlja trditev \ref{obstoj}. Enoličnost rešitve, oziroma zahtevo B), pa nam potrjuje lema \ref{enolicno}. 
        
        Osredotočimo se na zahtevo C). 
        Naj bo $\epsilon > 0$ poljubno majhen in $h$ zvezna funkcija, definirana na $\partial \mathbb{D}$. 
        Naj bo $g$ poljubna zvezna funkcija, definirana na $\partial \mathbb{D}$, za katero za vsak $z \in \partial \mathbb{D}$ velja: $|h(z) - g(z)| < \epsilon$. 
        Definirajmo \mbox{$f(z) = h(z) - g(z)$}. Opazimo, da velja $|f(z)| < \epsilon$, za vsak $z \in \partial \mathbb{D}$. Točka b) trditve \ref{lastpi} nam pove, da potem za vsak $z \in \mathbb{D}$ velja $|\widetilde{f}(z)| < \epsilon$, po točki a) trditve \ref{lastpi} pa sledi \mbox{$|\widetilde{h}(z) - \widetilde{g}(z)| < \epsilon$}, za vsak $z \in \mathbb{D}$.
        Definirajmo funkciji $H$ in $G$ kot:
        $$
            H(z) = \begin{cases}
                    h(z);~~z \in \partial \mathbb{D}\\
                    \widetilde{h}(z);~~z \in \mathbb{D}
            \end{cases},~~~~~~~
            G(z) = \begin{cases}
                g(z);~~z \in \partial \mathbb{D}\\
                \widetilde{g}(z);~~z \in \mathbb{D}
            \end{cases}.
        $$
        Trditev \ref{obstoj} nam pove, da je rešitev za Dirichletov problem na enotskem disku, z začetnim pogojem $h$, funkcija $H$, za začetni pogoj $g$ pa funkcija $G$.
        Zapisali smo, da se funkciji $G$ in $H$ na $\partial \mathbb{D}$ in $\mathbb{D}$ po absolutni vrednosti razlikujeta za manj kot $\epsilon$. Sledi, da velja $|H(z) - G(z)| < \epsilon$ za vsak $z \in \overline{\mathbb{D}}$.     
        Ker je bil $\epsilon$ poljubno majhen, smo s tem dokazali zvezno odvisnost problema od začetnih pogojev.
    \end{dokaz}

    Pred nadaljevanjem Poissonov integral zapišimo nekoliko drugače. Opazimo, da gre v resnici za krivuljni integral po robu enotskega diska, pri čemer smo rob enotskega diska parametrizirali s funkcijo $e^{it},~t \in [0,2 \pi]$. 
    Za nadaljevanje nam bo prav prišel zapis s krivujnim integralom, zato si ga izpeljimo. Po definiciji Poissonovega integrala in Poissonovega jedra velja:
    \begin{align*}
        \widetilde{h}(r e^{i\theta}) &= \int_{0}^{2\pi}{h(e^{i\varphi}) P_r(\theta - \varphi)~\frac{d\varphi}{2 \pi}} = \int_{0}^{2\pi}{h(e^{i\varphi}) \left[\frac{1 - |r e^{i (\theta - \varphi)}|^2}{|1 - r e^{i (\theta - \varphi)}|^2}\right]\frac{d\varphi}{2 \pi}} \\
        & = \int_{0}^{2\pi}{h(e^{i\varphi}) \left[\frac{1 - |r e^{i \theta}|^2}{|e^{i \varphi} - r e^{i \theta}|^2}\right]\frac{d\varphi}{2 \pi}};~~ r \in [0,1),~\theta \in [0,2 \pi].
    \end{align*}
    Sedaj označimo $z = re^{i \theta}$ in v integral vpeljimo $w = e^{i \varphi}$. Velja:
    \begin{align}
        \label{pi_kompl}
        \widetilde{h}(z) &= \frac{1}{2 \pi}\int_{0}^{2\pi}{h(e^{i\varphi}) \left[\frac{1 - |ze^{ -i\varphi}|^2}{|1 - ze^{-i\varphi}|^2}\right]\frac{d\varphi}{2 \pi}}= \frac{1}{2\pi}\int_{0}^{2\pi}{h(e^{i\varphi}) \left[\frac{|1 - |z|^2}{|e^{i \varphi} - z|^2}\right]\frac{d\varphi}{2 \pi}} \notag \\
        & = \frac{1}{2\pi i}\int_{\partial \mathbb{D}}{h(w) \left[\frac{1 - |z|^2}{|w - z|^2}\right]\frac{dw}{w}} =  \frac{1}{2\pi i}~\text{Re}\left[\int_{\partial \mathbb{D}}{h(w) \left(\frac{w + z }{w - z}\right)\frac{dw}{w}}\right];~ z \in \mathbb{D}.
    \end{align}

\subsection{Dirichletov problem na splošnejših območjih}
    Dirichletov problem smo do sedaj formulirali in reševali za enotski disk. Problem bi na podoben način lahko formulirali za poljubno območje $U \subseteq \mathbb{C}$. 
    Kot začetni pogoj bi podali zvezno funkcijo, definirano na $\partial U$, ter iskali njeno razširitev na $\overline{U}$, tako da bi bila na $\overline{U}$ zvezna, ter na $U$ harmonična.
    
    V splošnem Dirichletovega problema znotraj diplomske naloge ne bomo reševali, bomo pa komentirali obstoj in konstrukcijo rešitve, v primeru ko je $U$ omejeno enostavno povezano območje.
    Na reševanje tako zastavljenega problema, smo se v resnici pripravili z zapisom Poissonovega integrala s kompleksno spremenljivko.

    Nakažimo, kako se lotimo reševanja Dirichletovega problema za omejeno enostavno povezano območje $\Omega$. Naj bo $f$ zvezna funkcija, definirana na $\partial \Omega$. 
    Riemannov upodobitven izrek nam pove, da je $\Omega$ konformno ekvivalentna enotskemu disku. Denimo, da lahko biholomorfizem razširimo do biholomorfizma $\Phi: \overline{\Omega} \to \overline{\mathbb{D}}$. 
    Označimo $\Psi = \Phi^{-1}$, ter definirajmo $g = f \circ \Psi: \partial \mathbb{D} \to \mathbb{C}$. Funkcija $g$ zadošča pogojem za Dirichletov problem na enotskem disku, ki ga znamo rešiti s pomočjo Poissonovega integrala. 
    Tu nam sedaj na pomoč priskoči zapis \eqref{pi_kompl}. Velja:
    $$
    \widetilde{g}(z) = [\widetilde{f \circ \Psi}](z) = \frac{1}{2\pi i}\int_{\partial \mathbb{D}}{[f \circ \Psi](w) \left[\frac{1 - |z|^2}{|w - z|^2}\right]\frac{dw}{w}};~~z \in \mathbb{D}. 
    $$
    Sedaj v integral uvedimo novo spremenljivko, $\tau = \Psi(w)$, da bomo integrirali po $\partial \Omega$: 
    $$
    [\widetilde{f \circ \Psi}](z) = \frac{1}{2\pi i}\int_{\partial \Omega}{f(\tau) \left[\frac{1 - |z|^2}{|\Phi(\tau) - z|^2}\right]\frac{|\Phi'(\tau)|}{\Phi(\tau)}~d \tau};~~z \in \mathbb{D}. 
    $$
    Sedaj z biholomorfizmom slikajmo v drugo smer. Označimo $\xi = \Psi(z)$ oziroma $z = \Phi(\xi)$. Dobimo:
    \begin{equation}
        \label{eno_pov_obm}
        [\widetilde{f \circ \Psi} \circ \Phi](\xi) = \widetilde{f}(\xi) = \frac{1}{2\pi i}\int_{\partial \Omega}{f(\tau) \left[\frac{1 - |\Phi(\xi)|^2}{|\Phi(\tau) - \Phi(\xi)|^2}\right]\frac{|\Phi'(\tau)|}{\Phi(\tau)}~d \tau};~~\xi \in \Omega. 
    \end{equation}
    Zapisana funkcija se na $\partial \Omega$ ujema s $f$, na $\overline{\Omega}$ je zvezna, ter na $\Omega$ po trditvi \ref{komp_s_hol} harmonična. Zato je rešitev za Dirichletov problem na $\Omega$ pri začetnem pogoju $f$.

    Koraki izpeljave so nekoliko bolj jasni če sledimo puščicam na spodnji sliki. 
    \begin{figure}[H]
        \begin{center}
            \includegraphics[width = \textwidth]{dirichlet_splosno.png}
            \caption{Izpeljava formule za rešitev Dirichletovega problema na omejenih enostavno povezanih območjih.}
        \end{center}    
    \end{figure}
    Sedaj si zgoraj zapisan predpis oglejmo na nekoliko bolj konkretnem primeru. 
    
    \begin{trditev}
        \label{alldisk}
        Naj bo $h$ zvezna funkcija, definirana na $\partial \mathbb{D}(a,R)$, kjer $a \in \mathbb{C}$ in $R>0$.  
        Za robni pogoj $h$ obstaja rešitev za Dirichletov problem na disku $\mathbb{D}(a,R)$ in je podana s predpisom:
        $$
            H(a + r e^{i \theta}) = \begin{cases}
                    h(a + R e^{i \theta});~~r = R,~\theta \in [0, 2\pi]\\
                    \widetilde{h}(a + r e^{i \theta});~~ r \in [0,R),~ \theta \in [0, 2\pi]
            \end{cases},~\text{kjer:}
        $$
        $$
        \widetilde{h}(a + r e^{i \theta}) = \int_{0}^{2 \pi}{h(a + R e^{i \varphi}) \left[\frac{R^2 - r^2}{R^2 + r^2 - 2Rr \cos(\theta - \varphi)}\right] \frac{d \varphi}{2 \pi}}.
        $$
     \end{trditev}
     \begin{dokaz}
        Pomagali si bomo zgoraj zapisanim postopkom za konstrukcijo rešitve za \mbox{Dirichletov} problem na omejenih enostavnih povezanh območjih. 
        Sedaj definirajmo $\Psi(z) = Rz + a,~z \in \overline{\mathbb{D}}$. Gre za biholomorfizem, katerega inverz je \mbox{$\Phi(\xi) = \frac{1}{R}(\xi - a)$}, $\xi \in \overline{\mathbb{D}}(a,R)$.
        Prek zapisa \eqref{eno_pov_obm} dobimo:
        $$ 
        \widetilde{h}(\xi) = \frac{1}{2\pi i}\int_{\partial \mathbb{D}(a,R)}{h(\tau) \left[\frac{1 - |\frac{1}{R}(\xi - a)|^2}{|\frac{1}{R}(\tau - a) - \frac{1}{R}(\xi - a)|^2}\right]\frac{\frac{1}{R}}{\frac{1}{R}(\tau - a)}~d \tau};~~\xi \in \mathbb{D}(a,R). 
        $$
        Sedaj v integral uvedemo novo spremenljivko $\tau = a + Re^{i \varphi},~\varphi \in [0,2\pi]$, ter označimo $\xi = a + z,~z \in \mathbb{D}(0, R)$. Dobimo:
        \begin{align*}
            \widetilde{h}(a + z) &= \frac{1}{2\pi i}\int_{0}^{2 \pi}{h(a + R e^{i \varphi}) \left[\frac{1 - \frac{1}{R}|z|^2}{|e^{i \varphi} - \frac{1}{R} z|^2}\right]\frac{(i R e^{i \varphi}) d \varphi}{R e^{i \varphi}}}\\
            & = \frac{1}{2 \pi}\int_{0}^{2 \pi}{h(a + R e^{i \varphi}) \left[\frac{1 - \frac{1}{R}|z|^2}{|e^{i \varphi} - \frac{1}{R} z|^2}\right]d\varphi}\\
            & = \frac{1}{2 \pi}\int_{0}^{2 \pi}{h(a + R e^{i \varphi}) \left[\frac{R^2 - |z|^2}{|R e^{i \varphi} - z|^2}\right]d\varphi};~~z \in \mathbb{D}(0,R).
        \end{align*}
        Parametrizirajmo še $z \in \mathbb{D}(0,R)$, kot $z = r e^{i \theta},~r \in [0,R),~\theta \in [0,2 \pi]$. Sledi:
        \begin{align*}
            \widetilde{h}(a + r e^{i \theta}) &= \frac{1}{2 \pi}\int_{0}^{2 \pi}{h(a + R e^{i \varphi}) \left[\frac{|R|^2 - |r e^{i \theta}|^2}{|R e^{i \varphi} - r e^{i \theta}|^2}\right]d\varphi}; \\
            & = \frac{1}{2 \pi}\int_{0}^{2 \pi}{h(a + R e^{i \varphi}) \left[\frac{R^2 - r^2}{R^2 - 2Rr \cos(\theta - \varphi) + r^2}\right]d\varphi};~~r e^{i \theta} \in \mathbb{D}(0,R).
        \end{align*}
        Prišli smo do želenega predpisa na notranjosti diska $D(a,R)$. Da razširitev zares reši Dirichletov problem preverimo enako kot smo to storili pri enotskem disku. 
        Opazimo namreč, da je v zgornjem integralu namesto Poissonovega jedra prisotna funkcija, ki jo iz Poissonovega jedra dobimo če spremenljivko $r$ v predpisu zamenjamo z $\frac{r}{R}$. 
        Omenjena zamenjava vpliva le na spremembo definicijskega območja in lastnosti Poissonovega jedra ne spremeni.
     \end{dokaz}

\subsection{Karakterizacija harmoničnih funkcij}
    Spomnimo se glavne motivacije za razdelek, ki smo jo navedli v opombi \ref{motivacija}.
    Z znanjem, pridobljenim znotraj razdelka, je sedaj dokaz karakterizacije harmoničnih funkcij na dlani.
    
    \begin{trditev}
        \label{ekvhlp}
        Naj bo $h$ zvezna funkcija, definirana na območju $U \subseteq \mathbb{C}$. Velja, da je $h$ harmonična funkcija natanko tedaj, ko ima na $U$ lastnost povprečne vrednosti.
    \end{trditev}
    \begin{dokaz}
        Trditev \ref{harmonicnapovp} nam dokaže eno implikacijo. Dokazati je potrebno torej le še, da je zvezna funkcija $h$ z lastnostjo povprečne vrednosti na $U$ harmonična. 
        Dokaz temelji na obstoju rešitve Dirichletovega problema za poljuben disk. 
        
        Ker je $U$ območje, za poljubno točko $z_0 \in U$ obstaja $r>0$, da je $\overline{\mathbb{D}}(z_0,r) \subseteq U$. Ker je $h$ zvezna na $U$, $h$ na $\partial \overline{\mathbb{D}}(z_0, r)$ določa začetne pogoje za Dirichletov problem za disk $\mathbb{D}(z_0,r)$.
        Trditev \ref{obstoj} in \ref{alldisk} potrdita, da rešitev obstaja. Rešitev označimo s $H(z)$.
        Kot harmonična funkcija na $\mathbb{D}(z_0, r)$, ima po trditvi \ref{harmonicnapovp} $H$ na $\mathbb{D}(z_0, r)$ lastnost povprečne vrednosti. 
        Oglejmo si sedaj funkcijo $g(z) = h(z) - H(z)$ za $z \in \overline{\mathbb{D}}(z_0,r)$. 
        Kot razlika funkcij z lastnostjo povprečne vrednosti ima po trditvi \ref{linlpv} tudi $g$ na $\mathbb{D}(z_0, r)$ lastnost povprečne vrednosti in je kot razlika dveh zveznih funkcij zvezna na $\overline{D}(z_0, r)$.
        Vemo celo, da na $\partial \overline{\mathbb{D}}(z_0, r)$ velja $g \equiv 0$, saj je zožitev $H$ na $\partial \overline{\mathbb{D}}(z_0,r)$ enaka $h$. 
        Po posledici \ref{posledica_pm_lpv} je potem $g \equiv 0$ na $\overline{\mathbb{D}}(z_0, r)$, oziroma $H(z) =  h(z)$, za vsak $z \in \overline{\mathbb{D}}(z_0, r)$. Sledi, da je $h$ harmonična na $\mathbb{D}(z_0, r)$, oziroma ima poljubna točka $z_0$ v $U$ okolico, na kateri je $h$ harmonična. 
        Posledično je funkcija $h$ harmonična na $U$.
    \end{dokaz}
    \begin{posledica}
        Naj bo $u$ realna zvezna funkcija z lastnostjo povprečne vrednosti, definirana na območju $D$. Potem je $u$ na $D$ gladka, oziroma $u \in C^{\infty}(D)$.
    \end{posledica}
    \begin{dokaz}
        Po trditvi \ref{ekvhlp} je $u$ harmonična, po trditvi \ref{gladkosth} pa zato tudi gladka.
    \end{dokaz}

    \begin{opomba}
        \label{karakterizacija}
        Karakterizacija harmoničnih funkcij, prek lastnosti povprečne vrednosti, je primerljiva s karakterizacijo holomorfnih funkcij, ki jo ponuja izrek Morera. 
        Ta nam pove, da je na območju zvezna funkcija holomorfna natanko tedaj, ko je krivuljni integral funkcije po robu vsakega trikotniku, ki je v celoti vsebovan v območju, enak nič.
        Opazimo torej, da se za zvezne funkcije pojem harmoničnosti oziroma holomorfnosti izraža preko zahteve za krivuljni integral po robu (majhnega) diska oziroma robu trikotnika.
        Prednost vsake izmed karakterizacij bo jasna znotraj naslednjega poglavja. Karakterizacija harmoničnih funkcij bo ključna pri dokazu Schwarzovega principa zrcaljenja za harmonične funkcije, 
        karakterizacija holomorfnih funkcij pa bo ključna za dokaz Schwarzovega principa zrcaljenja za holomorfne funkcije. 
    \end{opomba}

\section{Schwarzov princip zrcaljenja}
\subsection{Uvod in definicije osnovnih pojmov}
    Nekoliko se še navežimo na prejšno poglavje. 
    Srečali smo se s problemom obstoja in konstrukcije razširitve, podane funkcije, ki bo obenem zadoščala zahtevam problema.  
    Pri Dirichletovem problemu, smo zahtevali harmoničnost razširitve na notranjosti in zveznost razširitve na zaprtju območja. 
    Pokazali smo, da lahko na enotskem disku zahtevam problema, z začetnim pogojem $h$, zadostimo s pomočjo Poissonovega integrala podane funkcije, tako da za razširitev definiramo:
    $$
        H(z) = 
        \begin{cases}
            h(z)~;~~&z \in \partial \mathbb{D} \\
            \widetilde{h}(z)~;~~&z \in \mathbb{D}
        \end{cases}.
    $$
    Problem, ki ga Schwarzov princip zrcaljenja reši, je ne glede na drugačne začetne pogoje iz vidika iskanja harmonične razširitve kljub temu blizu že obdelanemu Dirichletovemu problemu.
    Pri Dirichletovem problemu, smo podano funkcijo želeli harmonično razširiti na notranjost območja, pri čemer smo poznali le vrednosti funkcije na robu, tu pa bomo harmoničnost funkcije na manjšem območju že privzeli, ter iskali harmonično razširitev na novo območju podobne oblike.
    Kaj točno smatramo za območje podobne oblike bo jasno v nadaljevanju, vendar kot to že nakazuje beseda zrcaljanje v imenu principa, bomo razširitev želeli konstruirati na zrcalni sliki prvotnega območja. 
    
    Spoznajmo najprej nekaj osnovnih pojmov, ki nam bodo omogočili natančno formulacijo in dokaz glavnega izreka razdelka.
    \begin{definicija}
        Naj bo $U \subseteq \mathbb{C}$ območje. Potem je $U^* = \{\overline{z}~|~z \in U\}$ \emph{zrcalna slika območja U glede na realno os}.
        Če za območje $U$ velja $U^* = U$ pravimo, da je območje $U$ \emph{simetrično glede na realno os}.
    \end{definicija}

    \begin{definicija}
        Naj bo $u$ funkcija, definirana na območju $U \subseteq \mathbb{C}$. Potem z \emph{$u^*$} označimo funkcijo, definirano na $U^*$, s predpisom $u^*(w) = \overline{u(\overline{w})},~w \in U^*$
    \end{definicija}

    Opazimo, da predpis iz zgornje definicije, za funkcijo $u$ definirano na območju $U$, v resnici ponuja zvezo $u^*(\overline{z}) = (u^* \circ (z \mapsto \overline{z}))(z) = \overline{u(z)},~z \in U$.
    Zapisana zveza nakazuje, da se bodo v veliki meri lastnosti funkcije $u$ prenesle na funkcijo $u^*$.

    \begin{opomba}
        Na tem mestu le spomnimo, da je za realne funkcije $u$, definirane na $U$, konjugiranje funkcijskih vrednosti pri predpisu funkcije $u^*$ odveč. Torej je funkcija $u^*$ določena kar prek zveze $u^*(w) = u(\overline{w}),~w \in U^*$.
    \end{opomba}

\subsection{Schwarzov princip zrcaljenja za harmonične funkcije}
    \begin{trditev}
        \label{lemaharm}
        Naj bo $u$ realna funkcija, definirana na območju $U \subseteq \mathbb{C}$. Če je $u$ harmonična na $U$, potem je $u^*$ harmonična na $U^*$. 
    \end{trditev}
    \begin{proof}
        Trditev dokažimo na dva načina. Dokaz prvega načina temelji na uporabi karakterizacije harmoničnih funkcij z lastnostjo povprečne vrednosti. 
        
        Naj bo \mbox{$w_0 \in U^*$}. Ker je $U^*$ območje, obstaja $\rho>0$, da za vsak $0 < r < \rho$ velja $\overline{\mathbb{D}}(w_0, r) \subseteq U^*$. Po definiciji $U^*$ sledi, da za vsak \mbox{$0 < r < \rho$} velja $\overline{\mathbb{D}}(\overline{w_0}, r) \subseteq U$. 
        Ker je funkcija $u$ na $U$ harmonična, ima po trditvi \ref{ekvhlp} na $U$ lastnost povprečne vrednosti. Sledi:
        \begin{align*}
            u^*(w_0) & = u(\overline{w_0}) = \frac{1}{2 \pi} \int_{0}^{2 \pi}{u\left(\overline{w_0} + r e^{i \theta}\right) d\theta} = \frac{1}{2 \pi} \int_{0}^{2 \pi}{u\left(\overline{w_0 + r e^{-i \theta}}\right) d\theta}\\
            &= \frac{1}{2 \pi} \int_{0}^{2 \pi}{u^*\left(w_0 + r e^{-i \theta}\right) d\theta} = \frac{1}{2 \pi} \int_{0}^{2 \pi}{u^*\left(w_0 + r e^{i \theta}\right) d\theta};~r \in (0, \rho).
        \end{align*} 
        Pokazali smo, da ima funkcija $u^*$ na $U^*$ lastnost povprečne vrednosti. 
        Ker je funkcija $u$ zvezna na $U$, je očitno funkcija $u^*$ zvezna na $U^*$. 
        Po trditvi \ref{ekvhlp} sledi, da je funkcija $u^*$ na $U^*$ harmonična. 
        
        Dokažimo trditev še po definiciji, t.j. dokažimo, da $u^*$ na $U^*$ zadošča Laplaceovi parcialni diferencialni enačbi.
        Pomagali si bomo z standardno identifikacijo kompleksne spremenljivke $z = x + iy$ s parom realnega in imaginarnega dela $(x,y)$.
        Identifikacija nam narekuje zvezo $[u^* \circ ((x,y) \mapsto (x,-y))](x, y) = u(x, y),~ x + iy \in U$.
        Sedaj parcialno odvajajmo in dobimo:
        \begin{align*}
            \frac{\partial u}{\partial x} &= \frac{\partial u^* }{\partial x}~,~~~~~\frac{\partial u}{\partial y} = - \frac{\partial u^* }{\partial y}, \\
            \frac{\partial^2 u}{\partial x^2} & = \frac{\partial^2 u^* }{\partial x^2}~,~~~~~\frac{\partial^2 u}{\partial y^2} = \frac{\partial^2 u^* }{\partial y^2}.
        \end{align*}
        Ker je funkcija $u$ na $U$ harmonična, velja:
        $$
            \Delta u^* = \frac{\partial^2 u^*}{\partial x^2} + \frac{\partial^2 u^*}{\partial y^2} = \frac{\partial^2 u}{\partial x^2} + \frac{\partial^2 u}{\partial y^2} = 0.
        $$
        Sledi, da je $u^*$ harmonična na $U^*$.
    \end{proof}

    Sedaj smo pripravljeni na formulacijo in dokaz glavnega izreka.
    
    \begin{izrek}[Schwarzov princip zrcaljenja za harmonične funkcije]
        \label{schwarz_harm}
        Naj bo $D \subseteq \mathbb{C}$ območje, simetrično glede na realno os. 
        Označimo $D^{+} = \{z \in D~|~\text{Im}[z] > 0\},~D^{-} = \{z \in D~|~\text{Im}[z] < 0\}$ in $D^{0} = \{z \in D~|~\text{Im}[z] = 0\}$.
        Naj bo $u$ realna zvezna funkcija, definirana na $D^{+} \cup D^0$. Naj bo $u$ na $D^{+}$ harmonična in naj za vsak $z \in D^0$ velja $u(z) = 0$.
        Potem obstaja harmonična razširitev funkcije $u$ na $D$, ki je eksplicitno podana prek zveze $u(\bar{z}) = - u(z),~z \in D$.
    \end{izrek}
    \begin{opomba}
        \label{op_spz}
        Preden se lotimo dokaza izreka, se še nekoliko posvetimo temu kaj nam izrek v resnici pove. 
        Zapisana zveza za funkcijo $u$, nam v resnici narekuje predpis za funkcijo:
        $$
        U(z) = 
        \begin{cases}
            u(z)~;~~&z \in D^{+}\\
            -u(\overline{z})~;~~&z \in D^{-}\\
            0~;~~ &z \in D^0
        \end{cases}
        .
        $$
        Izrek pravi, da je funkcija $U$ na $D$ harmonična.
        Sedaj zapisano situacijo komentirajmo še s pomočjo spodnje slike. 
        \begin{figure}[H]
            \begin{center}
                \includegraphics[width = \textwidth]{schwarz_harm.png}
                \caption{Schwarzov princip zrcaljenja za harmonične funkcije.}
            \end{center}    
        \end{figure}

    Slika prikazuje glede na realno os simetrično območje $D \subseteq \mathbb{C}$. $D^{+}$ označuje območje, na katerem je poznana funkcija $u$ harmonična, $D^{-}$ pa zrcalno sliko območja $D^+$ glede na realno os. 
    Predpis razširitve je na $D^-$ podan s predpisom $u(\overline{z}) = - u(z)$.
    Vodoravna črta prikazuje odsek realne osi. Odsek, ki leži znotraj območja $D$ smo označili z $D^0$. Funkcija $u$ je na tem delu območja po predpostavki izreka ničelna.
    Točka $z_0$ in majhen disk okoli nje nakazujeta, da se bomo dokaza lotili prek karakterizacije harmoničnih funkcij z lastnostjo povprečne vrednosti.
    \end{opomba}

    \begin{dokaz}
        Opazimo, da je $D$ disjunktna unija $D^+$, $D^-$ in $D^0$.
        Sedaj na območju $D$ definirajmo funkcijo $u$, kot smo to storili v opombi \ref{op_spz}.
        Funkcija $U$ je po predpostavkah izreka zvezna na $D^+ \cup D^0$, iz definicije, prek predpisa $u(\overline{z}) = - u(z)$, pa je jasno, da je zvezna kar na $D$. 
        Posvetimo se dokazu harmoničnosti.
        Ker je $U$, oziroma $u$, na $D^+$ harmonična, ima po trditvi \ref{ekvhlp}, na $D^+$ lastnost povprečne vrednosti. 
        %Sledi, da za vsak $z_0 \in D^+$ obstaja $\epsilon_{z_0} > 0$, je $\overline{\mathbb{D}}(z_0, \epsilon_{z_0}) \subseteq D^+$ in za vsak $0 < \epsilon \leq \epsilon_{z_0}$ velja:
        %$$
        %U(z_0) = u(z_0) = \frac{1}{2 \pi} \int_{0}^{2 \pi}{u(z_0 + \epsilon e^{i \theta}) d\theta} = \frac{1}{2 \pi} \int_{0}^{2 \pi}{U(z_0 + \epsilon e^{i \theta}) d\theta}.
        %$$
        %Naj bo $w_0 \in D^-$. Iščemo $\epsilon_{w_0} > 0$, da bo $\overline{\mathbb{D}}(w_0, \epsilon_{w_0}) \subseteq D^-$ in bo za vsak $0 < \epsilon \leq \epsilon_{w_0}$ veljalo:
        %$$
        %U(w_0) = \frac{1}{2 \pi} \int_{0}^{2 \pi}{U(w_0 + \epsilon e^{i \theta}) d\theta}.
        %$$
        %Ker je $\overline{w_0} \in D^+$, lahko vzamemo $\epsilon_{w_0} = \epsilon_{\overline{w_0}}$. Za vsak $ 0 < \epsilon \leq \epsilon_{\overline{w_0}}$ namreč:
        %\begin{align*}
        %    U(w_0) & = -u(\overline{w_0}) = -\frac{1}{2 \pi} \int_{0}^{2 \pi}{u(\overline{w_0} + \epsilon e^{i \theta}) d\theta} = \frac{1}{2 \pi} \int_{0}^{2 \pi}{- u\left(\overline{w_0 + \epsilon e^{-i \theta}}\right) d\theta} \\
        %    & = \frac{1}{2 \pi} \int_{0}^{2 \pi}{U\left(w_0 + \epsilon e^{-i \theta}\right) d\theta} = \frac{1}{2 \pi} \int_{0}^{2 \pi}{U\left(w_0 + \epsilon e^{i \theta}\right) d\theta}.
        %\end{align*}
        %Torej $U$ zadošča lastnosti povprečne vrednosti tudi na $D^-$. Preostane nam $D^0$.
        Opazimo, da funkcijski predpis funkcije $U$ na območju $D^-$ sovpada s predpisom funkcije $- u^*(z)$. 
        Ker je $U$, oziroma $u$, na $D^+$ harmonična, je po trditvi \ref{lemaharm} $-u^*(z)$, oziroma $U$ harmonična na $(D^+)^* = D^-$.
        Po trditvi \ref{ekvhlp} ima $U$ na $D^-$ lastnost povprečne vrednosti. 
        Sedaj nam preostane le še $D^0$.
        Vzemimo poljubno točko $z_0 \in D^0 \subseteq \mathbb{R}$. Ker je $D$ območje, obstaja $\rho > 0$, da je $\overline{\mathbb{D}}(z_0, \rho) \subseteq D$. Za vsak $0 < r < \rho$ velja:
        \begin{align*}
            \frac{1}{2 \pi} \int_{0}^{2 \pi}{U(z_0 + re^{i \theta}) d\theta} &= \frac{1}{2 \pi} \left[\int_{0}^{\pi}U(z_0 + re^{i \theta})~d\theta + \int_{\pi}^{2\pi} U(z_0 + re^{i \theta})~d\theta\right] \\
            & = \frac{1}{2 \pi} \left[\int_{0}^{\pi}U(z_0 + re^{i \theta})~d\theta + \int_{0}^{\pi}U(\overline{z_0 + re^{i \theta}})~d\theta\right] \\
            &= \frac{1}{2 \pi} \left[\int_{0}^{\pi}u(z_0 + re^{i \theta})~d\theta - \int_{0}^{\pi}u(z_0 + re^{i \theta})~d\theta\right] = 0.
        \end{align*}        
        Ker je funkcija $U$ na $D^0$ ničelna, smo pokazali, da ima $U$ tudi na $D^0$ lastnost povprečne vrednosti. 
        Dokazali smo, da ima zvezna funkcija $U$ na $D = D^+ \cup D^- \cup D^0$ lastnost povprečne vrednosti, zato je po trditvi \ref{ekvhlp} na $D$ harmonična.
    \end{dokaz}

\subsection{Schwarzov princip zrcaljenja za holomorfne funkcije}
    \begin{trditev}
        \label{lemahol}
        Naj bo $f$ holomorfna funkcija, definirana na območju $U \subseteq \mathbb{C}$.
        Potem je funkcija $f^*$ holomorfna na $U^*$.
    \end{trditev}
    \begin{dokaz}
        Trditev ponovno dokažimo na dva načina. Dokaza se najprej lotimo s pomočjo Cauchy-Riemannovega sistema, nato pa holomorfnost funkcije $f^*$ dokažimo še po definiciji.

        Označimo $f(z) = u(z) + iv(z),~ z \in U$ in $f^*(w) = p(w) + iq(w),~w \in U^*$, kjer so $u,v, p$ in $q$ realne funkcije. 
        Iz definicije funkcije $f^*$ sledi $f^*(w) = \overline{u(\overline{w}) + iv(\overline{w})} = u(\overline{w}) - iv(\overline{w}),~ w \in U^*$. 
        Sedaj enačimo realni in imaginarni del zapisanih funkcij. Dobimo:
        \begin{align*}
            p(w) &= u(\overline{w}),~w \in U^*~,~~\text{oziroma}~~~~p(\overline{z}) = u(z),~z \in U~,~~\text{ter:} \\
            q(w) &= -v(\overline{w}),~w \in U^*~,~~\text{oziroma}~~~~q(\overline{z}) = -v(z),~z \in U.
        \end{align*}
        Enakosti lahko sedaj ob upoštevanju identifikacije $z \leftrightarrow (x,y)$ oziroma $\overline{z} \leftrightarrow (x, -y)$ parcialno odvajamo po realnih spremenljivkah $x$ in $y$. Dobimo:
        \begin{align*}
            \frac{\partial u}{\partial x} &= \frac{\partial p}{\partial x}~~~~\text{in}~~~~ -\frac{\partial v}{\partial x} = \frac{\partial q}{\partial x}~,~~\text{ter:}\\
            \frac{\partial u}{\partial y} &=  - \frac{\partial p}{\partial y}~~~~\text{in}~~~~ -\frac{\partial v}{\partial y} = - \frac{\partial q}{\partial y} \iff \frac{\partial v}{\partial y} = \frac{\partial q}{\partial y}~.      
        \end{align*}
        Ker je funkcija $f$ holomorfna, zadošča Cauchy-Riemannovemu sistemu enačb.  
        Sledi: 
        \begin{align*}
            \frac{\partial p}{\partial x} = \frac{\partial u}{\partial x} &= \frac{\partial v}{\partial y} = \frac{\partial q}{\partial y}~~~~\text{in}~~~~ \frac{\partial p}{\partial y} = -\frac{\partial u}{\partial y} = \frac{\partial v}{\partial x} = -\frac{\partial q}{\partial x}.
        \end{align*}
        Opazimo, da funkcija $f^* = p + iq$ zadošča Cauchy-Riemannovemu sistemu enačb, zato je holomorfna.

        Sedaj dokažimo holomorfnost funkcije $f^*$ še po definiciji. Želimo dokazati, da je $f^*$ v vsaki točki $w \in U^*$ kompleksno odvedljiva. Predpostavka trditve nam pove, da je $f$ kompleksno odvedljiva v vsaki točki $\overline{w} \in U$.
        Naj bo $w \in U^*$. Potem velja: 
        \begin{align*}
            \lim_{h \to 0}\left[{\frac{f^*(w + h) - f^*(w)}{h}}\right] &= \lim_{h \to 0}\left[{\frac{\overline{f\left(\overline{w + h}\right)} - \overline{f(\overline{w})}}{h}}\right] = \lim_{h \to 0}{\overline{\left[\frac{f\left(\overline{w + h}\right) - f(\overline{w})}{\overline{h}}\right]}} \\
            \lim_{h \to 0}{\overline{\left[\frac{f\left(\overline{w} + \overline{h}\right) - f(\overline{w})}{\overline{h}}\right]}} & = \overline{\lim_{\overline{h} \to 0}{\left[\frac{f\left(\overline{w} + \overline{h}\right) - f(\overline{w})}{\overline{h}}\right]}} = \overline{f'(\overline{w})}.
        \end{align*}
        Limita diferenčnega kvocienta obstaja v poljubni točki $w \in U^*$, zato je $f^*$ na $U^*$ kompleksno odvedljiva oziroma holomorfna.
    \end{dokaz}

    \begin{trditev}[Schwarzov princip zrcaljenja za holomorfne funkcije]
        Naj bo $D \subseteq \mathbb{C}$ območje, simetrično glede na realno os. 
        Označimo $D^{+} = \{z \in D~|~\text{Im}[z] > 0\},~D^{-} = \{z \in D~|~\text{Im}[z] < 0\}$ in $D^{0} = \{z \in D~|~\text{Im}[z] = 0\}$.
        Naj bo $f$ zvezna funkcija, definirana na $D^{+} \cup D^0$. Naj bo $f$ na $D^{+}$ holomorfna in naj $f$ na $D^0$ zavzame realne vrednosti.
        Potem je funkcija $F$, definirana s predpisom:
        $$
        F(z) = 
        \begin{cases}
            f(z)~;~~&z \in D^{+} \cup D^0\\
            f^*(z)~;~~&z \in D^{-} \cup D^0
        \end{cases}~,
        $$
        na območju $D$ holomorfna.
    \end{trditev}
    \begin{opomba}
        Trditev nam v resnici pove, da za holomorfno funkcijo $f$ obstaja holomorfna razširitev, podana s predpisom:
        $$
        F(z) = 
        \begin{cases}
            f(z)~;~~&z \in D^{+} \cup D^0\\
            \overline{f(\overline{z})}~;~~&z \in D^{-} \cup D^0
        \end{cases}.
        $$
        Opazimo, da imata veji predpisa funkcije $F$ neprazen presek, zato se je smiselno vprašati ali je predpis sploh dobro definiran. V resnici je potrebno preveriti ali se funkcijske vrednosti na preseku vej, oziroma $D^0$, ujemajo.  
        Naj bo $z_0 \in D^0$. Po definiciji sledi $z_0 \in \mathbb{R}$, ter zato $\overline{z_0} = z_0$. Ker funkcija $f$, po predpostavki trditve, na $D^0$ zavzame realne vrednosti, sledi $f(z_0) = f(\overline{z_0}) = \overline{f(\overline{z_0})}$.
    \end{opomba}
    \begin{dokaz}
        Območje $D$ je disjunktna unija $D^+$, $D^-$ in $D^0$. Vemo da je funkcija $F$, oziroma $f$, holomorfna na $D^+$, zato je po trditvi \ref{lemahol} funkcija $f^*$, oziroma $F$, holomorfna na $(D^+)^* = D^-$. 
        Dovolj je torej dokazati, da je $F$ holomorfna v okolici vsake točke iz $D^0$. 
        Naj bo $z_0 \in D^0$ in $\delta > 0$ tak, da velja $\overline{\mathbb{D}}(z_0,\delta) \subseteq \mathbb{D}$. Spomnimo se opombe \ref{karakterizacija}. 
        Funkcija $F$ je očitno zvezna kar na $D$, zato je dovolj pokazati, da za vsak zaprt trikotnik $T$, za katerega velja $T \subseteq \mathbb{D}(z_0, \delta)$ sledi: 
        $$ 
            \int_{\partial T}{F(\xi)~d\xi} = 0.
        $$
        Naj bo $T$ poljuben zaprt trikotnik, ki je v celoti vsebovan v $\mathbb{D}(z_0,\delta)$. Opazimo, da imamo dve možnosti.
        Bodisi je $T$ v celoti vsebovan v $\mathbb{D}(z_0,\delta) \cap D^+$ ali $\mathbb{D}(z_0,\delta) \cap D^-$, bodisi ima z $D^0$ neprazen presek.
        
        Če je $T$ v celoti vsebovan v $\mathbb{D}(z_0,\delta) \cap D^+$ ali $\mathbb{D}(z_0,\delta) \cap D^-$, nam holomorfnost funkcije $F$ na $D^+$ in $D^-$ omogoča uporabo Cauchyjevega izreka. 
        Ta nam pove, da je krivuljni integral po robu $T$ res enak nič. 
        Zapisana situacija je za zaprte trikotnike $T_1 \subseteq \mathbb{D}(z_0,\delta) \cap D^+$, ter zaprte trikotnike $T_2 \subseteq \mathbb{D}(z_0,\delta) \cap D^-$, grafično prikazana na spodnji sliki. 
        \begin{figure}[H]
            \begin{center}
                \includegraphics[width = \textwidth]{schwarz_hol_1.png}
                \caption{Zaprta trikotnika $T_1$ in $T_2$, vsebovana v $\mathbb{D}(z_0,\delta) \cap D^+$ oziroma v $\mathbb{D}(z_0,\delta) \cap D^-$.}
            \end{center}    
        \end{figure}

        Posvetimo se drugi možnosti, torej možnosti, da presek zaprtega trikotnika $T$ in $D^0$ ni prazen. Tokrat si situacijo najprej oglejmo grafično. 

        \begin{figure}[H]
            \begin{center}
                \includegraphics[width = \textwidth]{schwarz_hol_2.png}
                \caption{Delitev zaprtega trikotnika $T$ na $P^+$, $P^-$ in $P^0$.}
            \end{center}    
        \end{figure}

        Ideja je krivuljni integral po robu trikotnika dovolj blizu $D^0$ razbiti na krivuljna integrala, ki bosta v celoti znotraj območij $\mathbb{D}(z_0,\delta) \cap D^+$ oziroma $\mathbb{D}(z_0,\delta) \cap D^-$, 
        ter krivuljni integral, katerega prispevek bo dovolj majhen.
        
        Naj bo $\epsilon > 0$. Želimo najti dovolj majhen $\delta_0 > 0$, za katerega bo delitev zaprtega trikotnika $T$ na $P^+ = \{z \in T~|~\text{Im}[z] \geq \delta_0\}$, $P^- = \{z \in T~|~\text{Im}[z] \leq -\delta_0\}$ in $P^0 = \{z \in T~|~|\text{Im}[z]| \leq \delta_0\}$ zadoščala:
        $$ 
            \left|\int_{\partial P^0}{F(\xi)~d\xi}\right| < \epsilon~,~\text{ter zato po Cauchyjevem izreku za funkcijo $F$ na $P^+$ in $P^-$:}
        $$
        \begin{align*}
            \left|\int_{\partial T}{F(\xi)~d\xi}\right| &= \left|\int_{\partial P^+}{F(\xi)~d\xi} + \int_{\partial P^-}{F(\xi)~d\xi} + \int_{\partial P^0}{F(\xi)~d\xi}\right|\\
            &\leq \left|\int_{\partial P^+}{F(\xi)~d\xi} \right|+ \left|\int_{\partial P^-}{F(\xi)~d\xi}\right| + \left|\int_{\partial P^0}{F(\xi)~d\xi}\right| \\ 
            & =\left|\int_{\partial P^0}{F(\xi)~d\xi} \right| < \epsilon.
        \end{align*}
        Sedaj razbijmo še krivuljni integral po robu $\partial P^0$. Ponovno si delitev najprej oglejmo na spodnji sliki, ter jo šele nato uporabimo. 
        \begin{figure}[H]
            \begin{center}
                \includegraphics[width = \textwidth]{schwarz_hol_int.png}
                \caption{Delitev roba $P^0$ na $S_Z$, $S_D$, $S_L$ in $S_S$.}
            \end{center}    
        \end{figure}
        Slika nam nakazuje delitev krivuljnega integrala po robu na linearne odseke. Velja:
        \begin{align*}
            \left|\int_{\partial P^0}{F(\xi)~d\xi}\right| &= \left|\int_{S_Z}{F(\xi)~d\xi} - \int_{S_S}{F(\xi)~d\xi} + \int_{S_L}{F(\xi)~d\xi} - \int_{S_D}{F(\xi)~d\xi}\right|\\
            & \leq \left|\int_{S_Z}{F(\xi)~d\xi} - \int_{S_S}{F(\xi)~d\xi} \right| + \left|\int_{S_L}{F(\xi)~d\xi}\right| + \left|\int_{S_D}{F(\xi)~d\xi}\right|.\\
        \end{align*}
        Opazimo, da bi lahko krivuljna integrala $S_D$ in $S_L$ parametrizirali s pomočjo linearne funkcije. Zato obstaja konstanta $C_1 \in \mathbb{R}$, da velja $|z_{ZL} - z_{SL}| < C_1 \delta_0$ in  \mbox{$|z_{ZD} - z_{SD}| < C_1 \delta_0$}.
        Ker je $F$ na $D$ zvezna, $P^0$ pa zaprta podmožica, je $|F|$ na $P^0$ omejena. Sedaj opažanji združimo. Ko krivuljni integral parametriziramo, ter integrand navzgor omejimo, opazimo, da je absolutna vrednost integrala lahko poljubno majhna, če bo le $\delta_0$ dovolj majhen. 
        Zapišimo nekoliko drugače. Obstaja $\delta_1 > 0$, da za $\delta_0 \leq \delta_1$ velja:  
        $$ 
        \left|\int_{S_L}{F(\xi)~d\xi}\right| < \frac{\epsilon}{5}~~~\text{in}~~~\left|\int_{S_D}{F(\xi)~d\xi}\right| < \frac{\epsilon}{5}.
        $$
        Podoben sklep lahko naredimo pri integralu $\int_{S_1}{F(\xi)~d\xi}$ in $\int_{S_2}{F(\xi)~d\xi}$. Obstaja torej $\delta_2 > 0$, da za $\delta_0 \leq \delta_2$ velja:
        $$ 
        \left|\int_{S_1}{F(\xi)~d\xi}\right| < \frac{\epsilon}{5}~~~\text{in}~~~\left|\int_{S_2}{F(\xi)~d\xi}\right| < \frac{\epsilon}{5}.
        $$
        Sedaj uporabimo zgornje ocene. Dobimo: 
        \begin{align*}
            \left|\int_{\partial P^0}{F(\xi)~d\xi}\right|&< \left|\int_{S_Z}{F(\xi)~d\xi} - \int_{S_{S'}}{F(\xi)~d\xi} - \int_{S_1}{F(\xi)~d\xi} - \int_{S_2}{F(\xi)~d\xi}\right| + \frac{\epsilon}{5} +  \frac{\epsilon}{5}\\
            & \leq \left|\int_{S_Z}{F(\xi)~d\xi} - \int_{S_{S'}}{F(\xi)~d\xi}\right| + \left|\int_{S_1}{F(\xi)~d\xi}\right| + \left|\int_{S_2}{F(\xi)~d\xi}\right| + \frac{2\epsilon}{5}\\
            & < \left|\int_{S_Z}{F(\xi_Z)~d\xi_Z} - \int_{S_{S'}}{F(\xi_{S'})~d\xi_{S'}}\right| + \frac{4\epsilon}{5}.\\
        \end{align*}
        Sedaj preostala integrala parametrizirajmo z $\xi_Z = (1 -t)z_{ZD} + t z_{ZL},~t \in [0,1]$ in $\xi_{S'} = (1 -t)z_{Z2} + t z_{Z1},~t \in [0,1]$.
        Sledi: 
        \begin{align*}
            &\int_{S_Z}{F(\xi_Z)~d\xi_Z} - \int_{S_{S'}}{F(\xi_{S'})~d\xi_{S'}}\\
            &=\int_{0}^{1}{F((1 -t)z_{ZD} + t z_{ZL})(z_{ZL} - z_{ZD})~dt} - \int_{0}^{1}{F((1 -t)z_{2} + t z_{1})(z_{1} - z_{2})~dt}\\
            &=\int_{0}^{1}{\left[F((1 -t)z_{ZD} + t z_{ZL})(z_{ZL} - z_{ZD}) - F((1 -t)z_{2} + t z_{1})(z_{1} - z_{2})\right]dt}.
        \end{align*}  
        Opazimo, da velja $z_{ZL} = z_{1} + 2 \delta_0$ in $z_{ZD} = z_2 + 2 \delta_0$, ter zato:
        \begin{align*}
            &\int_{S_Z}{F(\xi_Z)~d\xi_Z} - \int_{S_{S'}}{F(\xi_{S'})~d\xi_{S'}}\\
            &=(z_{1} - z_{2})\int_{0}^{1}{\left[F((1 -t)z_{2} + t z_{1} + 2 \delta_0) - F((1 -t)z_{2} + t z_{1})\right]dt}.
        \end{align*}  
        Ker je $F$ zvezna na kompaktu $P^0$ obstaja $\delta_3 >0$, da za vsak $z, w \in P^0$ iz $|z -w| < \delta_3$ sledi $|F(z) - F(w)| < \frac{\epsilon}{5(z_1 - z_2)}$.
        Zato za $\delta_0 \leq \frac{\delta_3}{2}$ velja:
        \begin{align*}
            &\left|\int_{S_Z}{F(\xi_Z)~d\xi_Z} - \int_{S_{S'}}{F(\xi_{S'})~d\xi_{S'}}\right|
            \leq (z_{1} - z_{2})\int_{0}^{1}{\frac{\epsilon}{5(z_{1} - z_{2})}~dt} < \frac{\epsilon}{5}.
        \end{align*}  
        Sedaj ocene združimo. Vzemimo $\delta_0 = \text{min}\{\delta_1, \delta_2, \frac{\delta_3}{2}\}$. Velja:
        $$ 
        \left|\int_{\partial P^0}{F(\xi)~d\xi}\right|< \epsilon.
        $$
        Ker je bil $\epsilon$ poljubno majhen, iz tega že sledi $\int_{\partial T}{F(\xi)~d\xi} = 0$. 
        Da bo enakost veljala za vsak zaprt trikotnik iz okolice točke $z_0$, lahko namesto $\delta$ pri definiciji okolice vzamemo $\widetilde{\delta} = \delta_0 < \delta$. 
        Sledi, da je $F$ holomorfna na $\mathbb{D}(z_0, \widetilde{\delta})$. Ker je bil $z_0 \in D^0$ poljuben, smo dokazali, da je $F$ holomorfna tudi v okolici vsake točke iz $D^0$. 
    \end{dokaz}

\section{Zaključek}
    TU BO ZAKLJUCEK

\bibliographystyle{siam}
\begin{thebibliography}{9}
    \bibitem{osnova}
    Theodore W. Gamelin \emph{Complex Analysis}, Springer (2001), Chapter X, str. 274 - 288

    \bibitem{mean value p}
    Weisstein, Eric W. \emph{Mean-Value Property}, v: From MathWorld--A Wolfram Web Resource, [ogled 22. 2. 2023], dostopno na
\end{thebibliography}

\end{document}
