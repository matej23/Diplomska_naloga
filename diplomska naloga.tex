\documentclass[mat1]{fmfdelo}
\usepackage{graphicx}
\usepackage{amsmath}
\usepackage[shortlabels]{enumitem}
% \documentclass[fin1]{fmfdelo}
% \documentclass[isrm1]{fmfdelo}
% \documentclass[mat2]{fmfdelo}
% \documentclass[fin2]{fmfdelo}
% \documentclass[isrm2]{fmfdelo}

% naslednje ukaze ustrezno napolnite
\avtor{Matej Novoselec}

\naslov{Schwarzov princip zrcaljenja za harmonične funkcije}
\title{Schwarz Reflection Principle for Harmonic Functions}

% navedite ime mentorja s polnim nazivom: doc.~dr.~Ime Priimek,
% izr.~prof.~dr.~Ime Priimek, prof.~dr.~Ime Priimek
% uporabite le tisti ukaz/ukaze, ki je/so za vas ustrezni
\mentor{prof. dr. Barbara Drinovec Drnovšek}
% \mentorica{}
% \somentor{}
% \somentorica{}
% \mentorja{}{}
% \mentorici{}{}

\letnica{2023} % leto diplome

%  V povzetku na kratko opišite vsebinske rezultate dela. Sem ne sodi razlaga organizacije dela --
%  v katerem poglavju/razdelku je kaj, pač pa le opis vsebine.
\povzetek{}

%  Prevod slovenskega povzetka v angleščino.
\abstract{}

% navedite vsaj eno klasifikacijsko oznako --
% dostopne so na www.ams.org/mathscinet/msc/msc2020.html
\klasifikacija{}
\kljucnebesede{} % navedite nekaj ključnih pojmov, ki nastopajo v delu
\keywords{} % angleški prevod ključnih besed

\zapisiMetaPodatke  % poskrbi za metapodatke in veljaven PDF/A-1b standard

% aktivirajte pakete, ki jih potrebujete
% \usepackage{tikz}

% za številske množice uporabite naslednje simbole
\newcommand{\R}{\mathbb R}
\newcommand{\N}{\mathbb N}
\newcommand{\Z}{\mathbb Z}
\newcommand{\C}{\mathbb C}
\newcommand{\Q}{\mathbb Q}

% matematične operatorje deklarirajte kot take, da jih bo Latex pravilno stavil
% \DeclareMathOperator{\conv}{conv}

% vstavite svoje definicije ...
%  \newcommand{}{}

\begin{document}

\section{Uvod}
V diplomski nalogi bomo spoznali osnovne lastnosti harmoničnih fukcij, ki jih bomo proti koncu s pridom uporabili za dokaz glavnega izreka, katerega ime nosi naslov naloge.
Vlekli bomo številne vzporednice s kompleksno analizo, gre namreč za področje, močno povezano s študijo holomorfnih funkcij.

V prvem poglavju bomo spoznali, kaj so harmonične funkcije in poudarili, katere njihove lastnosti bodo za nadaljevanje pomembne. Ogledali si bomo tudi njihov odnos s holomorfnimi funkcijami. 
V drugem poglavju bomo spoznali Dirichletov problem za enotski disk, ki nam bo dal osnovo za definicijo Poissonovega jedra in Poissonovega integrala. Ogledali si bomo nekaj lastnosti obeh definiranih pojmov in z njuno pomočjo rešili Dirichletov problem za enotski disk.
Tretje poglavje je namenjeno karakterizaciji harmoničnih funkcij s pomočjo lastnosti povprečne vrednosti in analizi pomembnosti te karakterizacije. 
V zadnjem poglavju bomo s pomočjo orodij, spoznanih v prejšnih poglavjih, navedli in dokazali glavni izrek diplomskega dela - Schwarzov princip zrcaljenja za harmonične funkcije.
%\section*{Slovar strokovnih izrazov}
%
%\geslo{}{}
%\geslo{}{}

%------------------------------------------------------------
\newpage
\section{Harmonične funkcije}
    \begin{definicija}
        \label{harm}
        Naj bo $U$ odprta podmnožica v $\mathbb{R}^n$. Naj bo $u$ funkcija, definirana na $U$ in naj bo na definicijskem območju dvakrat zvezno odvedljiva.  
        Pravimo, da je funkcija $u(x_1, x_2, \dots, x_n)$ \emph{harmonična}, če velja
        $$
        \frac{\partial^2 u}{\partial x_1 ^ 2} +  \frac{\partial^2 u}{\partial x_2 ^ 2} + \dots + \frac{\partial^2 u}{\partial x_n ^ 2} = 0.
        $$
        Operatorju $\Delta  = \frac{\partial^2}{\partial x_1 ^ 2} +  \frac{\partial^2}{\partial x_2 ^ 2} + \dots + \frac{\partial^2}{\partial x_n ^ 2}$ pravimo Laplaceov operator in pišemo
        $$
        \Delta u = 0.
        $$
    \end{definicija}

    Pogoj za harmoničnost podaja Laplaceovo parcialno diferencialno enačbo, zapisano bodisi s Laplaceovim operatorjem ali razpisano s parcialnimi odvodi drugega reda. 
    Funkcija je torej harmonična, če zadošča zgoraj zapisani parcialni diferencialni enačbi. 
    %Po tihem tu seveda privzemamo obstoj (vsaj) drugih parcialnih odvodov, saj drugače o harmoničnosti funkcije ne moremo govoriti.

    \begin{opomba}
        Vredno je omeniti, da v zgornji definiciji nismo specificirali, ali gre pri funkciji $u$ za realno ali kompleksno funkcijo. 
        Pojem harmoničnosti smo definirali v splošnem, torej tako za kompleksne kot tudi realne funkcije.
        Znotraj diplomske naloge se bomo omejili na funkcije dveh realnih spremenljivk ali funkcijo ene kompleksne spremenljivke, ki jo bomo nato delili na realni in imaginarni del ($z = x + iy $) in na ta način prešli nazaj na funkcije dveh realnih spremenljivk.
        \newline
        Pogoj za harmoničnost takrat zapišemo kot: 
            $$
                \Delta u = \frac{\partial^2 u}{\partial x ^ 2} +  \frac{\partial^2 u}{\partial y ^ 2}= 0.
            $$
    \end{opomba}

    \begin{definicija}
        \emph{Območje} $D$ je povezana odprta množica v $\mathbb{R}^n$.
        Če obstaja $a \in D$, da za vse $b \in D$ in vse $t \in [0,~1]$ tudi $t a + (1-t)b \in D$, pravimo, da je $D$ \emph{zvezdasto območje}.
    \end{definicija}


    \begin{trditev}
        \label{hh}
        Naj bo $U \subseteq \mathbb{C}$ odprta množica. Naj bo $f = u + iv$ holomorfna funkcija, $u$ in $v$ pa realni funkciji, definirani na $U$. Potem sta realni funkciji $u$ in $v$ harmonični na $U$.
    \end{trditev}

    \begin{dokaz}
        Ker je $f$ holomorfna, zadošča Cauchy-Riemannovemu sistemu enačb prek katerih lahko bralec sam hitro preveri, da trditev drži. Dokaz je natančneje naveden v \cite{osnova}, na strani $55$.
    \end{dokaz}

    \begin{opomba}
        Če označimo $u = \text{Re}{f}$ in $v = \text{Im}{f}$, nam zgornja trditev v resnici pove, da sta realni in imaginarni del holomorfne funkcije harmonični funkciji. 
    \end{opomba}

    \begin{definicija}
        Naj bo $u$ realna harmonična funckija, definirana na območju $D$. Če obstaja realna harmonična funkcija $v$, definirana na $D$, da je funkcija $f = u + iv$ na $D$ holomorfna, potem funkciji $v$ pravimo \emph{harmonična konjugiranka funkcije $u$ na $D$}.    
    \end{definicija}

    \begin{trditev}
        \label{konj}
        Naj bo $u$ realna harmonična funkcija, definirana na zvezdastem območju $D$. Potem za $u$ na $D$ obstaja harmonična konjugiranka $v$ in je do konstante natančno enolično določena. 
    \end{trditev}
    \begin{dokaz}
        Konstrukcijo harmonične konjugiranke, si bralec za primer, ko je zvezdasto območje kar odprt disk lahko ogleda v \cite{osnova}, na strani $56$ in $57$. 
        Ideja dokaza za splošno zvezdasto območje je podobna. 
    \end{dokaz}

    \begin{opomba}
        V duhu zgornjih dveh trditev, je velikokrat smiselno na realno harmonično funkcijo $u$ smiselno gledati kot na realni del holomorfne funkcije $f = u + iv$, kjer je $v$ njena harmonična konjugiranka. To že nakazuje, da se bodo nekatere lepe lastnosti holomorfnih funkcij prenesle tudi na harmonične funkcije.
    \end{opomba}
    %%V duhu zgornjih dveh opomb, opazimo, da lahko kompleksno funkcijo $f$, zapišemo v obliki realnega in imaginarnega dela oziroma kot $f = u + i v$, s pomočjo nekih realnih funkcij (dveh spremenljivk) $u$ in $v$, ter lahko zaradi linearnosti parcialnih odvodov sklepamo, da je za harmoničnost $f$ kot kompleksne funkcije, dovolj zahtevati harmoničnost $u$ in $v$ kot realnih funkcij.
    %%Podoben argument nam da vedeti, da nam harmoničnost $f = u + iv$, implicira tudi harmoničnost $u$ in $v$. 
    \begin{opomba}
        \label{lin}
        Zaradi linearnosti parcialnih odvodov, je tudi linearna kombinacija harmoničnih funkcij harmonična. To bomo pri reševanju enega izmed glavnih problemov diplomskega dela s pridom uporabili.
    \end{opomba}
    \begin{opomba}
        V literaturi se v definiciji harmoničnosti za funkcijo $u$ pojavlja tudi zahteva, da je funkcija gladka, oziroma $u \in C^{\infty}$. V zgornji definiciji zahtevamo le obstoj drugih parcialnih odvodov, oziroma $u \in C^2$. 
        Da sta definiciji med seboj ekvivalentni, potrdi spodnja trditev. 
    \end{opomba}
    \begin{trditev}
        \label{gladkosth}
        Naj bo $u$ realna harmonična funkcija definirana na območju $D$. Potem je $u$ na $D$ gladka, oziroma $u \in C^{\infty}(D)$. 
    \end{trditev}
    \begin{dokaz}
        Ker je  $D$ odprta množica, lahko za vsako točko $z \in D$ najdemo dovolj majhno zvezdasto okolico $U_z$, da bo okolica v celoti vsebova v $D$. Dovolj je vzeti kar dovolj majhen disk. 
        Po trditvi \ref{konj} na $U_z$ obstaja harmonična konjugiranka $v$, da je $f = u+ iv$ na $U_z$ holomorfna, ter zato na $U_z$ tudi gladka. Sledi, da je na $U_z$ gladka tudi funkcija $u$.
    \end{dokaz}

\newpage
\section{Lastnost povprečne vrednosti}

    \begin{definicija}  
        Naj bo $h$ zvezna funkcija, definirana na območju $D$. Denimo, da za $z_0 \in D$ in $r > 0$ velja $\overline{\mathbb{D}}(z_0, r) \subseteq D$. \emph{Povprečje funkcije} $h$ na $\overline{\mathbb{D}}(z_0, r)$ definiramo kot:
        $$
            A(r) = \int_{0}^{2 \pi}{h \big(z_0 + r e^{i\theta}\big)\frac{d\theta}{2 \pi}}.
        $$
    \end{definicija}
    \begin{trditev}
        \label{zvpov}
        Naj bo $h$ zvezna funkcija, definirana na območju $D$, ter denimo, da za $z_0 \in D$ in $r > 0$ velja $\overline{\mathbb{D}}(z_0, r) \subseteq D$. 
        Potem je $A(r)$ zvezna na $(0,~r]$ in velja: $\lim_{r \to 0}{A(r)} = h(z_0)$.
    \end{trditev}
    \begin{dokaz}
        Ker je povprečje zvezne funkcije definirano kot integral zvezne funkcije, je funkcija $A(r)$ zvezna na $(0,~r]$. Drugi del trditve dokažimo po definiciji.
        Naj bo $\epsilon > 0$ poljubno majhen. Velja:
        $$
            |A(r) - h(z_0)| = \bigg|\int_{0}^{2\pi} \big[h(z_0 + r e^{i\theta})  - h(z_0)\big] \frac{d\theta}{2\pi} \bigg| \leq \int_{0}^{2 \pi} \big| h(z_0 + r e^{i\theta}) - h(z_0) \big| \frac{d\theta}{2 \pi}.
        $$
        Ker je $h$ zvezna, obstaja $\delta > 0$, da za vsak $|r| < \delta$ in vsak $\theta \in [0,~2\pi]$ velja \mbox{$|h(z_0 + r e^{i\theta}) - h(z_0)| < \epsilon$}.
        Za vsak $r$, ki je manjši od $\delta$ torej $|A(r) - h(z_0)| < \epsilon$. Ker je bil $\epsilon$ poljubno majhen, je trditev s tem dokazana.
    \end{dokaz}

    \begin{definicija}
        Naj bo $h$ zvezna funkcija, definirana na območju $D \subseteq \C$. Pravimo, da ima $h$ na $D$ \emph{lastnost povprečne vrednosti}, če za vsak $z_0 \in D$ obstaja $\epsilon_0 > 0$, da je $\overline{\mathbb{D}}(z_0, \epsilon_0) \subseteq D$ in za vsak $0 < \epsilon \leq \epsilon_0 $ velja:
        $$
            h(z_0) = \frac{1}{2 \pi} \int_{0}^{2 \pi}{h(z_0 + \epsilon e^{i \theta}) d\theta}.
        $$
    \end{definicija}
    \begin{opomba}
        Definicija nam pove, da ima zvezna funkcija $h$ na območju $D \subseteq \C$ lastnost povprečne vrednosti, če za vsak $z_0$ iz $D$ velja, 
        da je $h(z_0)$ povprečje vrednosti $h(z)$, kjer $z$ teče po majhni krožnici s središčem v $z_0$.
    \end{opomba}
    \begin{trditev}
        \label{linlpv}
        Linearna kombinacija funkcij z lastnostjo povprečne vrednosti je funkcija z lastnostjo povprečne vrednosti. 
    \end{trditev}
    \begin{dokaz}
        Trditev sledi iz linearnosti integrala.
    \end{dokaz}

    \begin{lema}
        Naj bo $u$ zvezna realna funkcija, definirana na območju $D$. Naj ima $u$ na $D$ lastnost povprečne vrednosti in naj obstaja $M \in \mathbb{R}$, da za vsak $z \in D$ velja: $u(z) \leq M$. 
        Če obstaja $z_0 \in D$, da velja: $u(z_0) = M$, potem je $u(z) = M$ za vsak $z \in D$.
    \end{lema}
    \begin{dokaz}
        Ker je $u$ na $D$ zvezna, je množica $A = \{z \in D~|~u(z) < M\}$ odprta. Po definiciji dokažimo, da je odprta tudi množica $B = \{z \in D~|~u(z) = M\}$. Naj bo $z_1 \in B$. Ker je $D$ odprta, obstaja $\rho_{z_1}$, da je za vsak $0 < r < \rho_{z_1}$ tudi $\overline{\mathbb{D}}(z_1, r) \subseteq D$. 
        Funkcija $u$ ima na $D$ lastnost povprečne vrednosti, zato velja:
        $$
            u(z_1) = \int_{0}^{2\pi}{u(z_1 + re^{i\theta}) \frac{d\theta}{2\pi}},~\text{za vsak}~0<r<\rho_{z_1}, 
        $$
        oziroma: 
        $$
        0 = \int_{0}^{2\pi}{\left[u(z_1) - u(z_1 + re^{i\theta}) \right]\frac{d\theta}{2\pi}},~\text{za vsak}~0<r<\rho_{z_1}.
        $$
        Ker je integrand nenegativen in zvezen, integral pa enak nič, je integrand enak nič. 
        Sledi, da je $M = u(z_1) = u(z_1 + r e^{i \theta})$, za vsak $\theta \in [0,~2\pi]$ in vsak $0 < r < \rho_{z_1}$, oziroma ima poljuben $z_1 \in B$ okolico, ki je vsebovana v $B$. Torej je $B$ odprta.
        Predpostavka nam pove, da za vsak $z \in D$ velja $u(z) \leq M$, zato je $D$ disjunktna unija množic $A$ in $B$. Ker je $D$ povezana, $A$ in $B$ pa odprti množici, je ena izmed niju prazna, druga pa posledično enaka $D$. Ker po predpostavki $B$ vsebuje $z_0$, je $A$ prazna, $B$ pa je enaka $D$. Sledi, da je $u(z) = M$ za vsak $z \in D$. 
    \end{dokaz}

    \begin{trditev}[Princip maksima za funkcije z lastnostjo povprečne vrednosti]
        
    \end{trditev}
    \begin{dokaz}
        test
    \end{dokaz}

    \begin{trditev}
        posledica princip maxima
    \end{trditev}
    \begin{dokaz}
        test
    \end{dokaz}


    \begin{trditev}
        Naj bo $f$ holomorfna funkcija na območju $D$. Potem ima funkcija $f$ na $D$ lastnost povprečne vrednosti.
    \end{trditev}
    \begin{dokaz}
        Ker je funkcija $f$ na območju $D$ holomorfna, lahko uporabimo Cauchyjevo integralsko formulo za vsak zaprt disk, ki je v celoti vsebovan v $D$. Za vsak $z \in D$ in vsak $r$, kjer je $\overline{\mathbb{D}}(z,r) \subseteq \mathbb{D}$ torej velja:
        $$
        f(z) = \frac{1}{2 \pi i} \int_{\partial \overline{\mathbb{D}}(z, r)}{\frac{f(\xi)}{\xi  - z}}d\xi.
        $$
        Ko rob diska parametriziramo z $\xi = z + r e^{i \varphi}$, dobimo:
        $$
        f(z) = \frac{1}{2 \pi} \int_{0}^{2\pi}{f(z + re^{i\varphi})}d\varphi.
        $$
    \end{dokaz}

    \begin{trditev}
        \label{harmonicnapovp}
        Naj bo $u$ harmonična funkcija, definirana na območju $D \subseteq \C$. Naj bo $z_0 \in D$ in $\rho > 0$, da velja $\overline{\mathbb{D}}(z_0, \rho) \subseteq D$. Za vsak $0 < r < \rho$ potem velja:
            $$
                u(z_0) = \frac{1}{2 \pi} \int_{0}^{2 \pi}{u(z_0 + r e^{i \theta}) d\theta}.
            $$
    \end{trditev}
    \begin{dokaz}
        Na $\overline{\mathbb{D}}(z_0, \rho) \subseteq D$ označimo $P = -\frac{\partial u}{\partial y}$ in $Q = \frac{\partial u}{\partial x}$ uporabimo Greenovo integralsko formulo:
        $$
            \int_{\partial \overline{\mathbb{D}}(z_0, \rho)}{P dx + Q dy} = \iint_{\overline{\mathbb{D}}(z_0, \rho)}{\bigg(\frac{\partial Q}{\partial x} - \frac{\partial P}{\partial y}\bigg)dx dy}.
        $$ 
        Sedaj upoštevajmo harmoničnost $u$ in vstavimo v Greenovo formulo:
        $$
        \int_{\partial \overline{\mathbb{D}}(z_0, \rho)}{-\frac{\partial u}{\partial y} dx + \frac{\partial u}{\partial x} dy} = \iint_{\overline{\mathbb{D}}(z_0, \rho)}{\bigg(\frac{\partial^2 u}{\partial x^2} + \frac{\partial^2 u}{\partial y^2}\bigg)dx dy} = 0. 
        $$
        Rob diska lahko pri $z_0 = x_0 + iy_0$ parametriziramo z $x(\theta) = x_0 + \rho \cos(\theta),~y(\theta) = y_0 + \rho \sin(\theta)$. Ko to vstavimo v zgornjo enakost, dobimo:
        $$
        0 = \rho \int_{0}^{2 \pi}{\bigg[\frac{\partial u}{\partial x} \cos(\theta) + \frac{\partial u}{\partial y} \sin(\theta)\bigg] d\theta} = \rho \int_{0}^{2\pi}{\frac{\partial u}{\partial \rho}\big({z_0 + \rho e^{i\theta}\big)d\theta}}.
        $$
        Ker je $u$ harmonična, je gladka, zato lahko zamenjamo limitna procesa integriranja in odvajanja. Delimo z $2\pi \rho$ in zapišemo:
        $$
        0 = \frac{\partial}{\partial \rho} \int_{0}^{2\pi}{u\big({z_0 + \rho e^{i\theta}\big)\frac{d\theta}{2 \pi}}}.
        $$
        Sledi, da je vrednost zgornjega integrala konstantna, za vsak $0 <r < \rho$. Obstaja $c \in \mathbb{C}$, da je: 
        $$
        \int_{0}^{2\pi}{u\big({z_0 + r e^{i\theta}\big)\frac{d\theta}{2 \pi}}} = c,~\text{za vsak}~ 0 < r < \rho.
        $$
        Potrebno je le še pokazati, da $c = u(z_0)$.
        Ker je $u$ zvezna, po trditvi \ref{zvpov} velja, da v limiti $r \to 0$ dobimo:
        $$
        c = \lim_{r \to 0}{\bigg[\int_{0}^{2\pi}{u\big({z_0 + r e^{i\theta}\big)\frac{d\theta}{2 \pi}}}\bigg]} = \int_{0}^{2\pi}{{u(z_0)~\frac{d\theta}{2 \pi}}} = u(z_0).
        $$
    \end{dokaz}


    PAZI KAM TOO
    \begin{opomba}
        Komentirali smo že, da imajo holomorfne funkcije lastnost povprečne vrednosti, zgornja trditev pa nam pove, da imajo lastnost povprečne vrednosti tudi harmonične funkcije. 
        Za harmonične funkcije velja celo več, da se jih namreč karakterizirati s pomočjo lastnosti povprečne vrednosti. 
    \end{opomba}
    PAZI KAM TOO

    \begin{trditev}
        \label{ekvhlp}
        Naj bo $h$ zvezna funkcija, definirana na območju $U \subseteq \C$. Velja, da je $h$ harmonična funkcija natanko tedaj, ko ima lastnost povprečne vrednosti.
    \end{trditev}
    \begin{dokaz}
        Trditev \ref{harmonicnapovp} nam dokaže eno implikacijo. Dovolj je torej preveriti, da je zvezna funkcija $h$ z lastnostjo povprečne vrednosti na $U$ harmonična. 
        Dokaz temelji na že dokazanem obstoju rešitve Dirichletovega problema za enotski disk oziroma na opombi \ref{alldisk}, ki nam potrdi obstoj rešitve Dirichletovega problema za poljuben disk. 
        \newline
        Ker je $U$ območje za poljubno točko $z_0 \in U$ obstaja $r$, da je $\overline{\mathbb{D}}(z_0,r) \subseteq U$. Ker je $h$ zvezna na $U$, $h$ na $\partial \overline{\mathbb{D}}(z_0, r)$ določa robne/začetne pogoje za Dirichletov problem za disk $\mathbb{D}(z_0,r)$.
        Vemo, da rešitev $\widetilde{h}$ obstaja. 
        Kot harmonična funkcija na $\mathbb{D}(z_0, r)$ ima po trditvi \ref{harmonicnapovp} na $\mathbb{D}(z_0, r)$ lastnost povprečne vrednosti. 
        Oglejmo si sedaj funkcijo $g(z) = h(z) - \widetilde{h}(z)$ za $z \in \overline{\mathbb{D}}(z_0,r)$. 
        Kot razlika funkcij z lastnostjo povprečne vrednosti ima po trditvi \ref{linlpv} tudi $g$ na $\mathbb{D}(z_0, r)$ lastnost povprečne vrednosti in je kot razlika dveh zveznih funkcij zvezna na $\overline{D}(z_0, r)$.
        Vemo celo, da na $\partial \overline{\mathbb{D}}(z_0, r)$ velja $g \equiv 0$, saj je zožitev $\widetilde{h}$ na $\partial \overline{\mathbb{D}}(z_0,r)$ enaka $h$. 
        Po principu maksima, za funkcije z lastnostjo povprečne vrednosti je potem $g \equiv 0$. Oziroma $\widetilde{h} \equiv h$ na $\overline{\mathbb{D}}(z_0, r)$, kar nam da harmoničnost $h$ na $\mathbb{D}(z_0, r)$. 
        Dokazali smo torej, da je $h$ harmonična v okolici vsake točke $z_0$ v $U$. To pa nam že zagotavlja harmoničnost $h$ na $U$. 
    \end{dokaz}
    \begin{trditev}
        Naj bo $u$ zvezna funkcija z lastnostjo povprečne vrednosti, definirana na območju $D$. Potem je $u$ na $D$ gladka, $u \in C^{\infty}(D)$.
    \end{trditev}
    \begin{dokaz}
        Po trditvi \ref{ekvhlp} je $u$ harmonična, po trditvi \ref{gladkosth} pa zato tudi gladka.
    \end{dokaz}

\section{Dirichletov problem za enotski disk}
    Naj bo $\mathbb{D}$ enotski disk. Zvezno kompleksno funkcijo $h$, definirano na $\partial \mathbb{D}$, bi želeli razširiti do zvezne funkcije $\widetilde{h}$, da bo $\widetilde{h}$ harmonična na $\mathbb{D}$ in zvezna na $\overline{\mathbb{D}}$, ter se bo zožitev $\widetilde{h}$ na $\partial \mathbb{D}$ ujemala s $h$.

    Kot smo že komentirali po definiciji \ref{harm} so funkcije harmonične, natanko tedaj ko zadoščajo Laplaceovi parcialni diferencialni enačbi.   
    Vredno je omeniti, da lahko na Direchletov problem za enotski disk gledamo tudi iz stališča teorije diferencialnih enačb. Gre za problem iskanja funkcije, ki na notranjosti območja reši Laplaceovo diferencialno enačbo ob robnem pogoju, ki ga določa v naprej podana funkcija na robu območja.     
    V našem primeru, je notranjost območja kar notranjost enotskega diska, roben pogoj zvezne funkcije pa podaja začetna zvezna funkcija, podana na enotski krožnici.
    Navadno se v teoriji parcialnih diferencialnih enačb srečamo s tako imenovanimi dobro postavljenimi matematičnimi problemi, o katerih si bralec več lahko prebere na REFERENCA.     
    Spodaj se bomo le seznanili z njihovo definicijo in pokazali, da zgoraj zastavljen Dirichletov problem ustreza definiciji. 

    \begin{definicija}[J. Hadamard 1902]
        Pravimo, da je matematičen problem (parcialni diferencialnih enačb z robnimi in začetnimi pogoji) \emph{dobro postavljen}, če zanj velja:
        \begin{itemize}
            \item rešitev problema obstaja,
            \item rešitev problema je ena sama, oziroma rešitev je enolično določena,
            \item rešitev je zvezno odvisna od začetnih podatkov problema.
        \end{itemize}
    \end{definicija}

    \begin{lema}
        \label{enolicno}
        Če rešitev za Dirichletov problem na enotskem disku obstaja, je enolično določena.
    \end{lema}
    \begin{dokaz}
        Denimo, da obstajata dve različni rešitvi, $\widetilde{h_1}$ in $\widetilde{h_2}$, Dirichletovega problema za enotski disk.
        Oglejmo si razliko $h_1 - h_2$. Vemo, da je njuna razlika na $\partial \mathbb{D}$ enaka $0$, saj sta tam enaki $h$. 
        Zaradi harmoničnosti $h_1$ in $h_2$ po principu o maksimu za harmonične funkcije vemo, da je njuna razlika ničelna tudi na $\mathbb{D}$. Sledi enakost $h_1$ in $h_2$ tudi na $\mathbb{D}$ in protislovje. 
    \end{dokaz}
    
    \begin{opomba}
        \label{op1}
        Lema \ref{enolicno} nam pove, da je rešitev problema največ ena, zato se je dovolj posvetiti konstrukciji potencialne rešitve. 
        Opazimo, da lahko točke $z \in \partial \mathbb{D}$ zapišemo z $e^{i \theta} \in \partial \mathbb{D}$, ter funkcijo $h(z)$ pišemo kot kompozitum $h(e^{i \theta})$.
        Poskusimo sedaj skonstruirati (harmonično) razširitev, ki bo zadoščala Dirichletovemu problemu za zvezno funkcijo $h(e^{i \theta})$.
    \end{opomba}

    \paragraph[short]{\emph{Konstrukcija}}
    Najprej posvetimo enostavnim primerom in si za tem teorijo oziroma konstrukcijo ogledamo v splošnem. 
    Naravno je za enostavne zvezne funkcije, definirane na $\partial \mathbb{D}$, vzeti kar polinome. V luči opombe \ref{op1} je polinomska spremenljivka lahko kar $e^{i\theta}$, s pomočjo opombe \ref{lin} pa vidimo, da zadošča rešitve poiskati za monome. 
    Zato si oglejmo funkcije oblike $h(e^{i \theta}) = e^{i k \theta}, k \in \mathbb{Z}$, ter za njih poskusimo skonstruirati željeno razširitev. 
    Hitro opazimo, da se nam v tem primeru pojavlja preprosta eksplicitna razširitev s predpisom $\widetilde{h}(r e^{i \theta}) = r^{|k|}e^{i k \theta},~\text{za}~r\in [0, 1]~\text{in}~~\theta \in [0, 2\pi]$. 
    Tako predpisana razširitev je za $k \geq 0$ očitno harmonična na $\mathbb{D}$ ($r^k e^{ik\theta} = z^k$ je namreč celo holomorfna funkcija), pri $k < 0$ pa dobimo razširitev, ki v splošnem ni holomorfna, a je harmonična ($r^{-k} e^{ik\theta} = \overline{z}^{-k}$ je monom v konjugirani spremenljivki, ki je harmoničen).
    Prav tako je za vsak $k \in \mathbb{D}$ razširitev očitno zvezna na $\overline{\mathbb{D}}$ in se na $\partial \mathbb{D}$ ujema z začetnimi pogoji, zato je rešitev Dirichletovega problema, ki je po lemi \ref{enolicno} enolična. 
    Z upoštevanjem linearnosti lahko za začetne funkcije oblike $h(e^{i\theta}) = \sum_{k = -N}^{N}{a_k e^{ik\theta}}$, konstruiramo razširitev s predpisom
    $\widetilde{h}(r e^{i \theta}) = \sum_{k = -N}^{N}{a_k r^{|k|}e^{ik\theta}}$. Smiselno bi bilo koeficiente $a_k$ izraziti direktno prek funkcije $h$. 
    V ta namen si oglejmo $\int_{-\pi}^{\pi}{e^{ij\theta} e^{-ik\theta}d\theta}$. S hitrim izračunom preverimo tako imenovano ortogonalno relacijo med kompleksnimi eksponenti:
        $$
        \int_{-\pi}^{\pi}{e^{ij\theta} e^{-ik\theta}\frac{d\theta}{2\pi}} = 
        \begin{cases}
            1~&j=k\\
            0~&j \neq k\\
        \end{cases}
        .$$

        Zgornja relacija nam za $h(e^{i\theta}) = \sum_{k = -N}^{N}{a_k e^{ik\theta}}$ omogoča izražavo koeficientov $a_k$ kot:
        $$
            a_ k = \int_{-\pi}^{\pi}{h(e^{i\theta}) e^{-ik\theta}\frac{d\theta}{2\pi}}.
        $$
    Sedaj uporabimo zgornjo zvezo in izrazimo:
    $$
        \sum_{k = - N}^{N}{ a_k r^{|k|}e^{ik\theta}} = \sum_{k = - N}^{N} \bigg(\int_{-\pi}^{\pi}{h(e^{i \varphi}) e^{- i k \varphi} \frac{d \varphi}{2 \pi}}\bigg) r^{|k|} e^{i k \theta}.
    $$
    Zamenjamo vsoto in integral, ter brez škode iteracijsko območje vsote razširimo na vsa cela števila (pri dodanih indeksih je člen vsote ničeln). Dobimo željeni ekspliciten zapis:
    \begin{equation}
        \label{int1}
        \tag{P}
        \widetilde{h}(r e^{i \theta}) = \int_{-\pi}^{\pi}{h(e^{i \varphi}) \bigg[\sum_{k = - \infty}^{\infty} r^{|k|} e^{- i k \varphi} e^{i k \theta}} \bigg]\frac{d \varphi}{2 \pi}, ~~~ r e^{i\theta} \in \overline{\mathbb{D}}.
    \end{equation}
    Že iz načina konstrukcije in sprotnih komentarjev je jasno, da funkcija te oblike za primere, ko je $h$ (trigonometričen) polinom, reši Dirichletov problem.
    Sedaj bomo zgornjo funkcijo, ki smo jo na intuitiven način konstruirali s pomočjo začetnega pogoja polinomov $h$ vzeli za definicijo novega pojma in z njegovo pomočjo prišli do razširitve za splošne zvezne funkcije $h$.
    \begin{definicija}
        \emph{Poissonovo jedro} je funkcija definirana s predpisom
        $$
           P_r(\theta) = \sum_{k = -\infty}^{\infty}{r^{|k|} e^{i k \theta}}\text{, kjer je}~\theta \in [-\pi, \pi]~\text{in}~ r < 1.
        $$
    \end{definicija}
    Na Poissonovo jedro lahko gledamo kot funkcijo dveh spremenljivk ($\theta$ in $r$), ali pa kot na družino funkcij, indeksiranih s parametrom $r$.
    \newline
    Smiselno se je vprašati, ali za vsako vrednost iz definicijskega območje Poissonovega jedra vrsta na desni strani definicijske enakosti sploh konvergira. Potencialne strahove pomiri naslednja trditev.
    \begin{trditev}
        Za vsak fiksen $\rho < 1$ vrsta, definirana s Poissonovim jedrom konvergira enakomerno za vsak $r\leq \rho < 1$ in vsak $ -\pi \leq \theta \leq \pi$.
    \end{trditev}
    \begin{dokaz}
        Za vsak člen vrste velja $|r^{|k|} e^{i k \phi}| \leq \rho^{|k|}$, zato po Weierstrassovem M-testu sledi, da vrsta konvergira enakomerno.
    \end{dokaz}

    Definicijo Poissonovega jedra bi lahko ekvivalentno zapisali tudi kot:
        \begin{equation}
            \label{eq1}
            P_r(\theta) = 1 + \sum_{k=1}^{\infty}{z^k} + \sum_{j=1}^{\infty}{\overline{z}^{j}},~\text{kjer}~z = r e^{i\theta} \in \mathbb{D},~~~\text{oziroma}
        \end{equation}
        \begin{equation}
            \label{eq2}
            P_r(\theta) = \frac{1 - |z|^2}{|1-z|^2} = \frac{1-r^2}{1+ r^2 - 2r \cos(\theta)},~\text{kjer}~z= re^{i\theta} \in \mathbb{D}.
        \end{equation}
    Enostavno je pokazati ekvivalenco z zapisom \ref{eq1}. Vrsto v definiciji le razbijemo na tri dele (glede na predznačenost iterativnega indeksa) in člen vsake izmed vsot zapišemo s kompleksno spremenljivko. 
    Za dokaz ekvivalence obliko \ref{eq2} se moramo nekoliko bolj potruditi in si pomagati z \ref{eq1}. Vrsti v \ref{eq1} sta definirani na notranjosti enotskega diska, kjer je absolutna vrednost manjša od 1, zato obe vrsti konvergirata in ju lahko seštejemo s pomočjo formule za geometrijsko vrsto. 
    Prek enakosti $|1 - z|^2 = (\overline{1 - z})(1 - z) = (1 - \overline{z})(1 - z) = 1 + r^2 - 2r \cos(\theta)$ dobimo ekvivalenco z zapisom \ref{eq2}:
    $$
        P_r(\theta) = 1 + \frac{z}{1 - z} + \frac{\overline{z}}{1 - \overline{z}} = \frac{1 - |z|^2}{|1 - z|^2} = \frac{1 - r^2}{1 + r^2 - 2r\cos(\theta)}.
    $$
    Preden se lotimo uporabe Poissonovega jedra, si oglejmo še nekaj njegovih lastnosti. 
    
    \begin{trditev}
        \label{lastpk}
        Funkcija Poissonovega jedra ima naslednje lastnosti:
        \begin{enumerate}
            \item funkcija Poissonovega jedra je periodična s periodo $2\pi$, 
            \item za vsak fiksen $r \in [0,1)~\text{velja}~\int_{-\pi}^{\pi}{P_r(\theta) \frac{d\theta}{2\pi}} = 1$,
            \item za vsak fiksen $r \in [0,1)~\text{in}~-\pi \leq \theta \leq \pi~\text{je}~P_r(\theta) > 0$,
            \item za vsak fiksen $r \in [0,1)~\text{in}~-\pi \leq \theta \leq \pi~\text{je}~P_r( - \theta) = P_r(\theta)$,
            \item za vsak fiksen $r \in [0,1): P_r(\theta)~\text{za}~\theta \in [-\pi, 0 ]~\text{narašča in za}~\theta \in [0, \pi ]~\text{pada}$,
            \item za vsak fiksen $\delta > 0: \max\{P_r(\theta)~|~ \delta \leq |\theta| \leq \pi\} \to 0$ ko gre $r \to 1$,
            \item Poissonovo jedro je kot funkcija dveh spremenljivk ($r$ in $\theta$) harmonična.
        \end{enumerate}
    \end{trditev}
    \begin{dokaz}
        $ $
        \begin{enumerate}
            \item Periodičnost funkcije s periodo $2\pi$ je zaradi definiranosti prek vsote členov $e^{ik\theta}$ očitna. 
            \item Oglejmo si \ref{int1} in vzemimo $h \equiv 1$. Dobimo: 
            $$
                \widetilde{h}(r e^{i \theta}) = \int_{-\pi}^{\pi}{\bigg[\sum_{k=-\infty}^{\infty}{r^{|k|} e^{ik(\varphi - \theta)}}\bigg] \frac{d \varphi}{2 \pi}} = \int_{-\pi}^{\pi}{P_r(\varphi - \theta)\frac{d \varphi}{2 \pi}}. 
            $$
            Opazimo, da $\widetilde{h} \equiv 1$ reši Dirichletov problem, zato je po lemi \ref{enolicno} edina rešitev. Uporabimo translacijo $\varphi \mapsto \varphi - \theta$ za uvedbo nove spremenljivke v integral ter ob upoštevanju periodičnosti dobimo:
            $$
                1 = \int_{-\pi}^{\pi}{P_r(\varphi - \theta)\frac{d \varphi}{2 \pi}} = \int_{-\pi}^{\pi}{P_r(\varphi)\frac{d \varphi}{2 \pi}}.
            $$
            \item Trditev očitno sledi iz definicije \ref{eq2}. 
            \item Trditev prek sodosti funkcije kosinus sledi iz definicije \ref{eq2}. 
            \item Trditev očitno sledi iz definicije \ref{eq2}.
            \item TODO 
            \item Po trditvi \ref{hh} zadošča pokazati, da lahko Poissonovo jedro zapišemo kot realni del holomorfne funkcije (definirane na $\mathbb{D}$). Opazimo, da po \ref{eq1} to res lahko storimo kot:
            $$
                P_r(\theta) = 1 + 2~\text{Re}\bigg(\frac{z}{1-z}\bigg) = \text{Re}\bigg(\frac{1+z}{1-z}\bigg),~\text{za}~z= re^{i\theta} \in \mathbb{D}.
            $$
        \end{enumerate}
    \end{dokaz}
    \begin{opomba}
        Lastnosti Poissonovega jedra, navedene v trditvi \ref{lastpk}, so tudi dobro razvidne na spodnji sliki. 
        \newline
        SLIKA - GRAF POISSONOVEGA JEDRA
        \newline
        Vredno je omeniti, da točki (2) in (3) trditve \ref{lastpk} nakazujeta, da funkcija $\frac{1}{2 \pi} P_r(\theta)$ določa gostoto zvezno porazdeljene slučajne spremenljivke.
    \end{opomba}
    Sedaj se vrnimo k reševanju Dirichletovega problema za enotski disk. 
    Spomnimo se, da smo za polinomske funkcije $h$ skonstruirali predpis razširitve, ki je rešila zastavljen problem. 
    Opazimo, da si sedaj lahko pomagamo z definiranim pojmom Poissonovega jedra in zapišemo:
    $$
    \widetilde{h}(r e^{i \theta}) = \int_{-\pi}^{\pi}{h(e^{i \varphi}) \bigg[\sum_{k = - \infty}^{\infty} r^{|k|} e^{- i k \varphi} e^{i k \theta}} \bigg]\frac{d \varphi}{2 \pi} = 
    \int_{-\pi}^{\pi}{h(e^{i \varphi}) P_r(\theta - \varphi)\frac{d \varphi}{2 \pi}}.
    $$
    Če (trigonometričen) polinom $h(e^{i\theta})$ (ki  bi seveda lahko zavzemal kompleksne vrednosti) zamenjamo s (trigonometričnim) polinomom $u(e^{i\theta})$, ki zavzema le realne vrednosti, se nam zgornja enakost v duhu 
    točke (7) trditve \ref{lastpk} še olepša. Takrat lahko namreč za $z = re^{i\theta} \in \mathbb{D}$ pišemo:
    \begin{equation}
        \label{realnidel}
        \begin{split}
            \widetilde{u}(z) = \widetilde{u}(r e^{i \theta}) = \int_{-\pi}^{\pi}{u(e^{i \varphi}) P_r(\theta - \varphi)\frac{d \varphi}{2 \pi}} = \int_{-\pi}^{\pi}{u(e^{i \varphi})~\text{Re}\bigg(\frac{1+re^{i(\theta - \varphi)}}{1-re^{i(\theta - \varphi)}}\bigg)\frac{d \varphi}{2 \pi}}= \\
            \int_{-\pi}^{\pi}{u(e^{i \varphi})~\text{Re}\bigg(\frac{e^{i\varphi}+re^{i\theta}}{e^{i\varphi}-re^{i\theta}}\bigg)\frac{d \varphi}{2 \pi}}=~\text{Re}~\bigg[\int_{-\pi}^{\pi}{u(e^{i \varphi})\bigg(\frac{e^{i\varphi}+z}{e^{i\varphi}-z}\bigg)\frac{d \varphi}{2 \pi}}\bigg].
        \end{split}
    \end{equation}
    S tem smo $\widetilde{u}(z)$ za $z \in \mathbb{D}$ izraziti kot realni del holomorfne funkcije, kar ne implicira le harmoničnost predpisa (po trditvi \ref{hh}), ampak tudi gladko odvisnost $\widetilde{u}(z)$ od spremenljivke $z$.
    \newline
    Posvetimo se sedaj spet iskanju splošne rešitve Dirichletovega problema. Zgoraj smo z intuitivno izpeljavo nevede že zapisali funkcijo, ki jo bomo sedaj vzeli za definicijo novega pojma. Pokazali smo, da to v splošnem pripelje do rešitve.

    \begin{definicija}
        \emph{Poissonov integral}, ki ga označimo z~$\widetilde{h}(z)$, zvezne funkcije $h(e^{i\theta})$, je funkcija, definirana na enotskem disku s predpisom
        $$
        \widetilde{h}(z) = \int_{-\pi}^{\pi}{h(e^{i\varphi}) P_r(\theta - \varphi)~\frac{d\varphi}{2 \pi}}~\text{, kjer je}~~z = r e^{i\theta} \in \mathbb{D}.
        $$
     \end{definicija}
     \begin{opomba}
        Ekvivalentno bi lahko definicijo Poissonovega integrala, zaradi 2 $\pi$ periodičnosti Poissonovega jedra, zapisali tudi kot:
        $$
        \widetilde{h}(z) = \int_{-\pi}^{\pi}{h\big(e^{i(\theta-\varphi)}\big) P_r(\varphi)~\frac{d\varphi}{2 \pi}}~\text{, kjer}~~z = r e^{i\theta} \in \mathbb{D}.
        $$
     \end{opomba}
     \begin{trditev}
        \label{lastpi}
        Za preslikavo $\Phi : h \mapsto \widetilde{h}$, t.j. preslikavo, ki zvezni funkciji $h$, definirani na $\partial \mathbb{D}$, priredi njen Poissonov integral, velja:
        \begin{enumerate}
            \item $\Phi$ je linearna preslikava, t.j. $\Phi(c_1 h_1 + c_2 h_2) = c_1 \widetilde{h_1} + c_2 \widetilde{h_2}$,
            \item $\Phi$ "ohranja omejenost", t.j. če je $|h| \leq M$ na $\partial \mathbb{D}$, potem je $|\widetilde{h}| \leq M$ na $\mathbb{D}$.
        \end{enumerate}
     \end{trditev}
     \begin{dokaz}
        $ $
        \begin{enumerate}
            \item Trditev trivialno sledi iz definicije Poissonovega integrala, zaradi linearnosti integrala. 
            \item Trditev sledi iz točke (2) in (3) trditve \ref{lastpk}.
        \end{enumerate}
     \end{dokaz}
        


     \begin{trditev}
        \label{obstoj}
        Naj bo $h$ zvezna kompleksna funkcija definirana na $\partial \mathbb{D}$. Rešitev Dirichletovega problema obstaja in je na $\mathbb{D}$ definirana kot Poissonov integral funkcije $h$.
        \newline
        Rečeno drugače, Poissonov integral zvezne kompleksne funkcije, definirane na $\partial \mathbb{D}$ je zvezna harmonična funkcija, definirana na $\mathbb{D}$, ki je zvezno harmonična razširitev $h$ na $\overline{\mathbb{D}}$, če jo dodefiniramo na $\mathbb{D}$.
     \end{trditev}
     \begin{dokaz}
        Trditev bomo dokazali v dveh korakih. Najprej dokažimo, da je Poissonov integral na $\mathbb{D}$ harmonična funkcija.
        Vemo, da lahko vsako kompleksno funkcijo $h$ razcepimo na njen realni in imaginarni del kot $h = u + iv$, pri čemer sta $u$ in $v$ funkciji, z zalogo vrednostjo znotraj realnih števil. 
        Zaradi linearnosti Poissonovega integrala, se lahko sedaj posebej skoncentriramo na Poissonov integral realnega dela (ki ga označimo z $\widetilde{u}$) in Poissonov integral imaginarnega dele (ki ga označimo z $\widetilde{v}$). 
        Tako, kot smo to naredili pri \ref{realnidel}, lahko vsakega izmed Poissonovih integralov $\widetilde{u}$ in $\widetilde{v}$ zapišemo kot realni del neke holomorfne funkcije, kar nam po trditvi \ref{hh} implicira harmoničnost $\widetilde{u}$ in $\widetilde{v}$.
        Zaradi opombe \ref{lin} nam to implicira tudi harmoničnost $\widetilde{h} = \widetilde{u} + i\widetilde{v}$ ter s tem dokaz prvega dela trditve. 
        \newline
        Za dokaz zveznosti na $\overline{\mathbb{D}}$ se moramo nekoliko bolj potruditi. Uporabili bomo lastnosti Poissonovega jedra iz trditve \ref{lastpk}. 
        Kot zvezna funkcija na kompaktu, je za $h$ na $\partial \mathbb{D}$ velja: 
        \begin{itemize}
            \item omejena, t.j.:$~\exists M: |h(e^{i\theta})| \leq M$ in 
            \item enakomerno zvezna, t.j.: $\forall \epsilon > 0, \exists \delta > 0: | \theta - \varphi | < \delta \Rightarrow |h(e^{i\theta}) - h(e^{i\varphi})| < \epsilon $.
        \end{itemize}
        Po točki (2) iz trditve \ref{lastpk} lahko zapišemo:
        $$
            \widetilde{h}(re^{i\theta}) - h(e^{i\theta}) = \int_{-\pi}^{\pi}{\bigg[h\big(e^{i(\theta - \varphi)}\big) - h(e^{i\theta})\bigg]P_r(\varphi)\frac{d\varphi}{2\pi}}.
        $$
        Sedaj vzamemo absolutno vrednost leve in desne strani enakosti, ter uporabimo trikotniško neenakost. Dobimo:
        $$
        |\widetilde{h}(re^{i\theta}) - h(e^{i\theta})| \leq \int_{-\pi}^{\pi}{\bigg| h\big(e^{i(\theta - \varphi)}\big) - h(e^{i\theta})\bigg|P_r(\varphi)\frac{d\varphi}{2\pi}}.
        $$
        Integral lahko razdelimo na dva dela, da bomo lahko uporabili enakomerno zveznost funkcije:
        $$
        |\widetilde{h}(re^{i\theta}) - h(e^{i\theta})| \leq \bigg(\int_{-\delta}^{\delta} + \int_{\delta \leq |\varphi| \le \pi}\bigg){\bigg| h\big(e^{i(\theta - \varphi)}\big) - h(e^{i\theta})\bigg|P_r(\varphi)\frac{d\varphi}{2\pi}}.
        $$
        Vrednosti znotraj prvega integral lahko zaradi enakomerne zveznosti navzgor ocenimo z $\epsilon$, vrednosti znotraj drugega integrala pa lahko prek omejenosti navzgor ocenimo kar z $2M$. 
        $$
        |\widetilde{h}(re^{i\theta}) - h(e^{i\theta})| \leq \epsilon \int_{-\delta}^{\delta}{P_r(\varphi) \frac{d\varphi}{2\pi}} + 2M\int_{\delta \leq |\varphi| \leq \pi}{P_r(\varphi)\frac{d\varphi}{2\pi}}.
        $$
        Vsakega od integralov lahko sedaj v duhu točke (2) trditve \ref{lastpk} navzgor ocenimo in dobimo:
        $$
        |\widetilde{h}(re^{i\theta}) - h(e^{i\theta})| \leq \epsilon  + 2M~\text{max}\{P_r(\varphi)~| ~\delta \leq |\varphi| \leq \pi \}.
        $$
        Sedaj uporabimo točko (6) trditve \ref{lastpk}, ki nam pove, da gre drugi sumand proti 0, ko gre $r$ proti 1. 
        Torej, za vsak $\widetilde{\epsilon} > 0$, pri $r$, ki je dovolj blizu 1, velja, da $|\widetilde{h}(re^{i\theta}) - h(e^{i\theta})| < \widetilde{\epsilon}$.
        Zveznost in s tem trditev je tako dokazana.
    \end{dokaz}
    \begin{posledica}
        Dirichletov problem za enotski disk je dobro postavljen matematičen problem. 
    \end{posledica}
    \begin{dokaz}
        Obstoj rešitve za vsako zvezno začetno podano funkcijo nam preko eksplicitne konstrukcije zagotavlja trditev \ref{obstoj}. 
        Enoličnost rešitve nam zagotavlja lema \ref{enolicno}, zvezna odvisnost rešitve od začetnih podatkov, pa je predpostavka problema (funkcija reši problem le v primeru, ko je zvezna na zaprtju enotskega diska, kar nam zagotavlja zvezno odvisnost od pogojev na robu diska oziroma zvezno odvisnost od začetnih pogojev).
    \end{dokaz}
    \begin{opomba}
        \label{alldisk}
        Dirichletov problem smo formulirali in reševali za enotski disk. Problem bi na podoben način lahko formulirali za poljubno območje $U$. 
        Kot začetni/robni pogoj bi podali zvezno funckijo $f$ (definirano na $\partial U$), ter iskali njeno razširitev $\widetilde{f}$ (na $\overline{U}$), tako da bi bila na $\overline{U}$ zvezna, 
        na $U$ harmonična, ter bi se zožitev $\widetilde{f}$ na $\partial U$ ujemala z $f$.
        V splošnem Dirichletovega problema znotraj diplomske naloge ne bomo reševali, vredno pa je omeniti, 
        da je bil prav enotski disk izbran zaradi preprostosti in bi lahko zgornji rezultat na podoben način zapisali za poljuben disk. To naredimo s pomočjo vpeljave novih spremenljivk ($z \mapsto az + b$) v Poissonov integral. 
        DOKONCAJ IZPELJAVO!
     \end{opomba}
\newpage
\section{Schwarzov princip zrcaljenja za harmonične funkcije}
    V tretjem poglavju, smo se že srečali s problemom razširitve podane funkcije do funkcije, ki mora zadoščati dodatnim pogojem. 
    Pri Dirichletovem problemu smo zahtevali harmoničnost razširitve na notranjosti območja/diska in zveznost razširitve na zaprtju območja/diska. 
    Pokazali smo, da to lahko naredimo s pomočjo Poissonovega integrala $\widetilde{h}$ kot:
    $$
        h^e(z) = 
        \begin{cases}
            h(z)~&z \in \partial \mathbb{D} \\
            \widetilde{h}(z)~&z \in \mathbb{D}
        \end{cases}.
    $$
    Problem, ki ga Schwarzov princip zrcaljenja reši je ne glede na drugačne začetne pogoje, iz vidika iskanja razširitve kljub temu blizu že rešenemu Dirichletovemu problemu.
    Glavna razlika je, da pri Schwarzovem principu zrcaljenja harmoničnost na območju vzamemo že za predpostavko, ter s pomočjo robnih zveznih začetnih pogojev poskušamo konstruirati harmonično razširitev na povsem novem območju podobne oblike.  
    Kot že nakazuje beseda zrcaljanje v imenu principa, bomo razširitev želeli konstruirati na "zrcalni sliki" prvotno definiranega območja. Prav zato bo potrebno predpostaviti nekaj simetrije.
    Spoznajmo sedaj nekaj osnovnih pojmov, ki nas bodo pripeljali do glavnega izreka.
    \begin{definicija}
        Naj bo $U$ območje. \emph{Zrcaljenje območja U čez realno os} definiramo kot $U^* = \{\overline{z}~|~z \in U\}$.
        \newline
        Pravimo, da je \emph{območje U simetrično glede na realno os}, če velja $U^* = U$.
        \newline
        Naj bo $u: U \to \mathbb{R}$. Potem lahko pri zgornjih oznakah definiramo $u^*: U^* \to \mathbb{R}$, kot $u^*(z) = u(\overline{z})$.
    \end{definicija}

    Naravno je pričakovati, da se zaradi dokaj preproste definicije $u^*$ lastnosti $u$ "prenesejo" tudi na $u^*$. 
    Spodnja lema nam to potrdi.

    \begin{lema}
        \label{lemaharm}
        Naj bo $U$ območje in $u: U \to \mathbb{R}$. Če je $u$ harmonična na $U$, je $u^*$ harmonična na $U^*$. 
    \end{lema}
    \begin{proof}
        Trditev lahko dokažemo na dva načina. Dokažimo najprej s pomočjo karakterizacije harmoničnih funkcij z lastnostjo povprečne vrednosti. 
        Ker je $u$ na $U$ harmonična ima na $U$ lastnost povprečne vrednosti.
        Sedaj je hitro jasno, da ima lastnost povprečne vrednosti tudi $u^*$ na $U^*$. 
        Iz trditve \ref{ekvhlp} potem sledi, da je $u^*$ na $U^*$ tudi harmonična.
        \newline
        Trditev je možno dokazati tudi kar direktno po definiciji, t.j. dokazati, da velja: $\frac{\partial^2 u^*}{\partial x^2} + \frac{\partial^2 u^*}{\partial y^2} = 0$. 
        Opazimo, da lahko $u^*$ (s pomočjo identifikacije $z = x + iy \leftrightarrow (x,y)$) zapišemo tudi kot kompozitum $u \circ ((x,y) \mapsto (x, -y))$. 
        Sedaj po definiciji parcialno odvajamo in dobimo enakosti:
        $$
            \frac{\partial u^*}{\partial y} = - \frac{\partial u }{\partial y},~~~~\frac{\partial^2 u^*}{\partial y^2} = \frac{\partial^2 u }{\partial y^2},~~~~\frac{\partial^2 u^*}{\partial x^2} = \frac{\partial^2 u }{\partial x^2},
        $$
        ki implicirajo:
        $$
            \frac{\partial^2 u^*}{\partial x^2} + \frac{\partial^2 u^*}{\partial y^2} = \frac{\partial^2 u}{\partial x^2} + \frac{\partial^2 u}{\partial y^2} = 0.
        $$
    \end{proof}
    \begin{lema}
        Če je $f$ holomorfna na območju $U$, je $g(z) = \overline{f({\overline{z}})}$ holomorfna na $U^*$.
    \end{lema}
    \begin{dokaz}
        Iz kompleksne analize vemo, da je dovolj pokazati, da tudi $g$ zadošča Cauchy-Riemannovemu sistemu enačb.
        \newline
        Pišimo: $f(z) = u(z) + iv(z)$ in $g(z) = p(z) + iq(z)$.
        Potem je $g(z) = u(\overline{z}) - iv(\overline{z})$ in zato $p(z) = u(\overline{z})~\text{ter}~q(z) = -v(\overline{z})$. 
        \newline
        Velja:
        \begin{equation*}
            \frac{\partial p}{\partial x} = \frac{\partial u}{\partial x} \overset{C-R}{=} \frac{\partial v}{\partial y} = - \bigg(- \frac{\partial q}{\partial y}\bigg) = \frac{\partial q}{\partial y}~~~\text{in}~~~
            \frac{\partial p}{\partial y} = -\frac{\partial u}{\partial y} \overset{C-R}{=} \frac{\partial v}{\partial x} = -\frac{\partial q}{\partial x}.
        \end{equation*}
        Alternativen dokaz si bralec lahko ogleda v REFERENCA.
    \end{dokaz}

    Sedaj smo pripravljeni na formulacijo in dokaz glavnega izreka.
    \begin{izrek}
        Naj bo $D \subseteq \C$ območje, simetrično glede na realno os. 
        Označimo $D^{+} = D \cap \{\text{Im} > 0\}$ in $D^{-} = D \cap \{\text{Im} < 0\}$.
        \newline
        Naj bo $u(z): D^{+} \to \mathbb{R}$ harmonična funkcija, za katero velja, da gre $u(z) \to 0$, ko gre $z \in D^{+}$ proti poljubni točki $D \cap \mathbb{R}$ t.j.: $$\lim_{\text{Im}(z) \to 0^+} u(z) = 0.$$
        Potem obstaja harmonična razširitev $u(z)$ na $D$, ki jo podaja predpis $u(\bar{z}) = - u(z)$ za $z \in D$:
        $$
            u^e(z) = 
            \begin{cases}
                u(z)~~&z \in D^{+}\\
                -u(\overline{z})~~&z \in D^{-}\\
                0~~ &z \in \mathbb{R}
            \end{cases}
            .
        $$
    \end{izrek}
    Preden se posvetimo dokazu, si na sliki oglejmo, kaj nam Schwarzov princip zrcaljenja zares omogoča.
    \begin{center}
        \includegraphics[width = 0.8 \textwidth]{schwarzov_princip_zrcaljenja.png}
    \end{center}
    Slika prikazuje glede na $\R$ simetrično območje $D \subseteq \C$. $D^{+}$ označuje območje, na katerem harmonično funkcijo $u$ poznamo, $D^{-}$ pa območje, na katerem lahko po izreku eksplicitno konstruiramo razširitev prek predpisa $u(\overline{z}) = - u(z)$.
    Vodoravna črta prikazuje odsek realne osi, na kateri bodo zaradi zahtevane zveznosti vrednosti trivialno poračunljive. 
    Točka $z_0$ in "majhen" disk okoli nje nakazujeta, da se bomo dokaza lotili z lastnostjo povprečne vrednosti.


    \begin{dokaz}
        Opazimo, da je $D$ disjunktna unija $D^+$, $D^-$ in $D^0 = D \cap \mathbb{R}$.
        Hitro je razvidno, da je $u^e$ na $D$ zaradi predpostavljene limite in ničelnih vrednosti na $D^0$ res zvezna. 
        \newline
        Posvetimo se dokazu harmoničnosti. Ker je $u^e$ na $D$ zvezna, je po trditvi \ref{ekvhlp} dovolj pokazati, da ima $u^e$ na $D$ lastnost povprečne vrednosti.
        Na $D^+$ je $u^e$ harmonična po predpostavki, na $D^-$ pa harmoničnost hitro preverimo prek leme \ref{lemaharm}. 
        Za poljubno točko $z_0 \in D^0$ pri dovolj majhnem $r$ (tako da je $\overline{\mathbb{D}}(z_0,r) \subseteq D$) velja, da se ob računanju povprečja paroma "zrcalne" vrednosti zaradi predpisa $u(\overline{z}) = - u(z)$ ravno odštejejo. 
        To pa že implicira, da je povprečje vrednosti na $\partial \overline{\mathbb{D}}(z_0,r)$ v poljubnem $z_0 \in D^0$ enaka $0$ in $u^e$ tudi v $z_0 \in D^0$ izpolnjuje lastnost povprečne vrednosti.
    \end{dokaz}

    HOLOMORFNE ?
\newpage

\bibliographystyle{siam}
\begin{thebibliography}{9}
    \bibitem{osnova}
    Theodore W. Gamelin \emph{Complex Analysis}, Springer (2001), Chapter X, str. 274 - 288

    \bibitem{mean value p}
    Weisstein, Eric W. \emph{Mean-Value Property}, v: From MathWorld--A Wolfram Web Resource, [ogled 22. 2. 2023], dostopno na \href{https://mathworld.wolfram.com/Mean-ValueProperty.html}{https://mathworld.wolfram.com/Mean-ValueProperty.html}
\end{thebibliography}

\end{document}